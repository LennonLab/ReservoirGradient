\documentclass[]{article}
\usepackage{lmodern}
\usepackage{amssymb,amsmath}
\usepackage{ifxetex,ifluatex}
\usepackage{fixltx2e} % provides \textsubscript
\ifnum 0\ifxetex 1\fi\ifluatex 1\fi=0 % if pdftex
  \usepackage[T1]{fontenc}
  \usepackage[utf8]{inputenc}
\else % if luatex or xelatex
  \ifxetex
    \usepackage{mathspec}
  \else
    \usepackage{fontspec}
  \fi
  \defaultfontfeatures{Ligatures=TeX,Scale=MatchLowercase}
\fi
% use upquote if available, for straight quotes in verbatim environments
\IfFileExists{upquote.sty}{\usepackage{upquote}}{}
% use microtype if available
\IfFileExists{microtype.sty}{%
\usepackage{microtype}
\UseMicrotypeSet[protrusion]{basicmath} % disable protrusion for tt fonts
}{}
\usepackage[margin=1in]{geometry}
\usepackage{hyperref}
\hypersetup{unicode=true,
            pdftitle={Dormancy and dispersal structure bacterial communities across ecosystem boundaries},
            pdfauthor={Nathan I. Wisnoski, Mario E. Muscarella, Megan L. Larsen, and Jay T. Lennon},
            pdfborder={0 0 0},
            breaklinks=true}
\urlstyle{same}  % don't use monospace font for urls
\usepackage{color}
\usepackage{fancyvrb}
\newcommand{\VerbBar}{|}
\newcommand{\VERB}{\Verb[commandchars=\\\{\}]}
\DefineVerbatimEnvironment{Highlighting}{Verbatim}{commandchars=\\\{\}}
% Add ',fontsize=\small' for more characters per line
\usepackage{framed}
\definecolor{shadecolor}{RGB}{248,248,248}
\newenvironment{Shaded}{\begin{snugshade}}{\end{snugshade}}
\newcommand{\AlertTok}[1]{\textcolor[rgb]{0.94,0.16,0.16}{#1}}
\newcommand{\AnnotationTok}[1]{\textcolor[rgb]{0.56,0.35,0.01}{\textbf{\textit{#1}}}}
\newcommand{\AttributeTok}[1]{\textcolor[rgb]{0.77,0.63,0.00}{#1}}
\newcommand{\BaseNTok}[1]{\textcolor[rgb]{0.00,0.00,0.81}{#1}}
\newcommand{\BuiltInTok}[1]{#1}
\newcommand{\CharTok}[1]{\textcolor[rgb]{0.31,0.60,0.02}{#1}}
\newcommand{\CommentTok}[1]{\textcolor[rgb]{0.56,0.35,0.01}{\textit{#1}}}
\newcommand{\CommentVarTok}[1]{\textcolor[rgb]{0.56,0.35,0.01}{\textbf{\textit{#1}}}}
\newcommand{\ConstantTok}[1]{\textcolor[rgb]{0.00,0.00,0.00}{#1}}
\newcommand{\ControlFlowTok}[1]{\textcolor[rgb]{0.13,0.29,0.53}{\textbf{#1}}}
\newcommand{\DataTypeTok}[1]{\textcolor[rgb]{0.13,0.29,0.53}{#1}}
\newcommand{\DecValTok}[1]{\textcolor[rgb]{0.00,0.00,0.81}{#1}}
\newcommand{\DocumentationTok}[1]{\textcolor[rgb]{0.56,0.35,0.01}{\textbf{\textit{#1}}}}
\newcommand{\ErrorTok}[1]{\textcolor[rgb]{0.64,0.00,0.00}{\textbf{#1}}}
\newcommand{\ExtensionTok}[1]{#1}
\newcommand{\FloatTok}[1]{\textcolor[rgb]{0.00,0.00,0.81}{#1}}
\newcommand{\FunctionTok}[1]{\textcolor[rgb]{0.00,0.00,0.00}{#1}}
\newcommand{\ImportTok}[1]{#1}
\newcommand{\InformationTok}[1]{\textcolor[rgb]{0.56,0.35,0.01}{\textbf{\textit{#1}}}}
\newcommand{\KeywordTok}[1]{\textcolor[rgb]{0.13,0.29,0.53}{\textbf{#1}}}
\newcommand{\NormalTok}[1]{#1}
\newcommand{\OperatorTok}[1]{\textcolor[rgb]{0.81,0.36,0.00}{\textbf{#1}}}
\newcommand{\OtherTok}[1]{\textcolor[rgb]{0.56,0.35,0.01}{#1}}
\newcommand{\PreprocessorTok}[1]{\textcolor[rgb]{0.56,0.35,0.01}{\textit{#1}}}
\newcommand{\RegionMarkerTok}[1]{#1}
\newcommand{\SpecialCharTok}[1]{\textcolor[rgb]{0.00,0.00,0.00}{#1}}
\newcommand{\SpecialStringTok}[1]{\textcolor[rgb]{0.31,0.60,0.02}{#1}}
\newcommand{\StringTok}[1]{\textcolor[rgb]{0.31,0.60,0.02}{#1}}
\newcommand{\VariableTok}[1]{\textcolor[rgb]{0.00,0.00,0.00}{#1}}
\newcommand{\VerbatimStringTok}[1]{\textcolor[rgb]{0.31,0.60,0.02}{#1}}
\newcommand{\WarningTok}[1]{\textcolor[rgb]{0.56,0.35,0.01}{\textbf{\textit{#1}}}}
\usepackage{longtable,booktabs}
\usepackage{graphicx,grffile}
\makeatletter
\def\maxwidth{\ifdim\Gin@nat@width>\linewidth\linewidth\else\Gin@nat@width\fi}
\def\maxheight{\ifdim\Gin@nat@height>\textheight\textheight\else\Gin@nat@height\fi}
\makeatother
% Scale images if necessary, so that they will not overflow the page
% margins by default, and it is still possible to overwrite the defaults
% using explicit options in \includegraphics[width, height, ...]{}
\setkeys{Gin}{width=\maxwidth,height=\maxheight,keepaspectratio}
\IfFileExists{parskip.sty}{%
\usepackage{parskip}
}{% else
\setlength{\parindent}{0pt}
\setlength{\parskip}{6pt plus 2pt minus 1pt}
}
\setlength{\emergencystretch}{3em}  % prevent overfull lines
\providecommand{\tightlist}{%
  \setlength{\itemsep}{0pt}\setlength{\parskip}{0pt}}
\setcounter{secnumdepth}{0}
% Redefines (sub)paragraphs to behave more like sections
\ifx\paragraph\undefined\else
\let\oldparagraph\paragraph
\renewcommand{\paragraph}[1]{\oldparagraph{#1}\mbox{}}
\fi
\ifx\subparagraph\undefined\else
\let\oldsubparagraph\subparagraph
\renewcommand{\subparagraph}[1]{\oldsubparagraph{#1}\mbox{}}
\fi

%%% Use protect on footnotes to avoid problems with footnotes in titles
\let\rmarkdownfootnote\footnote%
\def\footnote{\protect\rmarkdownfootnote}

%%% Change title format to be more compact
\usepackage{titling}

% Create subtitle command for use in maketitle
\providecommand{\subtitle}[1]{
  \posttitle{
    \begin{center}\large#1\end{center}
    }
}

\setlength{\droptitle}{-2em}

  \title{Dormancy and dispersal structure bacterial communities across ecosystem
boundaries}
    \pretitle{\vspace{\droptitle}\centering\huge}
  \posttitle{\par}
    \author{Nathan I. Wisnoski, Mario E. Muscarella, Megan L. Larsen, and Jay T.
Lennon}
    \preauthor{\centering\large\emph}
  \postauthor{\par}
      \predate{\centering\large\emph}
  \postdate{\par}
    \date{24 August, 2019}

\usepackage{array}
\usepackage{graphics}

\begin{document}
\maketitle

\hypertarget{initial-setup}{%
\section{Initial Setup}\label{initial-setup}}

First, we'll load the packages we'll need for the analysis, as well as
some other functions.

\begin{Shaded}
\begin{Highlighting}[]
\CommentTok{# Import Required Packages}
\KeywordTok{library}\NormalTok{(}\StringTok{"png"}\NormalTok{)}
\KeywordTok{library}\NormalTok{(}\StringTok{"grid"}\NormalTok{)}
\KeywordTok{library}\NormalTok{(}\StringTok{"tidyverse"}\NormalTok{)   }
\KeywordTok{library}\NormalTok{(}\StringTok{"vegan"}\NormalTok{)}
\CommentTok{#library("xtable")}
\KeywordTok{library}\NormalTok{(}\StringTok{"viridis"}\NormalTok{)}
\KeywordTok{library}\NormalTok{(}\StringTok{"cowplot"}\NormalTok{)}
\CommentTok{#library("adespatial")}
\KeywordTok{library}\NormalTok{(}\StringTok{"ggrepel"}\NormalTok{)}
\CommentTok{#library("gganimate")}
\CommentTok{#library("maps")}
\CommentTok{#library("rgdal")}
\KeywordTok{library}\NormalTok{(}\StringTok{"iNEXT"}\NormalTok{)}
\CommentTok{#library("officer")}
\CommentTok{#library("flextable") #must have gdtools installed also}
\KeywordTok{library}\NormalTok{(}\StringTok{"broom"}\NormalTok{)}
\KeywordTok{library}\NormalTok{(}\StringTok{"ggpmisc"}\NormalTok{)}
\KeywordTok{library}\NormalTok{(}\StringTok{"pander"}\NormalTok{)}
\KeywordTok{library}\NormalTok{(}\StringTok{"lubridate"}\NormalTok{)}

\KeywordTok{source}\NormalTok{(}\StringTok{"bin/mothur_tools.R"}\NormalTok{)}
\NormalTok{se <-}\StringTok{ }\ControlFlowTok{function}\NormalTok{(x, ...)\{}\KeywordTok{sd}\NormalTok{(x, }\DataTypeTok{na.rm =} \OtherTok{TRUE}\NormalTok{)}\OperatorTok{/}\KeywordTok{sqrt}\NormalTok{(}\KeywordTok{length}\NormalTok{(}\KeywordTok{na.omit}\NormalTok{(x)))\}}
\end{Highlighting}
\end{Shaded}

Next, we'll set the aesthetics of the figures we will produce.

\begin{Shaded}
\begin{Highlighting}[]
\NormalTok{my.cols <-}\StringTok{ }\NormalTok{RColorBrewer}\OperatorTok{::}\KeywordTok{brewer.pal}\NormalTok{(}\DataTypeTok{n =} \DecValTok{4}\NormalTok{, }\DataTypeTok{name =} \StringTok{"Greys"}\NormalTok{)[}\DecValTok{3}\OperatorTok{:}\DecValTok{4}\NormalTok{]}

\CommentTok{# Set theme for figures in the paper}
\KeywordTok{theme_set}\NormalTok{(}\KeywordTok{theme_classic}\NormalTok{() }\OperatorTok{+}\StringTok{ }
\StringTok{  }\KeywordTok{theme}\NormalTok{(}\DataTypeTok{axis.title =} \KeywordTok{element_text}\NormalTok{(}\DataTypeTok{size =} \DecValTok{16}\NormalTok{),}
        \DataTypeTok{axis.title.x =} \KeywordTok{element_text}\NormalTok{(}\DataTypeTok{margin =} \KeywordTok{margin}\NormalTok{(}\DataTypeTok{t =} \DecValTok{15}\NormalTok{, }\DataTypeTok{b =} \DecValTok{15}\NormalTok{)),}
        \DataTypeTok{axis.title.y =} \KeywordTok{element_text}\NormalTok{(}\DataTypeTok{margin =} \KeywordTok{margin}\NormalTok{(}\DataTypeTok{l =} \DecValTok{15}\NormalTok{, }\DataTypeTok{r =} \DecValTok{15}\NormalTok{)),}
        \DataTypeTok{axis.text =} \KeywordTok{element_text}\NormalTok{(}\DataTypeTok{size =} \DecValTok{14}\NormalTok{),}
        \DataTypeTok{axis.text.x =} \KeywordTok{element_text}\NormalTok{(}\DataTypeTok{margin =} \KeywordTok{margin}\NormalTok{(}\DataTypeTok{t =} \DecValTok{5}\NormalTok{)),}
        \DataTypeTok{axis.text.y =} \KeywordTok{element_text}\NormalTok{(}\DataTypeTok{margin =} \KeywordTok{margin}\NormalTok{(}\DataTypeTok{r =} \DecValTok{5}\NormalTok{)),}
        \CommentTok{#axis.line.x = element_line(size = 1),}
        \CommentTok{#axis.line.y = element_line(size = 1),}
        \DataTypeTok{axis.line.x =} \KeywordTok{element_blank}\NormalTok{(),}
        \DataTypeTok{axis.line.y =} \KeywordTok{element_blank}\NormalTok{(),}
        \DataTypeTok{axis.ticks.x =} \KeywordTok{element_line}\NormalTok{(}\DataTypeTok{size =} \DecValTok{1}\NormalTok{),}
        \DataTypeTok{axis.ticks.y =} \KeywordTok{element_line}\NormalTok{(}\DataTypeTok{size =} \DecValTok{1}\NormalTok{),}
        \DataTypeTok{axis.ticks.length =} \KeywordTok{unit}\NormalTok{(.}\DecValTok{1}\NormalTok{, }\StringTok{"in"}\NormalTok{),}
        \DataTypeTok{panel.border =} \KeywordTok{element_rect}\NormalTok{(}\DataTypeTok{color =} \StringTok{"black"}\NormalTok{, }\DataTypeTok{fill =} \OtherTok{NA}\NormalTok{, }\DataTypeTok{size =} \FloatTok{1.5}\NormalTok{),}
        \DataTypeTok{legend.title =} \KeywordTok{element_blank}\NormalTok{(),}
        \DataTypeTok{legend.text =} \KeywordTok{element_text}\NormalTok{(}\DataTypeTok{size =} \DecValTok{14}\NormalTok{),}
        \DataTypeTok{strip.text =} \KeywordTok{element_text}\NormalTok{(}\DataTypeTok{size =} \DecValTok{14}\NormalTok{),}
        \DataTypeTok{strip.background =} \KeywordTok{element_blank}\NormalTok{()}
\NormalTok{        ))}
\end{Highlighting}
\end{Shaded}

\hypertarget{import-data}{%
\subsection{Import Data}\label{import-data}}

Here, we read in the processed sequence files from mothur (shared and
taxonomy) and a design of the sampling. We also load in the
environmental data. We then remove the mock community from the dataset
and ensure the the design and OTU table are aligned by row.

\begin{Shaded}
\begin{Highlighting}[]
\CommentTok{# Define Inputs}
\CommentTok{# Design = general design file for experiment}
\CommentTok{# shared = OTU table from mothur with sequence similarity clustering}
\CommentTok{# Taxonomy = Taxonomic information for each OTU}
\NormalTok{design <-}\StringTok{ "data/UL.design.txt"}
\NormalTok{shared <-}\StringTok{ "data/ul_resgrad.trim.contigs.good.unique.good.filter.unique.precluster.pick.pick.pick.opti_mcc.shared"}
\NormalTok{taxon  <-}\StringTok{ "data/ul_resgrad.trim.contigs.good.unique.good.filter.unique.precluster.pick.pick.pick.opti_mcc.0.03.cons.taxonomy"}

\CommentTok{# Import Design}
\NormalTok{design <-}\StringTok{ }\KeywordTok{read.delim}\NormalTok{(design, }\DataTypeTok{header=}\NormalTok{T, }\DataTypeTok{row.names=}\DecValTok{1}\NormalTok{)}

\CommentTok{# Import Shared Files}
\NormalTok{OTUs <-}\StringTok{ }\KeywordTok{read.otu}\NormalTok{(}\DataTypeTok{shared =}\NormalTok{ shared, }\DataTypeTok{cutoff =} \StringTok{"0.03"}\NormalTok{)    }\CommentTok{# 97% Similarity}

\CommentTok{# Import Taxonomy}
\NormalTok{OTU.tax <-}\StringTok{ }\KeywordTok{read.tax}\NormalTok{(}\DataTypeTok{taxonomy =}\NormalTok{ taxon, }\DataTypeTok{format =} \StringTok{"rdp"}\NormalTok{)}

\CommentTok{# Load environmental data}
\NormalTok{env.dat <-}\StringTok{ }\KeywordTok{read.csv}\NormalTok{(}\StringTok{"data/ResGrad_EnvDat.csv"}\NormalTok{, }\DataTypeTok{header =} \OtherTok{TRUE}\NormalTok{)}
\NormalTok{env.dat <-}\StringTok{ }\NormalTok{env.dat[}\OperatorTok{-}\KeywordTok{c}\NormalTok{(}\DecValTok{16}\NormalTok{,}\DecValTok{17}\NormalTok{,}\DecValTok{18}\NormalTok{),]}

\CommentTok{# Subset to just the reservoir gradient sites}
\NormalTok{OTUs <-}\StringTok{ }\NormalTok{OTUs[}\KeywordTok{str_which}\NormalTok{(}\KeywordTok{rownames}\NormalTok{(OTUs), }\StringTok{"RG"}\NormalTok{),]}
\NormalTok{OTUs <-}\StringTok{ }\NormalTok{OTUs[}\OperatorTok{-}\KeywordTok{which}\NormalTok{(}\KeywordTok{rownames}\NormalTok{(OTUs) }\OperatorTok{==}\StringTok{ "RGMockComm"}\NormalTok{),]}

\CommentTok{# make sure OTU table matches up with design order}
\NormalTok{design <-}\StringTok{ }\NormalTok{design[}\OperatorTok{-}\KeywordTok{c}\NormalTok{(}\DecValTok{34}\OperatorTok{:}\DecValTok{39}\NormalTok{),]}
\NormalTok{OTUs <-}\StringTok{ }\NormalTok{OTUs[}\KeywordTok{match}\NormalTok{(}\KeywordTok{rownames}\NormalTok{(design), }\KeywordTok{rownames}\NormalTok{(OTUs)),]}
\NormalTok{design}\OperatorTok{$}\NormalTok{distance <-}\StringTok{ }\KeywordTok{max}\NormalTok{(}\KeywordTok{na.omit}\NormalTok{(design}\OperatorTok{$}\NormalTok{distance)) }\OperatorTok{-}\StringTok{ }\NormalTok{design}\OperatorTok{$}\NormalTok{distance}
\NormalTok{env.dat}\OperatorTok{$}\NormalTok{distance <-}\StringTok{ }\KeywordTok{max}\NormalTok{(}\KeywordTok{na.omit}\NormalTok{(env.dat}\OperatorTok{$}\NormalTok{dist.dam)) }\OperatorTok{-}\StringTok{ }\NormalTok{env.dat}\OperatorTok{$}\NormalTok{dist.dam}
\end{Highlighting}
\end{Shaded}

\hypertarget{clean-and-transform-otu-table}{%
\subsection{Clean and transform OTU
table}\label{clean-and-transform-otu-table}}

Here, we remove OTUs with low incidence across sites, we remove any
samples with low coverage, and we standardize the OTU table by
log-transforming the abundances and relativizing by site.

\begin{Shaded}
\begin{Highlighting}[]
\CommentTok{# Remove OTUs with less than two occurences across all sites}
\CommentTok{#OTUs <- OTUs[, which(colSums(OTUs) >= 2)]}

\CommentTok{# Sequencing Coverage}
\NormalTok{coverage <-}\StringTok{ }\KeywordTok{rowSums}\NormalTok{(OTUs)}

\CommentTok{# Remove Low Coverage Samples (This code removes two sites: Site 5DNA, Site 6cDNA)}
\NormalTok{lows <-}\StringTok{ }\KeywordTok{which}\NormalTok{(coverage }\OperatorTok{<}\StringTok{ }\DecValTok{10000}\NormalTok{)}
\NormalTok{OTUs <-}\StringTok{ }\NormalTok{OTUs[}\OperatorTok{-}\KeywordTok{which}\NormalTok{(coverage }\OperatorTok{<}\StringTok{ }\DecValTok{10000}\NormalTok{), ]}
\NormalTok{design <-}\StringTok{ }\NormalTok{design[}\OperatorTok{-}\KeywordTok{which}\NormalTok{(coverage }\OperatorTok{<}\StringTok{ }\DecValTok{10000}\NormalTok{), ]}
\NormalTok{otus.for.inext <-}\StringTok{  }\KeywordTok{t}\NormalTok{(OTUs)}
\CommentTok{# Remove OTUs with < 2 occurences across all sites}
\NormalTok{OTUs <-}\StringTok{ }\NormalTok{OTUs[, }\KeywordTok{which}\NormalTok{(}\KeywordTok{colSums}\NormalTok{(OTUs) }\OperatorTok{>=}\StringTok{ }\DecValTok{2}\NormalTok{)]}
\NormalTok{coverage <-}\StringTok{ }\KeywordTok{rowSums}\NormalTok{(OTUs)}
\KeywordTok{set.seed}\NormalTok{(}\DecValTok{47405}\NormalTok{)}
\NormalTok{OTUs <-}\StringTok{ }\KeywordTok{rrarefy}\NormalTok{(OTUs, }\KeywordTok{min}\NormalTok{(coverage))}

\CommentTok{# Make Relative Abundance Matrices}
\NormalTok{OTUsREL <-}\StringTok{ }\KeywordTok{decostand}\NormalTok{(OTUs, }\DataTypeTok{method =} \StringTok{"total"}\NormalTok{)}

\CommentTok{# Log Transform Relative Abundances}
\NormalTok{OTUsREL.log <-}\StringTok{ }\KeywordTok{decostand}\NormalTok{(OTUs, }\DataTypeTok{method =} \StringTok{"log"}\NormalTok{)}
\end{Highlighting}
\end{Shaded}

\hypertarget{reservoir-environmental-gradients}{%
\section{Reservoir environmental
gradients}\label{reservoir-environmental-gradients}}

Just to see if there are any strong underlying resource or nutrient
gradients in the reservoir, we'll plot them along the distance of the
reservoir.

\begin{Shaded}
\begin{Highlighting}[]
\NormalTok{facet.labs <-}\StringTok{ }\KeywordTok{c}\NormalTok{(}\StringTok{`}\DataTypeTok{chla}\StringTok{`}\NormalTok{ =}\StringTok{ "Chlorophyll-a"}\NormalTok{,}
                \StringTok{`}\DataTypeTok{color}\StringTok{`}\NormalTok{ =}\StringTok{ "Color"}\NormalTok{,}
                \StringTok{`}\DataTypeTok{DO}\StringTok{`}\NormalTok{ =}\StringTok{ "Dissolved Oxygen"}\NormalTok{,}
                \StringTok{`}\DataTypeTok{pH}\StringTok{`}\NormalTok{ =}\StringTok{ "pH"}\NormalTok{,}
                \StringTok{`}\DataTypeTok{TP}\StringTok{`}\NormalTok{ =}\StringTok{ "Total Phosphorus"}\NormalTok{)}

\NormalTok{env.dat }\OperatorTok\StringTok{ }\KeywordTok{select}\NormalTok{(distance, DO, pH, TP, chla) }\OperatorTok\StringTok{ }
\StringTok{  }\KeywordTok{gather}\NormalTok{(variable, value, }\OperatorTok{-}\NormalTok{distance) }\OperatorTok\StringTok{ }
\StringTok{  }\KeywordTok{ggplot}\NormalTok{(}\KeywordTok{aes}\NormalTok{(}\DataTypeTok{x =}\NormalTok{ distance, }\DataTypeTok{y =}\NormalTok{ value)) }\OperatorTok{+}\StringTok{ }
\StringTok{  }\KeywordTok{geom_point}\NormalTok{() }\OperatorTok{+}\StringTok{ }
\StringTok{  }\KeywordTok{geom_smooth}\NormalTok{(}\DataTypeTok{method =} \StringTok{"loess"}\NormalTok{, }\DataTypeTok{color =} \StringTok{"black"}\NormalTok{, }\DataTypeTok{se =}\NormalTok{ F) }\OperatorTok{+}\StringTok{ }
\StringTok{  }\KeywordTok{facet_grid}\NormalTok{(variable }\OperatorTok{~}\NormalTok{., }\DataTypeTok{scales =} \StringTok{"free"}\NormalTok{, }\DataTypeTok{switch =} \StringTok{"y"}\NormalTok{, }
             \DataTypeTok{labeller =} \KeywordTok{as_labeller}\NormalTok{(facet.labs)) }\OperatorTok{+}\StringTok{ }
\StringTok{  }\KeywordTok{theme}\NormalTok{(}\DataTypeTok{strip.background =} \KeywordTok{element_blank}\NormalTok{(), }
        \DataTypeTok{strip.text =} \KeywordTok{element_text}\NormalTok{(}\DataTypeTok{size =} \DecValTok{14}\NormalTok{),}
        \DataTypeTok{strip.placement =} \StringTok{"outside"}\NormalTok{) }\OperatorTok{+}\StringTok{ }
\StringTok{  }\KeywordTok{labs}\NormalTok{(}\DataTypeTok{x =} \StringTok{"Reservoir distance (m)"}\NormalTok{,}
       \DataTypeTok{y =} \StringTok{""}\NormalTok{) }\OperatorTok{+}
\StringTok{  }\KeywordTok{scale_y_continuous}\NormalTok{() }\OperatorTok{+}
\StringTok{  }\KeywordTok{ggsave}\NormalTok{(}\StringTok{"figures/env_vars.pdf"}\NormalTok{, }\DataTypeTok{height =} \DecValTok{3}\OperatorTok{/}\DecValTok{4}\OperatorTok{*}\DecValTok{4}\OperatorTok{*}\DecValTok{3}\NormalTok{, }\DataTypeTok{width =} \DecValTok{4}\NormalTok{, }\DataTypeTok{units =} \StringTok{"in"}\NormalTok{)}
\end{Highlighting}
\end{Shaded}

\begin{center}\includegraphics{ReservoirGradient_files/figure-latex/env_plot-1} \end{center}

So, there are some weak gradients, but nothing too prevailing.

\hypertarget{analyze-diversity}{%
\section{Analyze Diversity}\label{analyze-diversity}}

Now, we will analyze the bacterial diversity in the reservoir and nearby
soils to figure out how well they support different mechanisms of
community assembly.

\hypertarget{how-does-alpha-diversity-vary-along-the-reservoir}{%
\subsection{\texorpdfstring{How does \(\alpha\)-diversity vary along the
reservoir?}{How does \textbackslash{}alpha-diversity vary along the reservoir?}}\label{how-does-alpha-diversity-vary-along-the-reservoir}}

First, we use the method of rarefaction and extrapolation developed by
Chao et al.~in the iNEXT package.

\begin{Shaded}
\begin{Highlighting}[]
\CommentTok{# Observed Richness}
\NormalTok{S.obs <-}\StringTok{ }\KeywordTok{rowSums}\NormalTok{((OTUs }\OperatorTok{>}\StringTok{ }\DecValTok{0}\NormalTok{) }\OperatorTok{*}\StringTok{ }\DecValTok{1}\NormalTok{)}

\CommentTok{# Simpson's Evenness}
\NormalTok{SimpE <-}\StringTok{ }\ControlFlowTok{function}\NormalTok{(}\DataTypeTok{x =} \StringTok{""}\NormalTok{)\{}
\NormalTok{  x <-}\StringTok{ }\KeywordTok{as.data.frame}\NormalTok{(x)}
\NormalTok{  D <-}\StringTok{ }\KeywordTok{diversity}\NormalTok{(x, }\StringTok{"inv"}\NormalTok{)}
\NormalTok{  S <-}\StringTok{ }\KeywordTok{sum}\NormalTok{((x }\OperatorTok{>}\StringTok{ }\DecValTok{0}\NormalTok{) }\OperatorTok{*}\StringTok{ }\DecValTok{1}\NormalTok{) }
\NormalTok{  E <-}\StringTok{ }\NormalTok{(D)}\OperatorTok{/}\NormalTok{S }
  \KeywordTok{return}\NormalTok{(E)}
\NormalTok{\}}
\NormalTok{simpsE <-}\StringTok{ }\KeywordTok{round}\NormalTok{(}\KeywordTok{apply}\NormalTok{(OTUs, }\DecValTok{1}\NormalTok{, SimpE), }\DecValTok{3}\NormalTok{)}
\NormalTok{shan <-}\StringTok{ }\KeywordTok{diversity}\NormalTok{(OTUs, }\DataTypeTok{index =} \StringTok{"shannon"}\NormalTok{)}
\NormalTok{exp.shan <-}\StringTok{ }\KeywordTok{exp}\NormalTok{(shan)}
\NormalTok{alpha.div <-}\StringTok{ }\KeywordTok{cbind}\NormalTok{(design, S.obs, simpsE, shan, exp.shan)}

\CommentTok{# define singleton estimator from Chiu and Chao 2016 PeerJ}
\KeywordTok{source}\NormalTok{(}\StringTok{"bin/Chao_functions.R"}\NormalTok{)}

\CommentTok{# # estimate richness}
\NormalTok{singleton.apply <-}\StringTok{ }\ControlFlowTok{function}\NormalTok{(x)\{}
  \KeywordTok{singleton.Est}\NormalTok{(x, }\StringTok{"abundance"}\NormalTok{)}\OperatorTok{$}\NormalTok{corrected.data}
\NormalTok{\}}

\CommentTok{# otus.for.inext <- apply(otus.for.inext, MARGIN = 2, singleton.apply)}
\CommentTok{# divestim <- estimateD(otus.for.inext, datatype = "abundance",}
\CommentTok{#           base = "size", conf = 0.95)}
\CommentTok{# saveRDS(divestim, file = "intermediate-data/inext-output.rda")}
\NormalTok{divestim <-}\StringTok{ }\KeywordTok{readRDS}\NormalTok{(}\StringTok{"intermediate-data/inext-output.rda"}\NormalTok{)}
\NormalTok{divestim.df <-}\StringTok{ }\NormalTok{divestim }\OperatorTok\StringTok{ }
\StringTok{ }\KeywordTok{mutate}\NormalTok{(}\DataTypeTok{habitat =} \KeywordTok{str_to_title}\NormalTok{(design[}\KeywordTok{as.character}\NormalTok{(site),}\StringTok{"type"}\NormalTok{]))}
\end{Highlighting}
\end{Shaded}

Here is the resulting curve, showing the higher diversity in soil
samples relative to the lake samples.

\begin{Shaded}
\begin{Highlighting}[]
\CommentTok{# divestim.df %>%}
\CommentTok{#   ggplot(aes(x = x, y = y,}
\CommentTok{#              ymin = y.lwr, ymax = y.upr,}
\CommentTok{#              color = habitat, fill = habitat, group = site)) +}
\CommentTok{#   geom_ribbon(data=subset(divestim.df, method == "extrapolated"), alpha = 0.3) +}
\CommentTok{#   geom_line(data=subset(divestim.df, method == "interpolated"), size = 1, alpha = .8) +}
\CommentTok{#   geom_line(alpha = 1, linetype = "dashed") +}
\CommentTok{#   scale_x_continuous(labels = scales::comma, limits = c(0, 90000)) +}
\CommentTok{#   labs(x = "Sample size", y = "Estimated richness") +}
\CommentTok{#   theme(legend.position = "none") +}
\CommentTok{#   #theme(legend.position =  c(.88,.5)) +}
\CommentTok{#   annotate(label = "Soil", size = 6, geom = "text", x = 85000, y = 5000) +}
\CommentTok{#   annotate(label = "Water", size = 6, geom = "text", x = 85000, y = 1500) +}
\CommentTok{#   scale_color_grey(end = .7) +}
\CommentTok{#   scale_fill_grey(end = .7)}
\end{Highlighting}
\end{Shaded}

Next, we'll extract the estimates for the Hill numbers at different
levels of q, which differentially weight common versus rare species.

\begin{Shaded}
\begin{Highlighting}[]
\CommentTok{# hill.estim <- divestim$AsyEst %>% filter(Diversity == "Species richness") %>% }
\CommentTok{#   left_join(rownames_to_column(alpha.div), by = c("Observed" = "S.obs")) %>% }
\CommentTok{#   select(Site, rowname, station, molecule, type, distance) %>% }
\CommentTok{#   left_join(divestim$AsyEst, by = "Site")}

\NormalTok{hill.water <-}\StringTok{ }\NormalTok{divestim.df }\OperatorTok\StringTok{ }
\StringTok{  }\KeywordTok{filter}\NormalTok{(site }\OperatorTok\StringTok{ }\KeywordTok{rownames}\NormalTok{(OTUs)) }\OperatorTok\StringTok{ }
\StringTok{  }\KeywordTok{left_join}\NormalTok{(}\KeywordTok{rownames_to_column}\NormalTok{(alpha.div, }\DataTypeTok{var =} \StringTok{"site"}\NormalTok{)) }\OperatorTok\StringTok{ }
\StringTok{  }\KeywordTok{filter}\NormalTok{(habitat }\OperatorTok{==}\StringTok{ "Water"}\NormalTok{)}
\end{Highlighting}
\end{Shaded}

\begin{verbatim}
## Warning: Column `site` joining factor and character vector, coercing into
## character vector
\end{verbatim}

\begin{Shaded}
\begin{Highlighting}[]
\NormalTok{hill.water.rich <-}\StringTok{ }\KeywordTok{subset}\NormalTok{(hill.water, order }\OperatorTok{==}\StringTok{ }\DecValTok{0}\NormalTok{)}
\NormalTok{hill.water.shan <-}\StringTok{ }\KeywordTok{subset}\NormalTok{(hill.water, order }\OperatorTok{==}\StringTok{ }\DecValTok{1}\NormalTok{)}
\NormalTok{hill.water.simp <-}\StringTok{ }\KeywordTok{subset}\NormalTok{(hill.water, order }\OperatorTok{==}\StringTok{ }\DecValTok{2}\NormalTok{)}

\NormalTok{hill.water.mod.rich <-}\StringTok{ }\KeywordTok{lm}\NormalTok{(qD }\OperatorTok{~}\StringTok{ }\NormalTok{distance }\OperatorTok{*}\StringTok{ }\NormalTok{molecule, }\DataTypeTok{data =}\NormalTok{ hill.water.rich)}
\NormalTok{hill.water.mod.shan <-}\StringTok{ }\KeywordTok{lm}\NormalTok{(qD }\OperatorTok{~}\StringTok{ }\NormalTok{distance }\OperatorTok{*}\StringTok{ }\NormalTok{molecule, }\DataTypeTok{data =}\NormalTok{ hill.water.shan)}
\NormalTok{hill.water.mod.simp <-}\StringTok{ }\KeywordTok{lm}\NormalTok{(qD }\OperatorTok{~}\StringTok{ }\NormalTok{distance }\OperatorTok{*}\StringTok{ }\NormalTok{molecule, }\DataTypeTok{data =}\NormalTok{ hill.water.simp)}

\CommentTok{# summary(hill.water.mod.rich)}
\CommentTok{# summary(hill.water.mod.shan)}
\CommentTok{# summary(hill.water.mod.simp)}

\CommentTok{# tidy up the model output}
\NormalTok{hill.water.mods <-}\StringTok{ }\KeywordTok{as_tibble}\NormalTok{(}\KeywordTok{rbind.data.frame}\NormalTok{(}
  \KeywordTok{tidy}\NormalTok{(hill.water.mod.rich) }\OperatorTok\StringTok{ }\KeywordTok{add_column}\NormalTok{(}\DataTypeTok{Diversity =} \StringTok{"Richness"}\NormalTok{),}
  \KeywordTok{tidy}\NormalTok{(hill.water.mod.shan) }\OperatorTok\StringTok{ }\KeywordTok{add_column}\NormalTok{(}\DataTypeTok{Diversity =} \StringTok{"Shannon"}\NormalTok{),}
  \KeywordTok{tidy}\NormalTok{(hill.water.mod.simp) }\OperatorTok\StringTok{ }\KeywordTok{add_column}\NormalTok{(}\DataTypeTok{Diversity =} \StringTok{"Simpson"}\NormalTok{)}
\NormalTok{))}
\end{Highlighting}
\end{Shaded}

\begin{Shaded}
\begin{Highlighting}[]
\CommentTok{# Summary table of the model results. }
\NormalTok{hill.water.mods }\OperatorTok\StringTok{ }
\StringTok{  }\KeywordTok{group_by}\NormalTok{(Diversity) }\OperatorTok\StringTok{ }
\StringTok{  }\KeywordTok{rename}\NormalTok{(}\StringTok{"Term"}\NormalTok{ =}\StringTok{ }\NormalTok{term, }
         \StringTok{"Estimate"}\NormalTok{ =}\StringTok{ }\NormalTok{estimate, }
         \StringTok{"Std. Error"}\NormalTok{ =}\StringTok{ }\NormalTok{std.error, }
         \StringTok{"Statistic"}\NormalTok{ =}\StringTok{ }\NormalTok{statistic, }
         \StringTok{"p-value"}\NormalTok{ =}\StringTok{ }\NormalTok{p.value) }\OperatorTok\StringTok{ }
\StringTok{  }\KeywordTok{select}\NormalTok{(Diversity, }\KeywordTok{everything}\NormalTok{()) }\OperatorTok\StringTok{ }
\StringTok{  }\KeywordTok{pander}\NormalTok{(}\DataTypeTok{round =} \DecValTok{4}\NormalTok{)}
\end{Highlighting}
\end{Shaded}

\begin{longtable}[]{@{}cccccc@{}}
\toprule
\begin{minipage}[b]{0.12\columnwidth}\centering
Diversity\strut
\end{minipage} & \begin{minipage}[b]{0.23\columnwidth}\centering
Term\strut
\end{minipage} & \begin{minipage}[b]{0.11\columnwidth}\centering
Estimate\strut
\end{minipage} & \begin{minipage}[b]{0.13\columnwidth}\centering
Std. Error\strut
\end{minipage} & \begin{minipage}[b]{0.12\columnwidth}\centering
Statistic\strut
\end{minipage} & \begin{minipage}[b]{0.12\columnwidth}\centering
p-value\strut
\end{minipage}\tabularnewline
\midrule
\endhead
\begin{minipage}[t]{0.12\columnwidth}\centering
Richness\strut
\end{minipage} & \begin{minipage}[t]{0.23\columnwidth}\centering
(Intercept)\strut
\end{minipage} & \begin{minipage}[t]{0.11\columnwidth}\centering
1497\strut
\end{minipage} & \begin{minipage}[t]{0.13\columnwidth}\centering
100.6\strut
\end{minipage} & \begin{minipage}[t]{0.12\columnwidth}\centering
14.88\strut
\end{minipage} & \begin{minipage}[t]{0.12\columnwidth}\centering
0\strut
\end{minipage}\tabularnewline
\begin{minipage}[t]{0.12\columnwidth}\centering
Richness\strut
\end{minipage} & \begin{minipage}[t]{0.23\columnwidth}\centering
distance\strut
\end{minipage} & \begin{minipage}[t]{0.11\columnwidth}\centering
-3.176\strut
\end{minipage} & \begin{minipage}[t]{0.13\columnwidth}\centering
0.4976\strut
\end{minipage} & \begin{minipage}[t]{0.12\columnwidth}\centering
-6.381\strut
\end{minipage} & \begin{minipage}[t]{0.12\columnwidth}\centering
0\strut
\end{minipage}\tabularnewline
\begin{minipage}[t]{0.12\columnwidth}\centering
Richness\strut
\end{minipage} & \begin{minipage}[t]{0.23\columnwidth}\centering
moleculeRNA\strut
\end{minipage} & \begin{minipage}[t]{0.11\columnwidth}\centering
-1170\strut
\end{minipage} & \begin{minipage}[t]{0.13\columnwidth}\centering
142.3\strut
\end{minipage} & \begin{minipage}[t]{0.12\columnwidth}\centering
-8.222\strut
\end{minipage} & \begin{minipage}[t]{0.12\columnwidth}\centering
0\strut
\end{minipage}\tabularnewline
\begin{minipage}[t]{0.12\columnwidth}\centering
Richness\strut
\end{minipage} & \begin{minipage}[t]{0.23\columnwidth}\centering
distance:moleculeRNA\strut
\end{minipage} & \begin{minipage}[t]{0.11\columnwidth}\centering
2.985\strut
\end{minipage} & \begin{minipage}[t]{0.13\columnwidth}\centering
0.7003\strut
\end{minipage} & \begin{minipage}[t]{0.12\columnwidth}\centering
4.263\strut
\end{minipage} & \begin{minipage}[t]{0.12\columnwidth}\centering
3e-04\strut
\end{minipage}\tabularnewline
\begin{minipage}[t]{0.12\columnwidth}\centering
Shannon\strut
\end{minipage} & \begin{minipage}[t]{0.23\columnwidth}\centering
(Intercept)\strut
\end{minipage} & \begin{minipage}[t]{0.11\columnwidth}\centering
153.7\strut
\end{minipage} & \begin{minipage}[t]{0.13\columnwidth}\centering
19.41\strut
\end{minipage} & \begin{minipage}[t]{0.12\columnwidth}\centering
7.921\strut
\end{minipage} & \begin{minipage}[t]{0.12\columnwidth}\centering
0\strut
\end{minipage}\tabularnewline
\begin{minipage}[t]{0.12\columnwidth}\centering
Shannon\strut
\end{minipage} & \begin{minipage}[t]{0.23\columnwidth}\centering
distance\strut
\end{minipage} & \begin{minipage}[t]{0.11\columnwidth}\centering
-0.2941\strut
\end{minipage} & \begin{minipage}[t]{0.13\columnwidth}\centering
0.096\strut
\end{minipage} & \begin{minipage}[t]{0.12\columnwidth}\centering
-3.062\strut
\end{minipage} & \begin{minipage}[t]{0.12\columnwidth}\centering
0.0053\strut
\end{minipage}\tabularnewline
\begin{minipage}[t]{0.12\columnwidth}\centering
Shannon\strut
\end{minipage} & \begin{minipage}[t]{0.23\columnwidth}\centering
moleculeRNA\strut
\end{minipage} & \begin{minipage}[t]{0.11\columnwidth}\centering
-123.9\strut
\end{minipage} & \begin{minipage}[t]{0.13\columnwidth}\centering
27.46\strut
\end{minipage} & \begin{minipage}[t]{0.12\columnwidth}\centering
-4.513\strut
\end{minipage} & \begin{minipage}[t]{0.12\columnwidth}\centering
1e-04\strut
\end{minipage}\tabularnewline
\begin{minipage}[t]{0.12\columnwidth}\centering
Shannon\strut
\end{minipage} & \begin{minipage}[t]{0.23\columnwidth}\centering
distance:moleculeRNA\strut
\end{minipage} & \begin{minipage}[t]{0.11\columnwidth}\centering
0.2457\strut
\end{minipage} & \begin{minipage}[t]{0.13\columnwidth}\centering
0.1352\strut
\end{minipage} & \begin{minipage}[t]{0.12\columnwidth}\centering
1.818\strut
\end{minipage} & \begin{minipage}[t]{0.12\columnwidth}\centering
0.0815\strut
\end{minipage}\tabularnewline
\begin{minipage}[t]{0.12\columnwidth}\centering
Simpson\strut
\end{minipage} & \begin{minipage}[t]{0.23\columnwidth}\centering
(Intercept)\strut
\end{minipage} & \begin{minipage}[t]{0.11\columnwidth}\centering
55.44\strut
\end{minipage} & \begin{minipage}[t]{0.13\columnwidth}\centering
6.47\strut
\end{minipage} & \begin{minipage}[t]{0.12\columnwidth}\centering
8.57\strut
\end{minipage} & \begin{minipage}[t]{0.12\columnwidth}\centering
0\strut
\end{minipage}\tabularnewline
\begin{minipage}[t]{0.12\columnwidth}\centering
Simpson\strut
\end{minipage} & \begin{minipage}[t]{0.23\columnwidth}\centering
distance\strut
\end{minipage} & \begin{minipage}[t]{0.11\columnwidth}\centering
-0.0783\strut
\end{minipage} & \begin{minipage}[t]{0.13\columnwidth}\centering
0.032\strut
\end{minipage} & \begin{minipage}[t]{0.12\columnwidth}\centering
-2.446\strut
\end{minipage} & \begin{minipage}[t]{0.12\columnwidth}\centering
0.0221\strut
\end{minipage}\tabularnewline
\begin{minipage}[t]{0.12\columnwidth}\centering
Simpson\strut
\end{minipage} & \begin{minipage}[t]{0.23\columnwidth}\centering
moleculeRNA\strut
\end{minipage} & \begin{minipage}[t]{0.11\columnwidth}\centering
-36.78\strut
\end{minipage} & \begin{minipage}[t]{0.13\columnwidth}\centering
9.151\strut
\end{minipage} & \begin{minipage}[t]{0.12\columnwidth}\centering
-4.019\strut
\end{minipage} & \begin{minipage}[t]{0.12\columnwidth}\centering
5e-04\strut
\end{minipage}\tabularnewline
\begin{minipage}[t]{0.12\columnwidth}\centering
Simpson\strut
\end{minipage} & \begin{minipage}[t]{0.23\columnwidth}\centering
distance:moleculeRNA\strut
\end{minipage} & \begin{minipage}[t]{0.11\columnwidth}\centering
0.0402\strut
\end{minipage} & \begin{minipage}[t]{0.13\columnwidth}\centering
0.045\strut
\end{minipage} & \begin{minipage}[t]{0.12\columnwidth}\centering
0.8918\strut
\end{minipage} & \begin{minipage}[t]{0.12\columnwidth}\centering
0.3813\strut
\end{minipage}\tabularnewline
\bottomrule
\end{longtable}

\begin{Shaded}
\begin{Highlighting}[]
\CommentTok{# hill.estim %>% filter(type == "water") %>% }
\CommentTok{#   mutate(molecule = ifelse(molecule == "DNA", "Total", "Active")) %>% }
\CommentTok{#   ggplot(aes(x = distance, y = Estimator, }
\CommentTok{#              ymin = LCL, ymax = UCL,}
\CommentTok{#              color = molecule, fill = molecule, shape = molecule)) + }
\CommentTok{#   geom_point(size =3) + }
\CommentTok{#   # geom_errorbar(size = .5, aes(ymin = Estimator - s.e., ymax = Estimator + s.e.), }
\CommentTok{#   #                width = 10, alpha = 0.5) +}
\CommentTok{#   geom_smooth(method = "lm", aes(linetype = molecule)) +}
\CommentTok{#   labs(x = "Reservoir distance (m)",}
\CommentTok{#        y = "") +}
\CommentTok{#   scale_color_manual(values = my.cols) +}
\CommentTok{#   scale_fill_manual(values = my.cols) + }
\CommentTok{#   theme(legend.position = c(.88,.95), strip.placement = "outside",}
\CommentTok{#         strip.text = element_text(size = 16)) +}
\CommentTok{#   scale_x_reverse() +}
\CommentTok{#   facet_grid(Diversity ~ ., scales = "free", switch = "y") +}
\CommentTok{#   guides(fill = guide_legend(override.aes=list(fill=NA)))}
  \CommentTok{#facet_grid(Diversity ~ ., scales = "free")}

\CommentTok{# postitions for labels}
\NormalTok{xpos =}\StringTok{ }\KeywordTok{max}\NormalTok{((}\KeywordTok{na.omit}\NormalTok{(hill.water}\OperatorTok{$}\NormalTok{distance)))}
\NormalTok{yposDNA =}\StringTok{ }\KeywordTok{predict}\NormalTok{(hill.water.mod.rich, }\DataTypeTok{newdata =} \KeywordTok{data.frame}\NormalTok{(}\DataTypeTok{distance =} \DecValTok{0}\NormalTok{, }\DataTypeTok{molecule =} \StringTok{"DNA"}\NormalTok{))}
\NormalTok{yposRNA =}\StringTok{ }\KeywordTok{predict}\NormalTok{(hill.water.mod.rich, }\DataTypeTok{newdata =} \KeywordTok{data.frame}\NormalTok{(}\DataTypeTok{distance =} \DecValTok{0}\NormalTok{, }\DataTypeTok{molecule =} \StringTok{"RNA"}\NormalTok{))}
\NormalTok{alpha.fig <-}\StringTok{ }\NormalTok{hill.water }\OperatorTok\StringTok{ }\KeywordTok{filter}\NormalTok{(type }\OperatorTok{==}\StringTok{ "water"}\NormalTok{, order }\OperatorTok{==}\StringTok{ }\DecValTok{0}\NormalTok{) }\OperatorTok\StringTok{ }
\StringTok{  }\KeywordTok{mutate}\NormalTok{(}\DataTypeTok{molecule =} \KeywordTok{ifelse}\NormalTok{(molecule }\OperatorTok{==}\StringTok{ "DNA"}\NormalTok{, }\StringTok{"Total"}\NormalTok{, }\StringTok{"Active"}\NormalTok{)) }\OperatorTok\StringTok{ }
\StringTok{  }\KeywordTok{ggplot}\NormalTok{(}\KeywordTok{aes}\NormalTok{(}\DataTypeTok{x =}\NormalTok{ distance, }\DataTypeTok{y =}\NormalTok{ qD, }
             \DataTypeTok{ymin =}\NormalTok{ qD.LCL, }\DataTypeTok{ymax =}\NormalTok{ qD.UCL,}
             \DataTypeTok{shape =}\NormalTok{ molecule)) }\OperatorTok{+}\StringTok{ }
\StringTok{  }\CommentTok{# geom_errorbar(size = .5, width = 10, alpha = 0.5) +}
\StringTok{  }\KeywordTok{geom_smooth}\NormalTok{(}\DataTypeTok{method =} \StringTok{"lm"}\NormalTok{, }\KeywordTok{aes}\NormalTok{(}\DataTypeTok{linetype =}\NormalTok{ molecule), }\DataTypeTok{color =} \StringTok{"black"}\NormalTok{) }\OperatorTok{+}
\StringTok{  }\KeywordTok{geom_point}\NormalTok{(}\DataTypeTok{size =}\DecValTok{3}\NormalTok{, }\DataTypeTok{alpha =} \FloatTok{0.8}\NormalTok{) }\OperatorTok{+}\StringTok{ }
\StringTok{  }\KeywordTok{labs}\NormalTok{(}\DataTypeTok{x =} \StringTok{"Reservoir distance (m)"}\NormalTok{,}
       \DataTypeTok{y =} \StringTok{"Estimated richness"}\NormalTok{) }\OperatorTok{+}
\StringTok{  }\KeywordTok{scale_y_continuous}\NormalTok{(}\DataTypeTok{breaks =} \KeywordTok{seq}\NormalTok{(}\DecValTok{0}\NormalTok{, }\DecValTok{2000}\NormalTok{, }\DataTypeTok{by =} \DecValTok{500}\NormalTok{)) }\OperatorTok{+}
\StringTok{  }\KeywordTok{scale_x_continuous}\NormalTok{(}\DataTypeTok{limits =} \KeywordTok{c}\NormalTok{(}\OperatorTok{-}\DecValTok{49}\NormalTok{, }\DecValTok{350}\NormalTok{)) }\OperatorTok{+}
\StringTok{  }\KeywordTok{theme}\NormalTok{(}\DataTypeTok{legend.position =} \StringTok{"none"}\NormalTok{) }\OperatorTok{+}
\StringTok{  }\KeywordTok{guides}\NormalTok{(}\DataTypeTok{fill =} \KeywordTok{guide_legend}\NormalTok{(}\DataTypeTok{override.aes=}\KeywordTok{list}\NormalTok{(}\DataTypeTok{fill=}\OtherTok{NA}\NormalTok{))) }\OperatorTok{+}
\StringTok{  }\KeywordTok{annotate}\NormalTok{(}\StringTok{"text"}\NormalTok{, }\DataTypeTok{x =} \DecValTok{-33}\NormalTok{, }\DataTypeTok{y =}\NormalTok{ yposRNA , }
           \DataTypeTok{label =} \StringTok{"Active"}\NormalTok{, }\DataTypeTok{size =} \DecValTok{5}\NormalTok{) }\OperatorTok{+}
\StringTok{  }\KeywordTok{annotate}\NormalTok{(}\StringTok{"text"}\NormalTok{, }\DataTypeTok{x =} \DecValTok{-33}\NormalTok{, }\DataTypeTok{y =}\NormalTok{ yposDNA , }
           \DataTypeTok{label =} \StringTok{"Total"}\NormalTok{, }\DataTypeTok{size =} \DecValTok{5}\NormalTok{) }\OperatorTok{+}
\StringTok{  }\KeywordTok{annotate}\NormalTok{(}\DataTypeTok{geom =} \StringTok{"text"}\NormalTok{, }\DataTypeTok{x =}\NormalTok{ xpos, }\DataTypeTok{y =} \DecValTok{2000}\NormalTok{, }\DataTypeTok{hjust =} \DecValTok{1}\NormalTok{, }\DataTypeTok{vjust =} \DecValTok{1}\NormalTok{, }\DataTypeTok{size =} \DecValTok{5}\NormalTok{,}
           \DataTypeTok{label =} \KeywordTok{paste0}\NormalTok{(}\StringTok{"r^2== "}\NormalTok{,}\KeywordTok{round}\NormalTok{(}\KeywordTok{summary}\NormalTok{(hill.water.mod.rich)}\OperatorTok{$}\NormalTok{r.squared, }\DecValTok{2}\NormalTok{)), }\DataTypeTok{parse =}\NormalTok{ T) }\OperatorTok{+}
\StringTok{  }\KeywordTok{ggsave}\NormalTok{(}\StringTok{"figures/alpha_fig.pdf"}\NormalTok{)}
\NormalTok{alpha.fig}
\end{Highlighting}
\end{Shaded}

\begin{center}\includegraphics{ReservoirGradient_files/figure-latex/hill_div_plot-1} \end{center}

So, from the basis of these results, we can make the following
conclusions. First, we note that diversity in the total community decays
from the stream inlet to the dam of the reservoir. That is, all the
lines have a negative slope. However, we do not see this decay in the
metabolically active community. Second, we note that the metabolically
actively community has much lower diversity than the total community
near the soils, but this difference decreases toward the dam. Last,
because we quantified diversity across three orders of Hill numbers (q =
0, 1, and 2), we can also say something about the relative importance of
rare versus common taxa along the reservoir transect. We see the the
significance of the distance-by-molecule interaction term decrease as
rare taxa are downweighted in favor of common taxa. This suggests that
the differences between the active and total communities along the
transect is driven primarily by rare taxa. However, the general trend of
higher Simpson diversity across the whole transect suggests that
low-activity, but relatively common, taxa are maintained in the
reservoir.

\hypertarget{similarity-to-terrestrial-habitat-across-gradient-terrestrial-influence}{%
\subsection{Similarity To Terrestrial Habitat Across Gradient
(Terrestrial
Influence)}\label{similarity-to-terrestrial-habitat-across-gradient-terrestrial-influence}}

Here, we fit a linear model to the similarity of the aquatic community
to the soil community.

\begin{Shaded}
\begin{Highlighting}[]
\CommentTok{# Similarity to Soil Sample}
\NormalTok{UL.bray      <-}\StringTok{ }\DecValTok{1}\OperatorTok{-}\KeywordTok{as.matrix}\NormalTok{(}\KeywordTok{vegdist}\NormalTok{(OTUsREL.log, }\DataTypeTok{method=}\StringTok{"bray"}\NormalTok{))}
\NormalTok{UL.bray.lake <-}\StringTok{ }\NormalTok{UL.bray[}\OperatorTok{-}\KeywordTok{c}\NormalTok{(}\DecValTok{1}\OperatorTok{:}\DecValTok{3}\NormalTok{), }\DecValTok{1}\OperatorTok{:}\DecValTok{3}\NormalTok{] }
\NormalTok{bray.mean    <-}\StringTok{ }\KeywordTok{round}\NormalTok{(}\KeywordTok{apply}\NormalTok{(UL.bray.lake, }\DecValTok{1}\NormalTok{, mean), }\DecValTok{3}\NormalTok{)}
\NormalTok{bray.se      <-}\StringTok{ }\KeywordTok{round}\NormalTok{(}\KeywordTok{apply}\NormalTok{(UL.bray.lake, }\DecValTok{1}\NormalTok{, se), }\DecValTok{3}\NormalTok{)}
\NormalTok{UL.sim       <-}\StringTok{ }\KeywordTok{cbind}\NormalTok{(design[}\OperatorTok{-}\KeywordTok{c}\NormalTok{(}\DecValTok{1}\OperatorTok{:}\DecValTok{3}\NormalTok{), ], bray.mean, bray.se)}

\CommentTok{# Calculate Linear Model}
\NormalTok{model.terr <-}\StringTok{ }\KeywordTok{lm}\NormalTok{(bray.mean }\OperatorTok{~}\StringTok{ }\NormalTok{distance }\OperatorTok{*}\StringTok{ }\NormalTok{molecule, }\DataTypeTok{data =}\NormalTok{ UL.sim)}
\KeywordTok{predict}\NormalTok{(model.terr, }\DataTypeTok{newdata =} \KeywordTok{data.frame}\NormalTok{(}\DataTypeTok{distance =} \DecValTok{0}\NormalTok{, }\DataTypeTok{molecule =} \KeywordTok{c}\NormalTok{(}\StringTok{"RNA"}\NormalTok{, }\StringTok{"DNA"}\NormalTok{)))}
\end{Highlighting}
\end{Shaded}

\begin{verbatim}
##          1          2 
## 0.03090104 0.16890225
\end{verbatim}

\begin{Shaded}
\begin{Highlighting}[]
\KeywordTok{pander}\NormalTok{(model.terr)}
\end{Highlighting}
\end{Shaded}

\begin{longtable}[]{@{}ccccc@{}}
\caption{Fitting linear model: bray.mean \textasciitilde{} distance *
molecule}\tabularnewline
\toprule
\begin{minipage}[b]{0.31\columnwidth}\centering
~\strut
\end{minipage} & \begin{minipage}[b]{0.15\columnwidth}\centering
Estimate\strut
\end{minipage} & \begin{minipage}[b]{0.15\columnwidth}\centering
Std. Error\strut
\end{minipage} & \begin{minipage}[b]{0.11\columnwidth}\centering
t value\strut
\end{minipage} & \begin{minipage}[b]{0.14\columnwidth}\centering
Pr(\textgreater{}\textbar{}t\textbar{})\strut
\end{minipage}\tabularnewline
\midrule
\endfirsthead
\toprule
\begin{minipage}[b]{0.31\columnwidth}\centering
~\strut
\end{minipage} & \begin{minipage}[b]{0.15\columnwidth}\centering
Estimate\strut
\end{minipage} & \begin{minipage}[b]{0.15\columnwidth}\centering
Std. Error\strut
\end{minipage} & \begin{minipage}[b]{0.11\columnwidth}\centering
t value\strut
\end{minipage} & \begin{minipage}[b]{0.14\columnwidth}\centering
Pr(\textgreater{}\textbar{}t\textbar{})\strut
\end{minipage}\tabularnewline
\midrule
\endhead
\begin{minipage}[t]{0.31\columnwidth}\centering
\textbf{(Intercept)}\strut
\end{minipage} & \begin{minipage}[t]{0.15\columnwidth}\centering
0.1689\strut
\end{minipage} & \begin{minipage}[t]{0.15\columnwidth}\centering
0.01475\strut
\end{minipage} & \begin{minipage}[t]{0.11\columnwidth}\centering
11.45\strut
\end{minipage} & \begin{minipage}[t]{0.14\columnwidth}\centering
3.279e-11\strut
\end{minipage}\tabularnewline
\begin{minipage}[t]{0.31\columnwidth}\centering
\textbf{distance}\strut
\end{minipage} & \begin{minipage}[t]{0.15\columnwidth}\centering
-0.0004087\strut
\end{minipage} & \begin{minipage}[t]{0.15\columnwidth}\centering
7.298e-05\strut
\end{minipage} & \begin{minipage}[t]{0.11\columnwidth}\centering
-5.6\strut
\end{minipage} & \begin{minipage}[t]{0.14\columnwidth}\centering
9.19e-06\strut
\end{minipage}\tabularnewline
\begin{minipage}[t]{0.31\columnwidth}\centering
\textbf{moleculeRNA}\strut
\end{minipage} & \begin{minipage}[t]{0.15\columnwidth}\centering
-0.138\strut
\end{minipage} & \begin{minipage}[t]{0.15\columnwidth}\centering
0.02087\strut
\end{minipage} & \begin{minipage}[t]{0.11\columnwidth}\centering
-6.614\strut
\end{minipage} & \begin{minipage}[t]{0.14\columnwidth}\centering
7.688e-07\strut
\end{minipage}\tabularnewline
\begin{minipage}[t]{0.31\columnwidth}\centering
\textbf{distance:moleculeRNA}\strut
\end{minipage} & \begin{minipage}[t]{0.15\columnwidth}\centering
0.0003938\strut
\end{minipage} & \begin{minipage}[t]{0.15\columnwidth}\centering
0.0001027\strut
\end{minipage} & \begin{minipage}[t]{0.11\columnwidth}\centering
3.834\strut
\end{minipage} & \begin{minipage}[t]{0.14\columnwidth}\centering
0.0007998\strut
\end{minipage}\tabularnewline
\bottomrule
\end{longtable}

\begin{Shaded}
\begin{Highlighting}[]
\CommentTok{# # Calculate Confidance Intervals of Model}
\CommentTok{# newdata.terr <- data.frame(cbind(UL.sim$molecule, UL.sim$distance))}
\CommentTok{# conf95.terr <- predict(model.terr, newdata.terr, interval="confidence")}
\CommentTok{# }
\CommentTok{# # Dummy Variables Regression Model ("Terrestrial Influence")}
\CommentTok{# D2 <- (UL.sim$molecule == "RNA")*1}
\CommentTok{# fit.Fig.3b <- lm(UL.sim$bray.mean ~ UL.sim$distance + D2 + UL.sim$distance*D2)}
\CommentTok{# D2.R2 <- round(summary(fit.Fig.3b)$r.squared, 2)}
\CommentTok{# summary(fit.Fig.3b)}
\CommentTok{# }
\CommentTok{# }
\CommentTok{# DNA.int.3b <- fit.Fig.3b$coefficients[1]}
\CommentTok{# DNA.slp.3b <- fit.Fig.3b$coefficients[2]}
\CommentTok{# RNA.int.3b <- DNA.int.3b + fit.Fig.3b$coefficients[3]}
\CommentTok{# RNA.slp.3b <- DNA.slp.3b + fit.Fig.3b$coefficients[4]}
\end{Highlighting}
\end{Shaded}

\begin{Shaded}
\begin{Highlighting}[]
\NormalTok{ypred.act <-}\StringTok{ }\KeywordTok{predict}\NormalTok{(model.terr, }\DataTypeTok{newdata =} \KeywordTok{data.frame}\NormalTok{(}\DataTypeTok{distance =} \DecValTok{0}\NormalTok{, }\DataTypeTok{molecule =} \StringTok{"RNA"}\NormalTok{))}
\NormalTok{ypred.tot <-}\StringTok{ }\KeywordTok{predict}\NormalTok{(model.terr, }\DataTypeTok{newdata =} \KeywordTok{data.frame}\NormalTok{(}\DataTypeTok{distance =} \DecValTok{0}\NormalTok{, }\DataTypeTok{molecule =} \StringTok{"DNA"}\NormalTok{))}
\NormalTok{similarity.plot <-}\StringTok{ }\NormalTok{UL.sim }\OperatorTok\StringTok{ }
\StringTok{  }\KeywordTok{mutate}\NormalTok{(}\DataTypeTok{molecule =} \KeywordTok{ifelse}\NormalTok{(UL.sim}\OperatorTok{$}\NormalTok{molecule }\OperatorTok{==}\StringTok{ "DNA"}\NormalTok{, }\StringTok{"Total"}\NormalTok{, }\StringTok{"Active"}\NormalTok{)) }\OperatorTok\StringTok{ }
\StringTok{  }\KeywordTok{ggplot}\NormalTok{(}\KeywordTok{aes}\NormalTok{(}\DataTypeTok{x =}\NormalTok{ distance, }\DataTypeTok{y =}\NormalTok{ bray.mean, }\DataTypeTok{shape =}\NormalTok{ molecule)) }\OperatorTok{+}
\StringTok{  }\KeywordTok{geom_smooth}\NormalTok{(}\DataTypeTok{method =} \StringTok{"lm"}\NormalTok{, }\KeywordTok{aes}\NormalTok{(}\DataTypeTok{linetype =}\NormalTok{ molecule), }\DataTypeTok{color =} \StringTok{"black"}\NormalTok{, }\DataTypeTok{show.legend =}\NormalTok{ T) }\OperatorTok{+}\StringTok{ }
\StringTok{  }\KeywordTok{geom_point}\NormalTok{(}\DataTypeTok{alpha =} \FloatTok{0.8}\NormalTok{, }\DataTypeTok{size =} \DecValTok{3}\NormalTok{, }\DataTypeTok{show.legend =}\NormalTok{ T) }\OperatorTok{+}\StringTok{ }
\StringTok{  }\KeywordTok{labs}\NormalTok{(}\DataTypeTok{y =} \KeywordTok{str_wrap}\NormalTok{(}\StringTok{"Percent similarity to soil community"}\NormalTok{, }\DataTypeTok{width =} \DecValTok{20}\NormalTok{), }
       \DataTypeTok{x =} \StringTok{"Reservoir distance (m)"}\NormalTok{) }\OperatorTok{+}\StringTok{ }
\StringTok{  }\KeywordTok{theme}\NormalTok{(}\DataTypeTok{legend.position =} \StringTok{"none"}\NormalTok{) }\OperatorTok{+}
\StringTok{  }\KeywordTok{scale_x_continuous}\NormalTok{(}\DataTypeTok{limits =} \KeywordTok{c}\NormalTok{(}\OperatorTok{-}\DecValTok{49}\NormalTok{,}\DecValTok{350}\NormalTok{)) }\OperatorTok{+}\StringTok{ }
\StringTok{  }\KeywordTok{annotate}\NormalTok{(}\DataTypeTok{geom =} \StringTok{"text"}\NormalTok{, }\DataTypeTok{x =} \DecValTok{350}\NormalTok{, }\DataTypeTok{y =} \KeywordTok{max}\NormalTok{(UL.sim}\OperatorTok{$}\NormalTok{bray.mean), }\DataTypeTok{hjust =} \DecValTok{1}\NormalTok{, }\DataTypeTok{vjust =} \DecValTok{1}\NormalTok{, }\DataTypeTok{size =} \DecValTok{5}\NormalTok{,}
           \DataTypeTok{label =} \KeywordTok{paste0}\NormalTok{(}\StringTok{"r^2== "}\NormalTok{,}\KeywordTok{round}\NormalTok{(}\KeywordTok{summary}\NormalTok{(model.terr)}\OperatorTok{$}\NormalTok{r.squared, }\DecValTok{2}\NormalTok{)), }\DataTypeTok{parse =}\NormalTok{ T) }\OperatorTok{+}
\StringTok{  }\KeywordTok{annotate}\NormalTok{(}\StringTok{"text"}\NormalTok{, }\DataTypeTok{x =} \DecValTok{-33}\NormalTok{, }\DataTypeTok{y =}\NormalTok{ ypred.act, }\DataTypeTok{label =} \StringTok{"Active"}\NormalTok{, }\DataTypeTok{size =} \DecValTok{5}\NormalTok{) }\OperatorTok{+}
\StringTok{  }\KeywordTok{annotate}\NormalTok{(}\StringTok{"text"}\NormalTok{, }\DataTypeTok{x =} \DecValTok{-33}\NormalTok{, }\DataTypeTok{y =}\NormalTok{ ypred.tot, }\DataTypeTok{label =} \StringTok{"Total"}\NormalTok{, }\DataTypeTok{size =} \DecValTok{5}\NormalTok{) }\OperatorTok{+}
\StringTok{  }\KeywordTok{ggsave}\NormalTok{(}\StringTok{"figures/similarity_fig.pdf"}\NormalTok{)}

\NormalTok{similarity.plot}
\end{Highlighting}
\end{Shaded}

\begin{center}\includegraphics{ReservoirGradient_files/figure-latex/plot_similarity_to_soils-1} \end{center}

We find that our model captures most of the variation in community
structure \((R^2 = 0.7469401)\). We note a significant influence of
distance on community similarity and the presence of a significant
interaction between distance and whether the comparison is for active or
total bacterial communities. This indicates that total communities decay
faster with distance to soils than active communities do, which might be
explained by the large difference in initial intercept. Active
communities are always highly dissimilar to soil communities and remain
so across the lake, while total lake communities are initially similar
to soils, but this influence dissipates with distance into the
reservoir.

\hypertarget{create-combined-figure}{%
\subsubsection{Create combined figure}\label{create-combined-figure}}

\begin{Shaded}
\begin{Highlighting}[]
\KeywordTok{plot_grid}\NormalTok{(alpha.fig }\OperatorTok{+}\StringTok{ }\KeywordTok{labs}\NormalTok{(}\DataTypeTok{x =} \StringTok{""}\NormalTok{), similarity.plot, }
          \DataTypeTok{align =} \StringTok{"hv"}\NormalTok{,}
          \DataTypeTok{labels =} \StringTok{"auto"}\NormalTok{, }\DataTypeTok{ncol =} \DecValTok{1}\NormalTok{) }\OperatorTok{+}
\StringTok{  }\KeywordTok{ggsave}\NormalTok{(}\StringTok{"figures/alpha_similarity_paneled.pdf"}\NormalTok{)}
\end{Highlighting}
\end{Shaded}

\begin{center}\includegraphics{ReservoirGradient_files/figure-latex/combined-plots-1} \end{center}

\hypertarget{how-does-community-structure-change-along-the-gradient}{%
\subsection{How does community structure change along the
gradient?}\label{how-does-community-structure-change-along-the-gradient}}

First, we'll just get an overview of how the communities look along the
aquatic transect.

\begin{Shaded}
\begin{Highlighting}[]
\NormalTok{ul.pcoa <-}\StringTok{ }\KeywordTok{cmdscale}\NormalTok{(}\KeywordTok{vegdist}\NormalTok{(OTUsREL.log, }\DataTypeTok{method=}\StringTok{"bray"}\NormalTok{), }\DecValTok{2}\NormalTok{, }\DataTypeTok{eig =}\NormalTok{ T, }\DataTypeTok{add =}\NormalTok{ T)}
\NormalTok{explainvars <-}\StringTok{ }\KeywordTok{round}\NormalTok{(}\KeywordTok{eigenvals}\NormalTok{(ul.pcoa)[}\KeywordTok{c}\NormalTok{(}\DecValTok{1}\NormalTok{,}\DecValTok{2}\NormalTok{)]}\OperatorTok{/}\KeywordTok{sum}\NormalTok{(}\KeywordTok{eigenvals}\NormalTok{(ul.pcoa)),}\DecValTok{3}\NormalTok{) }\OperatorTok{*}\DecValTok{100}
\NormalTok{water.pcvals <-}\StringTok{ }\KeywordTok{data.frame}\NormalTok{(}\KeywordTok{scores}\NormalTok{(ul.pcoa)) }\OperatorTok\StringTok{ }
\StringTok{  }\KeywordTok{rownames_to_column}\NormalTok{(}\StringTok{"name"}\NormalTok{) }\OperatorTok\StringTok{ }
\StringTok{  }\KeywordTok{left_join}\NormalTok{(}\KeywordTok{rownames_to_column}\NormalTok{(design, }\StringTok{"name"}\NormalTok{)) }\OperatorTok\StringTok{ }
\StringTok{  }\KeywordTok{arrange}\NormalTok{(}\KeywordTok{desc}\NormalTok{(distance)) }\OperatorTok\StringTok{ }\KeywordTok{filter}\NormalTok{(type }\OperatorTok{==}\StringTok{ "water"}\NormalTok{)}
\NormalTok{soil.pcvals <-}\StringTok{ }\KeywordTok{data.frame}\NormalTok{(}\KeywordTok{scores}\NormalTok{(ul.pcoa)) }\OperatorTok\StringTok{ }
\StringTok{  }\KeywordTok{rownames_to_column}\NormalTok{(}\StringTok{"name"}\NormalTok{) }\OperatorTok\StringTok{ }
\StringTok{  }\KeywordTok{left_join}\NormalTok{(}\KeywordTok{rownames_to_column}\NormalTok{(design, }\StringTok{"name"}\NormalTok{)) }\OperatorTok\StringTok{ }
\StringTok{  }\KeywordTok{arrange}\NormalTok{(}\KeywordTok{desc}\NormalTok{(distance)) }\OperatorTok\StringTok{ }\KeywordTok{filter}\NormalTok{(type }\OperatorTok{==}\StringTok{ "soil"}\NormalTok{)}
\NormalTok{pc_dists <-}\StringTok{ }\KeywordTok{tibble}\NormalTok{(}
  \DataTypeTok{DNA_dim1 =} \KeywordTok{subset}\NormalTok{(water.pcvals, molecule }\OperatorTok{==}\StringTok{ "DNA"}\NormalTok{)}\OperatorTok{$}\NormalTok{Dim1,}
  \DataTypeTok{DNA_dim2 =} \KeywordTok{subset}\NormalTok{(water.pcvals, molecule }\OperatorTok{==}\StringTok{ "DNA"}\NormalTok{)}\OperatorTok{$}\NormalTok{Dim2,}
  \DataTypeTok{RNA_dim1 =} \KeywordTok{subset}\NormalTok{(water.pcvals, molecule }\OperatorTok{==}\StringTok{ "RNA"}\NormalTok{)}\OperatorTok{$}\NormalTok{Dim1,}
  \DataTypeTok{RNA_dim2 =} \KeywordTok{subset}\NormalTok{(water.pcvals, molecule }\OperatorTok{==}\StringTok{ "RNA"}\NormalTok{)}\OperatorTok{$}\NormalTok{Dim2)}

\NormalTok{pcoa.fig <-}\StringTok{ }\KeywordTok{data.frame}\NormalTok{(}\KeywordTok{scores}\NormalTok{(ul.pcoa)) }\OperatorTok\StringTok{ }
\StringTok{  }\KeywordTok{rownames_to_column}\NormalTok{(}\StringTok{"name"}\NormalTok{) }\OperatorTok\StringTok{ }
\StringTok{  }\KeywordTok{left_join}\NormalTok{(}\KeywordTok{rownames_to_column}\NormalTok{(design, }\StringTok{"name"}\NormalTok{)) }\OperatorTok\StringTok{ }
\StringTok{  }\KeywordTok{arrange}\NormalTok{(}\KeywordTok{desc}\NormalTok{(distance)) }\OperatorTok\StringTok{ }\KeywordTok{filter}\NormalTok{(type }\OperatorTok{==}\StringTok{ "water"}\NormalTok{) }\OperatorTok\StringTok{ }
\StringTok{  }\KeywordTok{mutate}\NormalTok{(}\DataTypeTok{molecule =} \KeywordTok{ifelse}\NormalTok{(molecule }\OperatorTok{==}\StringTok{ "DNA"}\NormalTok{, }\StringTok{"Total"}\NormalTok{, }\StringTok{"Active"}\NormalTok{)) }\OperatorTok\StringTok{ }
\StringTok{  }\KeywordTok{ggplot}\NormalTok{(}\KeywordTok{aes}\NormalTok{(}\DataTypeTok{x =}\NormalTok{ Dim1, }\DataTypeTok{y =}\NormalTok{ Dim2)) }\OperatorTok{+}
\StringTok{  }\KeywordTok{geom_path}\NormalTok{(}\DataTypeTok{size =} \DecValTok{1}\NormalTok{, }\DataTypeTok{alpha =} \FloatTok{0.75}\NormalTok{, }\DataTypeTok{arrow =} \KeywordTok{arrow}\NormalTok{(}\DataTypeTok{angle =} \DecValTok{20}\NormalTok{,}
                          \DataTypeTok{length =} \KeywordTok{unit}\NormalTok{(}\FloatTok{0.35}\NormalTok{, }\StringTok{"cm"}\NormalTok{),}
                          \DataTypeTok{type =} \StringTok{"closed"}\NormalTok{), }\KeywordTok{aes}\NormalTok{(}\DataTypeTok{color =}\NormalTok{ molecule, }\DataTypeTok{linetype =}\NormalTok{ molecule)) }\OperatorTok{+}
\StringTok{  }\KeywordTok{geom_point}\NormalTok{(}\DataTypeTok{size =} \DecValTok{3}\NormalTok{, }\DataTypeTok{alpha =} \FloatTok{0.8}\NormalTok{, }\KeywordTok{aes}\NormalTok{(}\DataTypeTok{color =}\NormalTok{ molecule, }\DataTypeTok{shape =}\NormalTok{ molecule)) }\OperatorTok{+}\StringTok{ }
\StringTok{  }\KeywordTok{geom_point}\NormalTok{(}\DataTypeTok{data =} \KeywordTok{select}\NormalTok{(soil.pcvals, Dim1, Dim2), }\DataTypeTok{col =} \StringTok{"black"}\NormalTok{, }\DataTypeTok{alpha =} \FloatTok{.8}\NormalTok{, }\DataTypeTok{size =} \DecValTok{3}\NormalTok{) }\OperatorTok{+}
\StringTok{  }\KeywordTok{scale_color_manual}\NormalTok{(}\StringTok{"Community Subset"}\NormalTok{, }\DataTypeTok{values =}\NormalTok{ my.cols) }\OperatorTok{+}
\StringTok{  }\KeywordTok{geom_segment}\NormalTok{(}\DataTypeTok{data =}\NormalTok{ pc_dists,}
               \KeywordTok{aes}\NormalTok{(}\DataTypeTok{x =}\NormalTok{ DNA_dim1, }\DataTypeTok{y =}\NormalTok{ DNA_dim2,}
                   \DataTypeTok{xend =}\NormalTok{ RNA_dim1, }\DataTypeTok{yend =}\NormalTok{ RNA_dim2),}
               \DataTypeTok{alpha =} \DecValTok{0}\NormalTok{) }\OperatorTok{+}
\StringTok{  }\KeywordTok{coord_fixed}\NormalTok{(}\DataTypeTok{ratio =} \DecValTok{1}\NormalTok{) }\OperatorTok{+}
\StringTok{  }\KeywordTok{labs}\NormalTok{(}\DataTypeTok{x =} \KeywordTok{paste0}\NormalTok{(}\StringTok{"PCoA 1 ("}\NormalTok{, explainvars[}\DecValTok{1}\NormalTok{],}\StringTok{"%)"}\NormalTok{),}
       \DataTypeTok{y =} \KeywordTok{paste0}\NormalTok{(}\StringTok{"PCoA 2 ("}\NormalTok{, explainvars[}\DecValTok{2}\NormalTok{],}\StringTok{"%)"}\NormalTok{)) }\OperatorTok{+}
\StringTok{  }\CommentTok{# theme(legend.position = "none") +}
\StringTok{  }\CommentTok{# annotate(geom = "text", x = .2, y = 0, label = "Active", size = 5) +}
\StringTok{  }\CommentTok{# annotate(geom = "text", x = -.25, y = -.3, label = "Total", size = 5) + }
\StringTok{  }\CommentTok{# annotate(geom = "text", x = .3, y = -.4, label = "Soils", size = 5) +}
\StringTok{  }\KeywordTok{ggsave}\NormalTok{(}\StringTok{"figures/pcoa.pdf"}\NormalTok{)}
\NormalTok{pcoa.fig}
\end{Highlighting}
\end{Shaded}

\begin{center}\includegraphics{ReservoirGradient_files/figure-latex/ordination-1} \end{center}

So, it appears that there is convergence in community structure along
the path from stream inlet to the dam. This could reflect a loss of
soil-derived taxa in the aquatic samples. To test this, we'll look at
\(\beta\)-diversity along the gradient with respect to the soil samples.
If we see a decay in similarity to soils, this suggests soil taxa are
having a comparatively lower influence with distance from the inlet.

\hypertarget{identifying-the-soil-bacteria}{%
\section{Identifying the Soil
Bacteria}\label{identifying-the-soil-bacteria}}

Now, we wish to determine whether soil-derived taxa are driving this
pattern, and then ask who these influential soil bacteria are.

To classify soil bacteria, we take an incidence-based approach and
classify OTUs as:\\
- present in the soil and present, but never active, in the reservoir\\
- present in the soil and active in the reservoir

\begin{Shaded}
\begin{Highlighting}[]
\CommentTok{# separate lake and soil samples}
\NormalTok{lake.total <-}\StringTok{ }\NormalTok{OTUs[}\KeywordTok{which}\NormalTok{(design}\OperatorTok{$}\NormalTok{molecule }\OperatorTok{==}\StringTok{ "DNA"}\NormalTok{, design}\OperatorTok{$}\NormalTok{type }\OperatorTok{==}\StringTok{ "water"}\NormalTok{),]}
\NormalTok{soil.total <-}\StringTok{ }\NormalTok{OTUs[}\KeywordTok{which}\NormalTok{(design}\OperatorTok{$}\NormalTok{molecule }\OperatorTok{==}\StringTok{ "DNA"}\NormalTok{, design}\OperatorTok{$}\NormalTok{type }\OperatorTok{==}\StringTok{ "soil"}\NormalTok{),]}

\CommentTok{# which otus are present in both lake and soil samples}
\NormalTok{lake.and.soil.total <-}\StringTok{ }\NormalTok{OTUs[}\KeywordTok{which}\NormalTok{(design}\OperatorTok{$}\NormalTok{molecule }\OperatorTok{==}\StringTok{ "DNA"}\NormalTok{, design}\OperatorTok{$}\NormalTok{type }\OperatorTok{==}\StringTok{ "water"}\NormalTok{),}
                            \KeywordTok{which}\NormalTok{(}\KeywordTok{colSums}\NormalTok{(lake.total) }\OperatorTok{>}\StringTok{ }\DecValTok{0} \OperatorTok{&}\StringTok{ }\KeywordTok{colSums}\NormalTok{(soil.total) }\OperatorTok{>}\StringTok{ }\DecValTok{0}\NormalTok{)]}

\CommentTok{# isolate just the dna and rna lake communities}
\NormalTok{w.dna <-}\StringTok{ }\NormalTok{OTUs[}\KeywordTok{which}\NormalTok{(design}\OperatorTok{$}\NormalTok{molecule }\OperatorTok{==}\StringTok{ "DNA"} \OperatorTok{&}\StringTok{ }\NormalTok{design}\OperatorTok{$}\NormalTok{type }\OperatorTok{==}\StringTok{ "water"}\NormalTok{), ]}
\NormalTok{w.rna <-}\StringTok{ }\NormalTok{OTUs[}\KeywordTok{which}\NormalTok{(design}\OperatorTok{$}\NormalTok{molecule }\OperatorTok{==}\StringTok{ "RNA"} \OperatorTok{&}\StringTok{ }\NormalTok{design}\OperatorTok{$}\NormalTok{type }\OperatorTok{==}\StringTok{ "water"}\NormalTok{), ]}

\CommentTok{# pull out the lake rna counts for otus found in lake and soil}
\NormalTok{lake.and.soil.act <-}\StringTok{ }\NormalTok{w.rna[,}\KeywordTok{colnames}\NormalTok{(lake.and.soil.total)]}

\CommentTok{# of these lake and soil taxa, which are never active? active?}
\NormalTok{nvr.act <-}\StringTok{ }\KeywordTok{which}\NormalTok{(}\KeywordTok{colSums}\NormalTok{(lake.and.soil.act) }\OperatorTok{==}\StringTok{ }\DecValTok{0}\NormalTok{)}
\NormalTok{yes.act <-}\StringTok{ }\KeywordTok{which}\NormalTok{(}\KeywordTok{colSums}\NormalTok{(lake.and.soil.act) }\OperatorTok{!=}\StringTok{ }\DecValTok{0}\NormalTok{)}

\CommentTok{# how many otus are active relative to the total number of otus }
\KeywordTok{length}\NormalTok{(nvr.act) }\OperatorTok{/}\StringTok{ }\KeywordTok{ncol}\NormalTok{(lake.and.soil.total) }\CommentTok{# 88% of soil-derived bac never active}
\end{Highlighting}
\end{Shaded}

\begin{verbatim}
## [1] 0.8825537
\end{verbatim}

\begin{Shaded}
\begin{Highlighting}[]
\KeywordTok{length}\NormalTok{(yes.act) }\OperatorTok{/}\StringTok{ }\KeywordTok{ncol}\NormalTok{(soil.total) }\CommentTok{# 8% of all soil taxa were active in lake}
\end{Highlighting}
\end{Shaded}

\begin{verbatim}
## [1] 0.08102603
\end{verbatim}

\begin{Shaded}
\begin{Highlighting}[]
\CommentTok{# of taxa who were never active, what fraction of the total community did they represent?}
\KeywordTok{sum}\NormalTok{(}\KeywordTok{rowSums}\NormalTok{(w.dna[,}\KeywordTok{names}\NormalTok{(nvr.act)]))}
\end{Highlighting}
\end{Shaded}

\begin{verbatim}
## [1] 23585
\end{verbatim}

\begin{Shaded}
\begin{Highlighting}[]
\KeywordTok{sum}\NormalTok{(}\KeywordTok{rowSums}\NormalTok{(w.dna[,}\KeywordTok{names}\NormalTok{(yes.act)]))}
\end{Highlighting}
\end{Shaded}

\begin{verbatim}
## [1] 495479
\end{verbatim}

\begin{Shaded}
\begin{Highlighting}[]
\KeywordTok{sum}\NormalTok{(}\KeywordTok{rowSums}\NormalTok{(w.dna[,}\KeywordTok{names}\NormalTok{(nvr.act)])) }\OperatorTok{/}\StringTok{ }\KeywordTok{sum}\NormalTok{(}\KeywordTok{rowSums}\NormalTok{(w.dna))}
\end{Highlighting}
\end{Shaded}

\begin{verbatim}
## [1] 0.04543756
\end{verbatim}

\begin{Shaded}
\begin{Highlighting}[]
\CommentTok{# of taxa who became active, what fraction of the dna community did they represent?}
\KeywordTok{sum}\NormalTok{(}\KeywordTok{rowSums}\NormalTok{(w.dna[,}\KeywordTok{names}\NormalTok{(yes.act)])) }\OperatorTok{/}\StringTok{ }\KeywordTok{sum}\NormalTok{(}\KeywordTok{rowSums}\NormalTok{(w.dna))}
\end{Highlighting}
\end{Shaded}

\begin{verbatim}
## [1] 0.9545624
\end{verbatim}

\begin{Shaded}
\begin{Highlighting}[]
\NormalTok{prop.nvr.act <-}\StringTok{ }\KeywordTok{rowSums}\NormalTok{(w.dna[,nvr.act]) }\OperatorTok{/}\StringTok{ }\KeywordTok{rowSums}\NormalTok{(w.dna)}
\CommentTok{# cbind.data.frame(design.dna, inactive = prop.nvr.act) %>% }
\CommentTok{#   ggplot(aes(x = distance, y = inactive)) +}
\CommentTok{#   geom_point() + }
\CommentTok{#   geom_line(stat = "smooth", method = "lm", formula = y ~ x, se = F) +}
\CommentTok{#   labs(x = "Reservoir transect (m)", y = "Rel. abundance of taxa\textbackslash{}n that are never active") +}
\CommentTok{#   scale_x_reverse()}
\end{Highlighting}
\end{Shaded}

We calculate the richness of the soil taxa that are never active in the
lake. We calculate richness from the DNA-based samples.

\begin{Shaded}
\begin{Highlighting}[]
\CommentTok{# pull out their dna abundances and calculate richness}
\NormalTok{terr.lake <-}\StringTok{ }\NormalTok{w.dna[ , }\KeywordTok{c}\NormalTok{(}\KeywordTok{names}\NormalTok{(nvr.act))]}
\NormalTok{terr.rich <-}\StringTok{ }\KeywordTok{rowSums}\NormalTok{((terr.lake }\OperatorTok{>}\StringTok{ }\DecValTok{0}\NormalTok{) }\OperatorTok{*}\StringTok{ }\DecValTok{1}\NormalTok{)}
\NormalTok{terr.REL <-}\StringTok{ }\KeywordTok{rowSums}\NormalTok{(terr.lake) }\OperatorTok{/}\StringTok{ }\KeywordTok{rowSums}\NormalTok{(w.dna) }
\NormalTok{design.dna <-}\StringTok{ }\NormalTok{design[}\KeywordTok{which}\NormalTok{(design}\OperatorTok{$}\NormalTok{molecule }\OperatorTok{==}\StringTok{ "DNA"} \OperatorTok{&}\StringTok{ }\NormalTok{design}\OperatorTok{$}\NormalTok{type }\OperatorTok{==}\StringTok{ "water"}\NormalTok{), ]}
\NormalTok{terr.rich.log <-}\StringTok{ }\KeywordTok{log10}\NormalTok{(terr.rich)}
\NormalTok{terr.REL.log <-}\StringTok{ }\KeywordTok{log10}\NormalTok{(terr.REL)}

\NormalTok{terr.mod1 <-}\StringTok{ }\KeywordTok{lm}\NormalTok{(terr.rich.log }\OperatorTok{~}\StringTok{ }\NormalTok{design.dna}\OperatorTok{$}\NormalTok{distance)}
\KeywordTok{summary}\NormalTok{(terr.mod1)}
\end{Highlighting}
\end{Shaded}

\begin{verbatim}
## 
## Call:
## lm(formula = terr.rich.log ~ design.dna$distance)
## 
## Residuals:
##       Min        1Q    Median        3Q       Max 
## -0.199417 -0.123300 -0.000783  0.080926  0.234711 
## 
## Coefficients:
##                       Estimate Std. Error t value Pr(>|t|)    
## (Intercept)          3.0266909  0.0726577  41.657 2.37e-14 ***
## design.dna$distance -0.0025661  0.0003595  -7.138 1.18e-05 ***
## ---
## Signif. codes:  0 '***' 0.001 '**' 0.01 '*' 0.05 '.' 0.1 ' ' 1
## 
## Residual standard error: 0.1478 on 12 degrees of freedom
## Multiple R-squared:  0.8094, Adjusted R-squared:  0.7935 
## F-statistic: 50.95 on 1 and 12 DF,  p-value: 1.184e-05
\end{verbatim}

\begin{Shaded}
\begin{Highlighting}[]
\NormalTok{T1.R2 <-}\StringTok{ }\KeywordTok{round}\NormalTok{(}\KeywordTok{summary}\NormalTok{(terr.mod1)}\OperatorTok{$}\NormalTok{r.squared, }\DecValTok{2}\NormalTok{)}
\NormalTok{T1.int <-}\StringTok{ }\NormalTok{terr.mod1}\OperatorTok{$}\NormalTok{coefficients[}\DecValTok{1}\NormalTok{]}
\NormalTok{T1.slp <-}\StringTok{ }\NormalTok{terr.mod1}\OperatorTok{$}\NormalTok{coefficients[}\DecValTok{2}\NormalTok{]}
\KeywordTok{pander}\NormalTok{(terr.mod1)}
\end{Highlighting}
\end{Shaded}

\begin{longtable}[]{@{}ccccc@{}}
\caption{Fitting linear model: terr.rich.log \textasciitilde{}
design.dna\$distance We find distance is a highly significant predictor
of the richness of these soil-derived taxa (on a
log-scale).}\tabularnewline
\toprule
\begin{minipage}[b]{0.31\columnwidth}\centering
~\strut
\end{minipage} & \begin{minipage}[b]{0.14\columnwidth}\centering
Estimate\strut
\end{minipage} & \begin{minipage}[b]{0.15\columnwidth}\centering
Std. Error\strut
\end{minipage} & \begin{minipage}[b]{0.12\columnwidth}\centering
t value\strut
\end{minipage} & \begin{minipage}[b]{0.14\columnwidth}\centering
Pr(\textgreater{}\textbar{}t\textbar{})\strut
\end{minipage}\tabularnewline
\midrule
\endfirsthead
\toprule
\begin{minipage}[b]{0.31\columnwidth}\centering
~\strut
\end{minipage} & \begin{minipage}[b]{0.14\columnwidth}\centering
Estimate\strut
\end{minipage} & \begin{minipage}[b]{0.15\columnwidth}\centering
Std. Error\strut
\end{minipage} & \begin{minipage}[b]{0.12\columnwidth}\centering
t value\strut
\end{minipage} & \begin{minipage}[b]{0.14\columnwidth}\centering
Pr(\textgreater{}\textbar{}t\textbar{})\strut
\end{minipage}\tabularnewline
\midrule
\endhead
\begin{minipage}[t]{0.31\columnwidth}\centering
\textbf{(Intercept)}\strut
\end{minipage} & \begin{minipage}[t]{0.14\columnwidth}\centering
3.027\strut
\end{minipage} & \begin{minipage}[t]{0.15\columnwidth}\centering
0.07266\strut
\end{minipage} & \begin{minipage}[t]{0.12\columnwidth}\centering
41.66\strut
\end{minipage} & \begin{minipage}[t]{0.14\columnwidth}\centering
2.374e-14\strut
\end{minipage}\tabularnewline
\begin{minipage}[t]{0.31\columnwidth}\centering
\textbf{design.dna\$distance}\strut
\end{minipage} & \begin{minipage}[t]{0.14\columnwidth}\centering
-0.002566\strut
\end{minipage} & \begin{minipage}[t]{0.15\columnwidth}\centering
0.0003595\strut
\end{minipage} & \begin{minipage}[t]{0.12\columnwidth}\centering
-7.138\strut
\end{minipage} & \begin{minipage}[t]{0.14\columnwidth}\centering
1.184e-05\strut
\end{minipage}\tabularnewline
\bottomrule
\end{longtable}

\begin{Shaded}
\begin{Highlighting}[]
\NormalTok{transient.plot <-}\StringTok{ }\KeywordTok{tibble}\NormalTok{(}\DataTypeTok{transient_rich =}\NormalTok{ terr.rich, }\DataTypeTok{distance =}\NormalTok{ design.dna}\OperatorTok{$}\NormalTok{distance) }\OperatorTok\StringTok{ }
\StringTok{  }\KeywordTok{ggplot}\NormalTok{(}\KeywordTok{aes}\NormalTok{(}\DataTypeTok{x =}\NormalTok{ distance, }\DataTypeTok{y =}\NormalTok{ transient_rich)) }\OperatorTok{+}\StringTok{ }
\StringTok{  }\KeywordTok{geom_smooth}\NormalTok{(}\DataTypeTok{method =} \StringTok{"lm"}\NormalTok{, }\DataTypeTok{color =} \StringTok{"black"}\NormalTok{, }\DataTypeTok{fill =} \StringTok{"grey"}\NormalTok{) }\OperatorTok{+}
\StringTok{  }\KeywordTok{geom_point}\NormalTok{(}\DataTypeTok{size =} \DecValTok{3}\NormalTok{, }\DataTypeTok{alpha =} \FloatTok{.8}\NormalTok{, }\DataTypeTok{color =} \StringTok{"black"}\NormalTok{) }\OperatorTok{+}\StringTok{ }
\StringTok{  }\KeywordTok{scale_y_log10}\NormalTok{() }\OperatorTok{+}
\StringTok{  }\KeywordTok{annotation_logticks}\NormalTok{(}\DataTypeTok{sides =} \StringTok{"l"}\NormalTok{, }\DataTypeTok{size =} \DecValTok{1}\NormalTok{) }\OperatorTok{+}
\StringTok{  }\KeywordTok{labs}\NormalTok{(}\DataTypeTok{x =} \StringTok{"Reservoir distance (m)"}\NormalTok{,}
       \DataTypeTok{y =} \StringTok{"Inactive soil taxa in reservoir"}\NormalTok{) }\OperatorTok{+}
\StringTok{  }\KeywordTok{annotate}\NormalTok{(}\StringTok{"text"}\NormalTok{, }\DataTypeTok{x =} \DecValTok{350}\NormalTok{, }\DataTypeTok{y =} \KeywordTok{max}\NormalTok{(terr.rich) }\OperatorTok{+}\StringTok{ }\DecValTok{200}\NormalTok{, }\DataTypeTok{hjust =} \DecValTok{1}\NormalTok{, }\DataTypeTok{vjust =} \DecValTok{0}\NormalTok{, }\DataTypeTok{size =} \DecValTok{5}\NormalTok{,}
           \DataTypeTok{label =} \KeywordTok{paste0}\NormalTok{(}\StringTok{"r^2== "}\NormalTok{,T1.R2), }\DataTypeTok{parse =}\NormalTok{ T)  }\OperatorTok{+}
\StringTok{  }\KeywordTok{ggsave}\NormalTok{(}\StringTok{"figures/transients.pdf"}\NormalTok{)}
\NormalTok{transient.plot}
\end{Highlighting}
\end{Shaded}

\begin{center}\includegraphics{ReservoirGradient_files/figure-latex/plot_transient-1} \end{center}

\begin{Shaded}
\begin{Highlighting}[]
\CommentTok{# plot_grid(alpha.fig, }
\CommentTok{#           similarity.plot, }
\CommentTok{#           pcoa.fig + , }
\CommentTok{#           transient.plot,}
\CommentTok{#           align = "hv", axis = "tlbr",}
\CommentTok{#           labels = "auto", ncol = 2) +}
\CommentTok{#   ggsave("figures/large_panel.pdf", width = 12, height = 8)}
\end{Highlighting}
\end{Shaded}

\hypertarget{what-is-the-fate-of-soil-derived-taxa-in-the-reservoir}{%
\section{What is the fate of soil-derived taxa in the
reservoir?}\label{what-is-the-fate-of-soil-derived-taxa-in-the-reservoir}}

So, we observe that most soil-derived taxa appear to decay once they
enter the reservoir. Do any soil-derived taxa persist in the active
bacterial community of the reservoir and do they rise to high relative
abundances?

\begin{Shaded}
\begin{Highlighting}[]
\CommentTok{# identify otus in soil samples and lake samples}
\NormalTok{in.soil <-}\StringTok{ }\NormalTok{OTUs[, }\KeywordTok{which}\NormalTok{(}\KeywordTok{colSums}\NormalTok{(OTUs[}\KeywordTok{c}\NormalTok{(}\DecValTok{1}\OperatorTok{:}\DecValTok{3}\NormalTok{),]) }\OperatorTok{>}\StringTok{ }\DecValTok{0}\NormalTok{ )]}
\CommentTok{#in.lake <- OTUs[, which(colSums(OTUs[-c(1:3),]) > 0)]}

\CommentTok{# isolate just the rna water samples and convert to presence-absence}
\NormalTok{in.lake.rna <-}\StringTok{ }\NormalTok{OTUs[}\KeywordTok{which}\NormalTok{(design}\OperatorTok{$}\NormalTok{molecule }\OperatorTok{==}\StringTok{ "RNA"} \OperatorTok{&}\StringTok{ }\NormalTok{design}\OperatorTok{$}\NormalTok{type }\OperatorTok{==}\StringTok{ "water"}\NormalTok{), ]}
\NormalTok{in.lake.rna.pa <-}\StringTok{ }\NormalTok{(in.lake.rna }\OperatorTok{>}\StringTok{ }\DecValTok{0}\NormalTok{) }\OperatorTok{*}\StringTok{ }\DecValTok{1}

\CommentTok{# define the 'core' taxa as otus present in 50% of samples}
\NormalTok{in.lake.core <-}\StringTok{ }\NormalTok{w.dna[, }\KeywordTok{which}\NormalTok{((}\KeywordTok{colSums}\NormalTok{(in.lake.rna.pa) }\OperatorTok{/}\StringTok{ }\KeywordTok{nrow}\NormalTok{(in.lake.rna.pa)) }\OperatorTok{>=}\StringTok{ }\FloatTok{0.75}\NormalTok{)]}

\CommentTok{# of the core, how many are also in the soil samples?}
\NormalTok{in.lake.core.from.soils <-}\StringTok{ }\NormalTok{in.lake.core[, }\KeywordTok{intersect}\NormalTok{(}\KeywordTok{colnames}\NormalTok{(in.lake.core), }\KeywordTok{colnames}\NormalTok{(in.soil))]}

\CommentTok{# of the core which are not in the soil samples}
\NormalTok{in.lake.core.not.soils <-}\StringTok{ }\NormalTok{in.lake.core[, }\KeywordTok{setdiff}\NormalTok{(}\KeywordTok{colnames}\NormalTok{(in.lake.core), }\KeywordTok{colnames}\NormalTok{(in.soil))]}

\CommentTok{# Find the relative abundance of the core taxa and prepare data frame to plot}
\NormalTok{in.lake.core.from.soils.REL <-}\StringTok{ }\NormalTok{in.lake.core.from.soils }\OperatorTok{/}\StringTok{ }\KeywordTok{rowSums}\NormalTok{(w.dna)}

\NormalTok{in.soil.to.plot <-}\StringTok{ }\KeywordTok{as.data.frame}\NormalTok{(in.lake.core.from.soils.REL) }\OperatorTok\StringTok{ }
\StringTok{  }\KeywordTok{rownames_to_column}\NormalTok{(}\StringTok{"sample_ID"}\NormalTok{) }\OperatorTok\StringTok{ }
\StringTok{  }\KeywordTok{gather}\NormalTok{(otu_id, rel_abundance, }\OperatorTok{-}\NormalTok{sample_ID) }\OperatorTok\StringTok{ }
\StringTok{  }\KeywordTok{left_join}\NormalTok{(}\KeywordTok{rownames_to_column}\NormalTok{(design.dna, }\StringTok{"sample_ID"}\NormalTok{)) }\OperatorTok\StringTok{ }
\StringTok{  }\KeywordTok{add_column}\NormalTok{(}\DataTypeTok{found =} \StringTok{"soils"}\NormalTok{)}

\NormalTok{in.lake.core.not.soils.REL <-}\StringTok{ }\NormalTok{in.lake.core.not.soils }\OperatorTok{/}\StringTok{ }\KeywordTok{rowSums}\NormalTok{(w.dna)}

\NormalTok{in.lake.to.plot <-}\StringTok{ }\KeywordTok{as.data.frame}\NormalTok{(in.lake.core.not.soils.REL) }\OperatorTok\StringTok{ }
\StringTok{  }\KeywordTok{rownames_to_column}\NormalTok{(}\StringTok{"sample_ID"}\NormalTok{) }\OperatorTok\StringTok{ }
\StringTok{  }\KeywordTok{gather}\NormalTok{(otu_id, rel_abundance, }\OperatorTok{-}\NormalTok{sample_ID) }\OperatorTok\StringTok{ }
\StringTok{  }\KeywordTok{left_join}\NormalTok{(}\KeywordTok{rownames_to_column}\NormalTok{(design.dna, }\StringTok{"sample_ID"}\NormalTok{)) }\OperatorTok\StringTok{ }
\StringTok{  }\KeywordTok{add_column}\NormalTok{(}\DataTypeTok{found =} \StringTok{"lake"}\NormalTok{)}
\end{Highlighting}
\end{Shaded}

Now, lets plot the abundances of the OTUs across the reservoir and split
them up into whether they were recovered in soils or not.

\begin{Shaded}
\begin{Highlighting}[]
\KeywordTok{bind_rows}\NormalTok{(in.soil.to.plot, in.lake.to.plot) }\OperatorTok\StringTok{ }
\StringTok{  }\KeywordTok{ggplot}\NormalTok{(}\KeywordTok{aes}\NormalTok{(}\DataTypeTok{x =}\NormalTok{ distance, }\DataTypeTok{y =}\NormalTok{ rel_abundance, }\DataTypeTok{group =}\NormalTok{ otu_id)) }\OperatorTok{+}\StringTok{ }
\StringTok{  }\KeywordTok{labs}\NormalTok{(}\DataTypeTok{x =} \StringTok{"Reservoir distance (m)"}\NormalTok{, }
       \DataTypeTok{y =} \StringTok{"OTU relative abundance"}\NormalTok{) }\OperatorTok{+}
\StringTok{  }\KeywordTok{geom_line}\NormalTok{(}\DataTypeTok{alpha =} \FloatTok{0.25}\NormalTok{, }\DataTypeTok{stat =} \StringTok{"smooth"}\NormalTok{, }\DataTypeTok{method =} \StringTok{"lm"}\NormalTok{, }\DataTypeTok{se =}\NormalTok{ F, }\DataTypeTok{show.legend =}\NormalTok{ F) }\OperatorTok{+}
\StringTok{  }\KeywordTok{scale_y_log10}\NormalTok{() }\OperatorTok{+}
\StringTok{  }\KeywordTok{facet_wrap}\NormalTok{(}\OperatorTok{~}\StringTok{ }\NormalTok{found, }\DataTypeTok{ncol =} \DecValTok{1}\NormalTok{, }
             \DataTypeTok{labeller =} \KeywordTok{as_labeller}\NormalTok{(}\KeywordTok{c}\NormalTok{(}
               \StringTok{`}\DataTypeTok{lake}\StringTok{`}\NormalTok{ =}\StringTok{ "Undetected in soils"}\NormalTok{,}
               \StringTok{`}\DataTypeTok{soils}\StringTok{`}\NormalTok{ =}\StringTok{ "Present in soils"}\NormalTok{)))}
\end{Highlighting}
\end{Shaded}

\begin{verbatim}
## Warning: Transformation introduced infinite values in continuous y-axis
\end{verbatim}

\begin{verbatim}
## Warning: Removed 46 rows containing non-finite values (stat_smooth).
\end{verbatim}

\begin{center}\includegraphics{ReservoirGradient_files/figure-latex/coreplot-1} \end{center}

From this figure, we note a few important points. First, we observe that
core reservoir taxa that are not detected in the soil samples tend to
increase in relative abundance along the reservoir transect. We also
note that for the taxa that are present in the soil samples, some tend
to increase drastically, while others tend to increase, along the
transect. This suggests that there may be two classes of soil-derived
OTUs that contribute to reservoir bacterial diversity:\\
- taxa where the reservoir is a sink (i.e., maintained via mass effects
from the soils) - aquatic taxa seeded by populations stored in the soils

\begin{Shaded}
\begin{Highlighting}[]
\CommentTok{# model distance effect on rel abundance to get slope and pval}
\NormalTok{soil.core.mods <-}\StringTok{ }\KeywordTok{apply}\NormalTok{(in.lake.core.from.soils.REL, }\DataTypeTok{MARGIN =} \DecValTok{2}\NormalTok{, }
    \DataTypeTok{FUN =} \ControlFlowTok{function}\NormalTok{(x) }\KeywordTok{summary}\NormalTok{(}\KeywordTok{lm}\NormalTok{(x }\OperatorTok{~}\StringTok{ }\NormalTok{design.dna}\OperatorTok{$}\NormalTok{distance))}\OperatorTok{$}\NormalTok{coefficients[}\DecValTok{2}\NormalTok{,}\KeywordTok{c}\NormalTok{(}\DecValTok{1}\NormalTok{,}\DecValTok{4}\NormalTok{)])}
\KeywordTok{rownames}\NormalTok{(soil.core.mods) <-}\StringTok{ }\KeywordTok{c}\NormalTok{(}\StringTok{"slope"}\NormalTok{, }\StringTok{"pval"}\NormalTok{)}

\CommentTok{# classify otus as significantly increasing or decreasing along reservoir}
\NormalTok{soil.core.decreasing <-}\StringTok{ }\KeywordTok{as.data.frame}\NormalTok{(}\KeywordTok{t}\NormalTok{(soil.core.mods)) }\OperatorTok\StringTok{ }
\StringTok{  }\KeywordTok{rownames_to_column}\NormalTok{(}\StringTok{"OTU"}\NormalTok{) }\OperatorTok\StringTok{ }
\StringTok{  }\KeywordTok{filter}\NormalTok{(slope }\OperatorTok{<}\StringTok{ }\DecValTok{0}\NormalTok{) }\OperatorTok\StringTok{   }\CommentTok{# rel abund decreases toward dam}
\StringTok{  }\KeywordTok{left_join}\NormalTok{(OTU.tax)}
\end{Highlighting}
\end{Shaded}

\begin{verbatim}
## Warning: Column `OTU` joining character vector and factor, coercing into
## character vector
\end{verbatim}

\begin{Shaded}
\begin{Highlighting}[]
\NormalTok{soil.core.increasing <-}\StringTok{ }\KeywordTok{as.data.frame}\NormalTok{(}\KeywordTok{t}\NormalTok{(soil.core.mods)) }\OperatorTok\StringTok{ }
\StringTok{  }\KeywordTok{rownames_to_column}\NormalTok{(}\StringTok{"OTU"}\NormalTok{) }\OperatorTok\StringTok{ }
\StringTok{  }\KeywordTok{filter}\NormalTok{(slope }\OperatorTok{>}\StringTok{ }\DecValTok{0}\NormalTok{) }\OperatorTok\StringTok{   }\CommentTok{# rel abund increases toward dam}
\StringTok{  }\KeywordTok{left_join}\NormalTok{(OTU.tax)}
\end{Highlighting}
\end{Shaded}

\begin{verbatim}
## Warning: Column `OTU` joining character vector and factor, coercing into
## character vector
\end{verbatim}

\begin{Shaded}
\begin{Highlighting}[]
\NormalTok{nonsoil.core.mods <-}\StringTok{ }\KeywordTok{apply}\NormalTok{(in.lake.core.not.soils.REL, }\DataTypeTok{MARGIN =} \DecValTok{2}\NormalTok{, }
    \DataTypeTok{FUN =} \ControlFlowTok{function}\NormalTok{(x) }\KeywordTok{summary}\NormalTok{(}\KeywordTok{lm}\NormalTok{(x }\OperatorTok{~}\StringTok{ }\NormalTok{design.dna}\OperatorTok{$}\NormalTok{distance))}\OperatorTok{$}\NormalTok{coefficients[}\DecValTok{2}\NormalTok{,}\KeywordTok{c}\NormalTok{(}\DecValTok{1}\NormalTok{,}\DecValTok{4}\NormalTok{)])}
\KeywordTok{rownames}\NormalTok{(nonsoil.core.mods) <-}\StringTok{ }\KeywordTok{c}\NormalTok{(}\StringTok{"slope"}\NormalTok{, }\StringTok{"pval"}\NormalTok{)}
\NormalTok{nonsoil.core.decreasing <-}\StringTok{ }\KeywordTok{as.data.frame}\NormalTok{(}\KeywordTok{t}\NormalTok{(nonsoil.core.mods)) }\OperatorTok\StringTok{ }
\StringTok{  }\KeywordTok{rownames_to_column}\NormalTok{(}\StringTok{"OTU"}\NormalTok{) }\OperatorTok\StringTok{ }
\StringTok{  }\KeywordTok{filter}\NormalTok{(slope }\OperatorTok{<}\StringTok{ }\DecValTok{0}\NormalTok{) }\OperatorTok\StringTok{   }\CommentTok{# rel abund decreases toward dam}
\StringTok{  }\KeywordTok{left_join}\NormalTok{(OTU.tax)}
\end{Highlighting}
\end{Shaded}

\begin{verbatim}
## Warning: Column `OTU` joining character vector and factor, coercing into
## character vector
\end{verbatim}

\begin{Shaded}
\begin{Highlighting}[]
\NormalTok{nonsoil.core.increasing <-}\StringTok{ }\KeywordTok{as.data.frame}\NormalTok{(}\KeywordTok{t}\NormalTok{(nonsoil.core.mods)) }\OperatorTok\StringTok{ }
\StringTok{  }\KeywordTok{rownames_to_column}\NormalTok{(}\StringTok{"OTU"}\NormalTok{) }\OperatorTok\StringTok{ }
\StringTok{  }\KeywordTok{filter}\NormalTok{(slope }\OperatorTok{>}\StringTok{ }\DecValTok{0}\NormalTok{) }\OperatorTok\StringTok{   }\CommentTok{# rel abund increases toward dam}
\StringTok{  }\KeywordTok{left_join}\NormalTok{(OTU.tax)}
\end{Highlighting}
\end{Shaded}

\begin{verbatim}
## Warning: Column `OTU` joining character vector and factor, coercing into
## character vector
\end{verbatim}

Now we will visualize the significant taxa

\begin{Shaded}
\begin{Highlighting}[]
\KeywordTok{pander}\NormalTok{(nonsoil.core.decreasing, }\DataTypeTok{caption =} \StringTok{"Core taxa not found in soils that get rarer along the transect."}\NormalTok{)}
\end{Highlighting}
\end{Shaded}

\begin{longtable}[]{@{}ccccc@{}}
\caption{Core taxa not found in soils that get rarer along the transect.
(continued below)}\tabularnewline
\toprule
\begin{minipage}[b]{0.13\columnwidth}\centering
OTU\strut
\end{minipage} & \begin{minipage}[b]{0.16\columnwidth}\centering
slope\strut
\end{minipage} & \begin{minipage}[b]{0.13\columnwidth}\centering
pval\strut
\end{minipage} & \begin{minipage}[b]{0.13\columnwidth}\centering
Domain\strut
\end{minipage} & \begin{minipage}[b]{0.29\columnwidth}\centering
Phylum\strut
\end{minipage}\tabularnewline
\midrule
\endfirsthead
\toprule
\begin{minipage}[b]{0.13\columnwidth}\centering
OTU\strut
\end{minipage} & \begin{minipage}[b]{0.16\columnwidth}\centering
slope\strut
\end{minipage} & \begin{minipage}[b]{0.13\columnwidth}\centering
pval\strut
\end{minipage} & \begin{minipage}[b]{0.13\columnwidth}\centering
Domain\strut
\end{minipage} & \begin{minipage}[b]{0.29\columnwidth}\centering
Phylum\strut
\end{minipage}\tabularnewline
\midrule
\endhead
\begin{minipage}[t]{0.13\columnwidth}\centering
Otu00007\strut
\end{minipage} & \begin{minipage}[t]{0.16\columnwidth}\centering
-8.015e-06\strut
\end{minipage} & \begin{minipage}[t]{0.13\columnwidth}\centering
0.2431\strut
\end{minipage} & \begin{minipage}[t]{0.13\columnwidth}\centering
Bacteria\strut
\end{minipage} & \begin{minipage}[t]{0.29\columnwidth}\centering
Proteobacteria\strut
\end{minipage}\tabularnewline
\begin{minipage}[t]{0.13\columnwidth}\centering
Otu00020\strut
\end{minipage} & \begin{minipage}[t]{0.16\columnwidth}\centering
-1.704e-05\strut
\end{minipage} & \begin{minipage}[t]{0.13\columnwidth}\centering
0.4607\strut
\end{minipage} & \begin{minipage}[t]{0.13\columnwidth}\centering
Bacteria\strut
\end{minipage} & \begin{minipage}[t]{0.29\columnwidth}\centering
Proteobacteria\strut
\end{minipage}\tabularnewline
\begin{minipage}[t]{0.13\columnwidth}\centering
Otu00024\strut
\end{minipage} & \begin{minipage}[t]{0.16\columnwidth}\centering
-2.897e-06\strut
\end{minipage} & \begin{minipage}[t]{0.13\columnwidth}\centering
0.3675\strut
\end{minipage} & \begin{minipage}[t]{0.13\columnwidth}\centering
Bacteria\strut
\end{minipage} & \begin{minipage}[t]{0.29\columnwidth}\centering
Bacteroidetes\strut
\end{minipage}\tabularnewline
\begin{minipage}[t]{0.13\columnwidth}\centering
Otu00057\strut
\end{minipage} & \begin{minipage}[t]{0.16\columnwidth}\centering
-3.017e-05\strut
\end{minipage} & \begin{minipage}[t]{0.13\columnwidth}\centering
0.009476\strut
\end{minipage} & \begin{minipage}[t]{0.13\columnwidth}\centering
Bacteria\strut
\end{minipage} & \begin{minipage}[t]{0.29\columnwidth}\centering
Proteobacteria\strut
\end{minipage}\tabularnewline
\begin{minipage}[t]{0.13\columnwidth}\centering
Otu00138\strut
\end{minipage} & \begin{minipage}[t]{0.16\columnwidth}\centering
-3.401e-05\strut
\end{minipage} & \begin{minipage}[t]{0.13\columnwidth}\centering
0.016\strut
\end{minipage} & \begin{minipage}[t]{0.13\columnwidth}\centering
Bacteria\strut
\end{minipage} & \begin{minipage}[t]{0.29\columnwidth}\centering
Firmicutes\strut
\end{minipage}\tabularnewline
\begin{minipage}[t]{0.13\columnwidth}\centering
Otu00169\strut
\end{minipage} & \begin{minipage}[t]{0.16\columnwidth}\centering
-1.048e-05\strut
\end{minipage} & \begin{minipage}[t]{0.13\columnwidth}\centering
0.3397\strut
\end{minipage} & \begin{minipage}[t]{0.13\columnwidth}\centering
Bacteria\strut
\end{minipage} & \begin{minipage}[t]{0.29\columnwidth}\centering
Bacteria\_unclassified\strut
\end{minipage}\tabularnewline
\begin{minipage}[t]{0.13\columnwidth}\centering
Otu01010\strut
\end{minipage} & \begin{minipage}[t]{0.16\columnwidth}\centering
-3.563e-08\strut
\end{minipage} & \begin{minipage}[t]{0.13\columnwidth}\centering
0.635\strut
\end{minipage} & \begin{minipage}[t]{0.13\columnwidth}\centering
Bacteria\strut
\end{minipage} & \begin{minipage}[t]{0.29\columnwidth}\centering
Actinobacteria\strut
\end{minipage}\tabularnewline
\bottomrule
\end{longtable}

\begin{longtable}[]{@{}cc@{}}
\caption{Table continues below}\tabularnewline
\toprule
\begin{minipage}[b]{0.38\columnwidth}\centering
Class\strut
\end{minipage} & \begin{minipage}[b]{0.38\columnwidth}\centering
Order\strut
\end{minipage}\tabularnewline
\midrule
\endfirsthead
\toprule
\begin{minipage}[b]{0.38\columnwidth}\centering
Class\strut
\end{minipage} & \begin{minipage}[b]{0.38\columnwidth}\centering
Order\strut
\end{minipage}\tabularnewline
\midrule
\endhead
\begin{minipage}[t]{0.38\columnwidth}\centering
Betaproteobacteria\strut
\end{minipage} & \begin{minipage}[t]{0.38\columnwidth}\centering
Burkholderiales\strut
\end{minipage}\tabularnewline
\begin{minipage}[t]{0.38\columnwidth}\centering
Betaproteobacteria\strut
\end{minipage} & \begin{minipage}[t]{0.38\columnwidth}\centering
Burkholderiales\strut
\end{minipage}\tabularnewline
\begin{minipage}[t]{0.38\columnwidth}\centering
Bacteroidetes\_unclassified\strut
\end{minipage} & \begin{minipage}[t]{0.38\columnwidth}\centering
Bacteroidetes\_unclassified\strut
\end{minipage}\tabularnewline
\begin{minipage}[t]{0.38\columnwidth}\centering
Gammaproteobacteria\strut
\end{minipage} & \begin{minipage}[t]{0.38\columnwidth}\centering
Methylococcales\strut
\end{minipage}\tabularnewline
\begin{minipage}[t]{0.38\columnwidth}\centering
Bacilli\strut
\end{minipage} & \begin{minipage}[t]{0.38\columnwidth}\centering
Bacillales\strut
\end{minipage}\tabularnewline
\begin{minipage}[t]{0.38\columnwidth}\centering
Bacteria\_unclassified\strut
\end{minipage} & \begin{minipage}[t]{0.38\columnwidth}\centering
Bacteria\_unclassified\strut
\end{minipage}\tabularnewline
\begin{minipage}[t]{0.38\columnwidth}\centering
Actinobacteria\strut
\end{minipage} & \begin{minipage}[t]{0.38\columnwidth}\centering
Actinomycetales\strut
\end{minipage}\tabularnewline
\bottomrule
\end{longtable}

\begin{longtable}[]{@{}cc@{}}
\toprule
\begin{minipage}[b]{0.38\columnwidth}\centering
Family\strut
\end{minipage} & \begin{minipage}[b]{0.42\columnwidth}\centering
Genus\strut
\end{minipage}\tabularnewline
\midrule
\endhead
\begin{minipage}[t]{0.38\columnwidth}\centering
Burkholderiaceae\strut
\end{minipage} & \begin{minipage}[t]{0.42\columnwidth}\centering
Polynucleobacter\strut
\end{minipage}\tabularnewline
\begin{minipage}[t]{0.38\columnwidth}\centering
Alcaligenaceae\strut
\end{minipage} & \begin{minipage}[t]{0.42\columnwidth}\centering
Alcaligenaceae\_unclassified\strut
\end{minipage}\tabularnewline
\begin{minipage}[t]{0.38\columnwidth}\centering
Bacteroidetes\_unclassified\strut
\end{minipage} & \begin{minipage}[t]{0.42\columnwidth}\centering
Bacteroidetes\_unclassified\strut
\end{minipage}\tabularnewline
\begin{minipage}[t]{0.38\columnwidth}\centering
Methylococcaceae\strut
\end{minipage} & \begin{minipage}[t]{0.42\columnwidth}\centering
Methylococcaceae\_unclassified\strut
\end{minipage}\tabularnewline
\begin{minipage}[t]{0.38\columnwidth}\centering
Bacillaceae\_1\strut
\end{minipage} & \begin{minipage}[t]{0.42\columnwidth}\centering
Bacillus\strut
\end{minipage}\tabularnewline
\begin{minipage}[t]{0.38\columnwidth}\centering
Bacteria\_unclassified\strut
\end{minipage} & \begin{minipage}[t]{0.42\columnwidth}\centering
Bacteria\_unclassified\strut
\end{minipage}\tabularnewline
\begin{minipage}[t]{0.38\columnwidth}\centering
Dermabacteraceae\strut
\end{minipage} & \begin{minipage}[t]{0.42\columnwidth}\centering
Brachybacterium\strut
\end{minipage}\tabularnewline
\bottomrule
\end{longtable}

\begin{Shaded}
\begin{Highlighting}[]
\KeywordTok{pander}\NormalTok{(nonsoil.core.increasing, }\DataTypeTok{caption =} \StringTok{"Core taxa not found in soils that get more common along the transect."}\NormalTok{)}
\end{Highlighting}
\end{Shaded}

\begin{longtable}[]{@{}ccccc@{}}
\caption{Core taxa not found in soils that get more common along the
transect. (continued below)}\tabularnewline
\toprule
\begin{minipage}[b]{0.13\columnwidth}\centering
OTU\strut
\end{minipage} & \begin{minipage}[b]{0.14\columnwidth}\centering
slope\strut
\end{minipage} & \begin{minipage}[b]{0.14\columnwidth}\centering
pval\strut
\end{minipage} & \begin{minipage}[b]{0.13\columnwidth}\centering
Domain\strut
\end{minipage} & \begin{minipage}[b]{0.29\columnwidth}\centering
Phylum\strut
\end{minipage}\tabularnewline
\midrule
\endfirsthead
\toprule
\begin{minipage}[b]{0.13\columnwidth}\centering
OTU\strut
\end{minipage} & \begin{minipage}[b]{0.14\columnwidth}\centering
slope\strut
\end{minipage} & \begin{minipage}[b]{0.14\columnwidth}\centering
pval\strut
\end{minipage} & \begin{minipage}[b]{0.13\columnwidth}\centering
Domain\strut
\end{minipage} & \begin{minipage}[b]{0.29\columnwidth}\centering
Phylum\strut
\end{minipage}\tabularnewline
\midrule
\endhead
\begin{minipage}[t]{0.13\columnwidth}\centering
Otu00004\strut
\end{minipage} & \begin{minipage}[t]{0.14\columnwidth}\centering
0.0001345\strut
\end{minipage} & \begin{minipage}[t]{0.14\columnwidth}\centering
1.671e-05\strut
\end{minipage} & \begin{minipage}[t]{0.13\columnwidth}\centering
Bacteria\strut
\end{minipage} & \begin{minipage}[t]{0.29\columnwidth}\centering
Actinobacteria\strut
\end{minipage}\tabularnewline
\begin{minipage}[t]{0.13\columnwidth}\centering
Otu00008\strut
\end{minipage} & \begin{minipage}[t]{0.14\columnwidth}\centering
3.306e-05\strut
\end{minipage} & \begin{minipage}[t]{0.14\columnwidth}\centering
0.02659\strut
\end{minipage} & \begin{minipage}[t]{0.13\columnwidth}\centering
Bacteria\strut
\end{minipage} & \begin{minipage}[t]{0.29\columnwidth}\centering
Actinobacteria\strut
\end{minipage}\tabularnewline
\begin{minipage}[t]{0.13\columnwidth}\centering
Otu00015\strut
\end{minipage} & \begin{minipage}[t]{0.14\columnwidth}\centering
0.0001372\strut
\end{minipage} & \begin{minipage}[t]{0.14\columnwidth}\centering
0.0003621\strut
\end{minipage} & \begin{minipage}[t]{0.13\columnwidth}\centering
Bacteria\strut
\end{minipage} & \begin{minipage}[t]{0.29\columnwidth}\centering
Actinobacteria\strut
\end{minipage}\tabularnewline
\begin{minipage}[t]{0.13\columnwidth}\centering
Otu00016\strut
\end{minipage} & \begin{minipage}[t]{0.14\columnwidth}\centering
5.151e-05\strut
\end{minipage} & \begin{minipage}[t]{0.14\columnwidth}\centering
0.002113\strut
\end{minipage} & \begin{minipage}[t]{0.13\columnwidth}\centering
Bacteria\strut
\end{minipage} & \begin{minipage}[t]{0.29\columnwidth}\centering
Actinobacteria\strut
\end{minipage}\tabularnewline
\begin{minipage}[t]{0.13\columnwidth}\centering
Otu00025\strut
\end{minipage} & \begin{minipage}[t]{0.14\columnwidth}\centering
4.63e-05\strut
\end{minipage} & \begin{minipage}[t]{0.14\columnwidth}\centering
0.006728\strut
\end{minipage} & \begin{minipage}[t]{0.13\columnwidth}\centering
Bacteria\strut
\end{minipage} & \begin{minipage}[t]{0.29\columnwidth}\centering
Actinobacteria\strut
\end{minipage}\tabularnewline
\begin{minipage}[t]{0.13\columnwidth}\centering
Otu00038\strut
\end{minipage} & \begin{minipage}[t]{0.14\columnwidth}\centering
4.561e-05\strut
\end{minipage} & \begin{minipage}[t]{0.14\columnwidth}\centering
0.0001738\strut
\end{minipage} & \begin{minipage}[t]{0.13\columnwidth}\centering
Bacteria\strut
\end{minipage} & \begin{minipage}[t]{0.29\columnwidth}\centering
Actinobacteria\strut
\end{minipage}\tabularnewline
\begin{minipage}[t]{0.13\columnwidth}\centering
Otu00040\strut
\end{minipage} & \begin{minipage}[t]{0.14\columnwidth}\centering
3.744e-05\strut
\end{minipage} & \begin{minipage}[t]{0.14\columnwidth}\centering
2.589e-05\strut
\end{minipage} & \begin{minipage}[t]{0.13\columnwidth}\centering
Bacteria\strut
\end{minipage} & \begin{minipage}[t]{0.29\columnwidth}\centering
Proteobacteria\strut
\end{minipage}\tabularnewline
\begin{minipage}[t]{0.13\columnwidth}\centering
Otu00071\strut
\end{minipage} & \begin{minipage}[t]{0.14\columnwidth}\centering
4.8e-05\strut
\end{minipage} & \begin{minipage}[t]{0.14\columnwidth}\centering
0.0004517\strut
\end{minipage} & \begin{minipage}[t]{0.13\columnwidth}\centering
Bacteria\strut
\end{minipage} & \begin{minipage}[t]{0.29\columnwidth}\centering
Planctomycetes\strut
\end{minipage}\tabularnewline
\begin{minipage}[t]{0.13\columnwidth}\centering
Otu00079\strut
\end{minipage} & \begin{minipage}[t]{0.14\columnwidth}\centering
8.122e-06\strut
\end{minipage} & \begin{minipage}[t]{0.14\columnwidth}\centering
0.001732\strut
\end{minipage} & \begin{minipage}[t]{0.13\columnwidth}\centering
Bacteria\strut
\end{minipage} & \begin{minipage}[t]{0.29\columnwidth}\centering
Bacteroidetes\strut
\end{minipage}\tabularnewline
\begin{minipage}[t]{0.13\columnwidth}\centering
Otu00080\strut
\end{minipage} & \begin{minipage}[t]{0.14\columnwidth}\centering
1.601e-05\strut
\end{minipage} & \begin{minipage}[t]{0.14\columnwidth}\centering
0.1586\strut
\end{minipage} & \begin{minipage}[t]{0.13\columnwidth}\centering
Bacteria\strut
\end{minipage} & \begin{minipage}[t]{0.29\columnwidth}\centering
Bacteroidetes\strut
\end{minipage}\tabularnewline
\begin{minipage}[t]{0.13\columnwidth}\centering
Otu00118\strut
\end{minipage} & \begin{minipage}[t]{0.14\columnwidth}\centering
6.59e-06\strut
\end{minipage} & \begin{minipage}[t]{0.14\columnwidth}\centering
0.03765\strut
\end{minipage} & \begin{minipage}[t]{0.13\columnwidth}\centering
Bacteria\strut
\end{minipage} & \begin{minipage}[t]{0.29\columnwidth}\centering
Actinobacteria\strut
\end{minipage}\tabularnewline
\begin{minipage}[t]{0.13\columnwidth}\centering
Otu00156\strut
\end{minipage} & \begin{minipage}[t]{0.14\columnwidth}\centering
8.854e-06\strut
\end{minipage} & \begin{minipage}[t]{0.14\columnwidth}\centering
0.002739\strut
\end{minipage} & \begin{minipage}[t]{0.13\columnwidth}\centering
Bacteria\strut
\end{minipage} & \begin{minipage}[t]{0.29\columnwidth}\centering
Bacteria\_unclassified\strut
\end{minipage}\tabularnewline
\bottomrule
\end{longtable}

\begin{longtable}[]{@{}cc@{}}
\caption{Table continues below}\tabularnewline
\toprule
\begin{minipage}[b]{0.38\columnwidth}\centering
Class\strut
\end{minipage} & \begin{minipage}[b]{0.39\columnwidth}\centering
Order\strut
\end{minipage}\tabularnewline
\midrule
\endfirsthead
\toprule
\begin{minipage}[b]{0.38\columnwidth}\centering
Class\strut
\end{minipage} & \begin{minipage}[b]{0.39\columnwidth}\centering
Order\strut
\end{minipage}\tabularnewline
\midrule
\endhead
\begin{minipage}[t]{0.38\columnwidth}\centering
Actinobacteria\strut
\end{minipage} & \begin{minipage}[t]{0.39\columnwidth}\centering
Actinomycetales\strut
\end{minipage}\tabularnewline
\begin{minipage}[t]{0.38\columnwidth}\centering
Actinobacteria\strut
\end{minipage} & \begin{minipage}[t]{0.39\columnwidth}\centering
Actinomycetales\strut
\end{minipage}\tabularnewline
\begin{minipage}[t]{0.38\columnwidth}\centering
Actinobacteria\strut
\end{minipage} & \begin{minipage}[t]{0.39\columnwidth}\centering
Actinobacteria\_unclassified\strut
\end{minipage}\tabularnewline
\begin{minipage}[t]{0.38\columnwidth}\centering
Actinobacteria\strut
\end{minipage} & \begin{minipage}[t]{0.39\columnwidth}\centering
Actinomycetales\strut
\end{minipage}\tabularnewline
\begin{minipage}[t]{0.38\columnwidth}\centering
Actinobacteria\strut
\end{minipage} & \begin{minipage}[t]{0.39\columnwidth}\centering
Actinomycetales\strut
\end{minipage}\tabularnewline
\begin{minipage}[t]{0.38\columnwidth}\centering
Actinobacteria\strut
\end{minipage} & \begin{minipage}[t]{0.39\columnwidth}\centering
Actinomycetales\strut
\end{minipage}\tabularnewline
\begin{minipage}[t]{0.38\columnwidth}\centering
Alphaproteobacteria\strut
\end{minipage} & \begin{minipage}[t]{0.39\columnwidth}\centering
Rhodospirillales\strut
\end{minipage}\tabularnewline
\begin{minipage}[t]{0.38\columnwidth}\centering
Planctomycetia\strut
\end{minipage} & \begin{minipage}[t]{0.39\columnwidth}\centering
Planctomycetales\strut
\end{minipage}\tabularnewline
\begin{minipage}[t]{0.38\columnwidth}\centering
Bacteroidetes\_unclassified\strut
\end{minipage} & \begin{minipage}[t]{0.39\columnwidth}\centering
Bacteroidetes\_unclassified\strut
\end{minipage}\tabularnewline
\begin{minipage}[t]{0.38\columnwidth}\centering
Flavobacteriia\strut
\end{minipage} & \begin{minipage}[t]{0.39\columnwidth}\centering
Flavobacteriales\strut
\end{minipage}\tabularnewline
\begin{minipage}[t]{0.38\columnwidth}\centering
Actinobacteria\strut
\end{minipage} & \begin{minipage}[t]{0.39\columnwidth}\centering
Actinobacteria\_unclassified\strut
\end{minipage}\tabularnewline
\begin{minipage}[t]{0.38\columnwidth}\centering
Bacteria\_unclassified\strut
\end{minipage} & \begin{minipage}[t]{0.39\columnwidth}\centering
Bacteria\_unclassified\strut
\end{minipage}\tabularnewline
\bottomrule
\end{longtable}

\begin{longtable}[]{@{}cc@{}}
\toprule
\begin{minipage}[b]{0.41\columnwidth}\centering
Family\strut
\end{minipage} & \begin{minipage}[b]{0.43\columnwidth}\centering
Genus\strut
\end{minipage}\tabularnewline
\midrule
\endhead
\begin{minipage}[t]{0.41\columnwidth}\centering
Actinomycetales\_unclassified\strut
\end{minipage} & \begin{minipage}[t]{0.43\columnwidth}\centering
Actinomycetales\_unclassified\strut
\end{minipage}\tabularnewline
\begin{minipage}[t]{0.41\columnwidth}\centering
Actinomycetales\_unclassified\strut
\end{minipage} & \begin{minipage}[t]{0.43\columnwidth}\centering
Actinomycetales\_unclassified\strut
\end{minipage}\tabularnewline
\begin{minipage}[t]{0.41\columnwidth}\centering
Actinobacteria\_unclassified\strut
\end{minipage} & \begin{minipage}[t]{0.43\columnwidth}\centering
Actinobacteria\_unclassified\strut
\end{minipage}\tabularnewline
\begin{minipage}[t]{0.41\columnwidth}\centering
Microbacteriaceae\strut
\end{minipage} & \begin{minipage}[t]{0.43\columnwidth}\centering
Microbacteriaceae\_unclassified\strut
\end{minipage}\tabularnewline
\begin{minipage}[t]{0.41\columnwidth}\centering
Microbacteriaceae\strut
\end{minipage} & \begin{minipage}[t]{0.43\columnwidth}\centering
Microbacteriaceae\_unclassified\strut
\end{minipage}\tabularnewline
\begin{minipage}[t]{0.41\columnwidth}\centering
Actinomycetales\_unclassified\strut
\end{minipage} & \begin{minipage}[t]{0.43\columnwidth}\centering
Actinomycetales\_unclassified\strut
\end{minipage}\tabularnewline
\begin{minipage}[t]{0.41\columnwidth}\centering
Acetobacteraceae\strut
\end{minipage} & \begin{minipage}[t]{0.43\columnwidth}\centering
Roseomonas\strut
\end{minipage}\tabularnewline
\begin{minipage}[t]{0.41\columnwidth}\centering
Planctomycetaceae\strut
\end{minipage} & \begin{minipage}[t]{0.43\columnwidth}\centering
Planctomycetaceae\_unclassified\strut
\end{minipage}\tabularnewline
\begin{minipage}[t]{0.41\columnwidth}\centering
Bacteroidetes\_unclassified\strut
\end{minipage} & \begin{minipage}[t]{0.43\columnwidth}\centering
Bacteroidetes\_unclassified\strut
\end{minipage}\tabularnewline
\begin{minipage}[t]{0.41\columnwidth}\centering
Flavobacteriaceae\strut
\end{minipage} & \begin{minipage}[t]{0.43\columnwidth}\centering
Flavobacterium\strut
\end{minipage}\tabularnewline
\begin{minipage}[t]{0.41\columnwidth}\centering
Actinobacteria\_unclassified\strut
\end{minipage} & \begin{minipage}[t]{0.43\columnwidth}\centering
Actinobacteria\_unclassified\strut
\end{minipage}\tabularnewline
\begin{minipage}[t]{0.41\columnwidth}\centering
Bacteria\_unclassified\strut
\end{minipage} & \begin{minipage}[t]{0.43\columnwidth}\centering
Bacteria\_unclassified\strut
\end{minipage}\tabularnewline
\bottomrule
\end{longtable}

\begin{Shaded}
\begin{Highlighting}[]
\KeywordTok{pander}\NormalTok{(soil.core.decreasing, }\DataTypeTok{caption =} \StringTok{"Core taxa found in soils that get rarer along the transect."}\NormalTok{)}
\end{Highlighting}
\end{Shaded}

\begin{longtable}[]{@{}ccccc@{}}
\caption{Core taxa found in soils that get rarer along the transect.
(continued below)}\tabularnewline
\toprule
\begin{minipage}[b]{0.13\columnwidth}\centering
OTU\strut
\end{minipage} & \begin{minipage}[b]{0.16\columnwidth}\centering
slope\strut
\end{minipage} & \begin{minipage}[b]{0.12\columnwidth}\centering
pval\strut
\end{minipage} & \begin{minipage}[b]{0.13\columnwidth}\centering
Domain\strut
\end{minipage} & \begin{minipage}[b]{0.21\columnwidth}\centering
Phylum\strut
\end{minipage}\tabularnewline
\midrule
\endfirsthead
\toprule
\begin{minipage}[b]{0.13\columnwidth}\centering
OTU\strut
\end{minipage} & \begin{minipage}[b]{0.16\columnwidth}\centering
slope\strut
\end{minipage} & \begin{minipage}[b]{0.12\columnwidth}\centering
pval\strut
\end{minipage} & \begin{minipage}[b]{0.13\columnwidth}\centering
Domain\strut
\end{minipage} & \begin{minipage}[b]{0.21\columnwidth}\centering
Phylum\strut
\end{minipage}\tabularnewline
\midrule
\endhead
\begin{minipage}[t]{0.13\columnwidth}\centering
Otu00009\strut
\end{minipage} & \begin{minipage}[t]{0.16\columnwidth}\centering
-5.159e-05\strut
\end{minipage} & \begin{minipage}[t]{0.12\columnwidth}\centering
0.02755\strut
\end{minipage} & \begin{minipage}[t]{0.13\columnwidth}\centering
Bacteria\strut
\end{minipage} & \begin{minipage}[t]{0.21\columnwidth}\centering
Proteobacteria\strut
\end{minipage}\tabularnewline
\begin{minipage}[t]{0.13\columnwidth}\centering
Otu00010\strut
\end{minipage} & \begin{minipage}[t]{0.16\columnwidth}\centering
-4.34e-05\strut
\end{minipage} & \begin{minipage}[t]{0.12\columnwidth}\centering
0.5521\strut
\end{minipage} & \begin{minipage}[t]{0.13\columnwidth}\centering
Bacteria\strut
\end{minipage} & \begin{minipage}[t]{0.21\columnwidth}\centering
Proteobacteria\strut
\end{minipage}\tabularnewline
\begin{minipage}[t]{0.13\columnwidth}\centering
Otu00011\strut
\end{minipage} & \begin{minipage}[t]{0.16\columnwidth}\centering
-1.949e-05\strut
\end{minipage} & \begin{minipage}[t]{0.12\columnwidth}\centering
0.6012\strut
\end{minipage} & \begin{minipage}[t]{0.13\columnwidth}\centering
Bacteria\strut
\end{minipage} & \begin{minipage}[t]{0.21\columnwidth}\centering
Proteobacteria\strut
\end{minipage}\tabularnewline
\begin{minipage}[t]{0.13\columnwidth}\centering
Otu00018\strut
\end{minipage} & \begin{minipage}[t]{0.16\columnwidth}\centering
-4.676e-05\strut
\end{minipage} & \begin{minipage}[t]{0.12\columnwidth}\centering
0.02114\strut
\end{minipage} & \begin{minipage}[t]{0.13\columnwidth}\centering
Bacteria\strut
\end{minipage} & \begin{minipage}[t]{0.21\columnwidth}\centering
Proteobacteria\strut
\end{minipage}\tabularnewline
\begin{minipage}[t]{0.13\columnwidth}\centering
Otu00022\strut
\end{minipage} & \begin{minipage}[t]{0.16\columnwidth}\centering
-2.524e-05\strut
\end{minipage} & \begin{minipage}[t]{0.12\columnwidth}\centering
0.1182\strut
\end{minipage} & \begin{minipage}[t]{0.13\columnwidth}\centering
Bacteria\strut
\end{minipage} & \begin{minipage}[t]{0.21\columnwidth}\centering
Verrucomicrobia\strut
\end{minipage}\tabularnewline
\begin{minipage}[t]{0.13\columnwidth}\centering
Otu00028\strut
\end{minipage} & \begin{minipage}[t]{0.16\columnwidth}\centering
-3.068e-05\strut
\end{minipage} & \begin{minipage}[t]{0.12\columnwidth}\centering
0.02359\strut
\end{minipage} & \begin{minipage}[t]{0.13\columnwidth}\centering
Bacteria\strut
\end{minipage} & \begin{minipage}[t]{0.21\columnwidth}\centering
Proteobacteria\strut
\end{minipage}\tabularnewline
\begin{minipage}[t]{0.13\columnwidth}\centering
Otu00030\strut
\end{minipage} & \begin{minipage}[t]{0.16\columnwidth}\centering
-2.244e-06\strut
\end{minipage} & \begin{minipage}[t]{0.12\columnwidth}\centering
0.2763\strut
\end{minipage} & \begin{minipage}[t]{0.13\columnwidth}\centering
Bacteria\strut
\end{minipage} & \begin{minipage}[t]{0.21\columnwidth}\centering
Actinobacteria\strut
\end{minipage}\tabularnewline
\begin{minipage}[t]{0.13\columnwidth}\centering
Otu00039\strut
\end{minipage} & \begin{minipage}[t]{0.16\columnwidth}\centering
-8.596e-06\strut
\end{minipage} & \begin{minipage}[t]{0.12\columnwidth}\centering
0.1787\strut
\end{minipage} & \begin{minipage}[t]{0.13\columnwidth}\centering
Bacteria\strut
\end{minipage} & \begin{minipage}[t]{0.21\columnwidth}\centering
Proteobacteria\strut
\end{minipage}\tabularnewline
\begin{minipage}[t]{0.13\columnwidth}\centering
Otu00045\strut
\end{minipage} & \begin{minipage}[t]{0.16\columnwidth}\centering
-8.037e-06\strut
\end{minipage} & \begin{minipage}[t]{0.12\columnwidth}\centering
0.5276\strut
\end{minipage} & \begin{minipage}[t]{0.13\columnwidth}\centering
Bacteria\strut
\end{minipage} & \begin{minipage}[t]{0.21\columnwidth}\centering
Proteobacteria\strut
\end{minipage}\tabularnewline
\begin{minipage}[t]{0.13\columnwidth}\centering
Otu00059\strut
\end{minipage} & \begin{minipage}[t]{0.16\columnwidth}\centering
-6.541e-05\strut
\end{minipage} & \begin{minipage}[t]{0.12\columnwidth}\centering
0.02553\strut
\end{minipage} & \begin{minipage}[t]{0.13\columnwidth}\centering
Bacteria\strut
\end{minipage} & \begin{minipage}[t]{0.21\columnwidth}\centering
Actinobacteria\strut
\end{minipage}\tabularnewline
\begin{minipage}[t]{0.13\columnwidth}\centering
Otu00065\strut
\end{minipage} & \begin{minipage}[t]{0.16\columnwidth}\centering
-5.579e-05\strut
\end{minipage} & \begin{minipage}[t]{0.12\columnwidth}\centering
0.02116\strut
\end{minipage} & \begin{minipage}[t]{0.13\columnwidth}\centering
Bacteria\strut
\end{minipage} & \begin{minipage}[t]{0.21\columnwidth}\centering
Bacteroidetes\strut
\end{minipage}\tabularnewline
\begin{minipage}[t]{0.13\columnwidth}\centering
Otu00072\strut
\end{minipage} & \begin{minipage}[t]{0.16\columnwidth}\centering
-1.895e-05\strut
\end{minipage} & \begin{minipage}[t]{0.12\columnwidth}\centering
0.09149\strut
\end{minipage} & \begin{minipage}[t]{0.13\columnwidth}\centering
Bacteria\strut
\end{minipage} & \begin{minipage}[t]{0.21\columnwidth}\centering
Proteobacteria\strut
\end{minipage}\tabularnewline
\begin{minipage}[t]{0.13\columnwidth}\centering
Otu00077\strut
\end{minipage} & \begin{minipage}[t]{0.16\columnwidth}\centering
-5.886e-05\strut
\end{minipage} & \begin{minipage}[t]{0.12\columnwidth}\centering
0.01187\strut
\end{minipage} & \begin{minipage}[t]{0.13\columnwidth}\centering
Bacteria\strut
\end{minipage} & \begin{minipage}[t]{0.21\columnwidth}\centering
Bacteroidetes\strut
\end{minipage}\tabularnewline
\begin{minipage}[t]{0.13\columnwidth}\centering
Otu00086\strut
\end{minipage} & \begin{minipage}[t]{0.16\columnwidth}\centering
-1.265e-05\strut
\end{minipage} & \begin{minipage}[t]{0.12\columnwidth}\centering
0.03621\strut
\end{minipage} & \begin{minipage}[t]{0.13\columnwidth}\centering
Bacteria\strut
\end{minipage} & \begin{minipage}[t]{0.21\columnwidth}\centering
Proteobacteria\strut
\end{minipage}\tabularnewline
\begin{minipage}[t]{0.13\columnwidth}\centering
Otu00094\strut
\end{minipage} & \begin{minipage}[t]{0.16\columnwidth}\centering
-2.23e-05\strut
\end{minipage} & \begin{minipage}[t]{0.12\columnwidth}\centering
0.03169\strut
\end{minipage} & \begin{minipage}[t]{0.13\columnwidth}\centering
Bacteria\strut
\end{minipage} & \begin{minipage}[t]{0.21\columnwidth}\centering
Proteobacteria\strut
\end{minipage}\tabularnewline
\begin{minipage}[t]{0.13\columnwidth}\centering
Otu00095\strut
\end{minipage} & \begin{minipage}[t]{0.16\columnwidth}\centering
-3.578e-05\strut
\end{minipage} & \begin{minipage}[t]{0.12\columnwidth}\centering
0.03614\strut
\end{minipage} & \begin{minipage}[t]{0.13\columnwidth}\centering
Bacteria\strut
\end{minipage} & \begin{minipage}[t]{0.21\columnwidth}\centering
Proteobacteria\strut
\end{minipage}\tabularnewline
\begin{minipage}[t]{0.13\columnwidth}\centering
Otu00170\strut
\end{minipage} & \begin{minipage}[t]{0.16\columnwidth}\centering
-2.494e-05\strut
\end{minipage} & \begin{minipage}[t]{0.12\columnwidth}\centering
0.02878\strut
\end{minipage} & \begin{minipage}[t]{0.13\columnwidth}\centering
Bacteria\strut
\end{minipage} & \begin{minipage}[t]{0.21\columnwidth}\centering
Bacteroidetes\strut
\end{minipage}\tabularnewline
\begin{minipage}[t]{0.13\columnwidth}\centering
Otu00545\strut
\end{minipage} & \begin{minipage}[t]{0.16\columnwidth}\centering
-1.236e-06\strut
\end{minipage} & \begin{minipage}[t]{0.12\columnwidth}\centering
0.02985\strut
\end{minipage} & \begin{minipage}[t]{0.13\columnwidth}\centering
Bacteria\strut
\end{minipage} & \begin{minipage}[t]{0.21\columnwidth}\centering
Actinobacteria\strut
\end{minipage}\tabularnewline
\bottomrule
\end{longtable}

\begin{longtable}[]{@{}cc@{}}
\caption{Table continues below}\tabularnewline
\toprule
\begin{minipage}[b]{0.39\columnwidth}\centering
Class\strut
\end{minipage} & \begin{minipage}[b]{0.44\columnwidth}\centering
Order\strut
\end{minipage}\tabularnewline
\midrule
\endfirsthead
\toprule
\begin{minipage}[b]{0.39\columnwidth}\centering
Class\strut
\end{minipage} & \begin{minipage}[b]{0.44\columnwidth}\centering
Order\strut
\end{minipage}\tabularnewline
\midrule
\endhead
\begin{minipage}[t]{0.39\columnwidth}\centering
Gammaproteobacteria\strut
\end{minipage} & \begin{minipage}[t]{0.44\columnwidth}\centering
Pseudomonadales\strut
\end{minipage}\tabularnewline
\begin{minipage}[t]{0.39\columnwidth}\centering
Proteobacteria\_unclassified\strut
\end{minipage} & \begin{minipage}[t]{0.44\columnwidth}\centering
Proteobacteria\_unclassified\strut
\end{minipage}\tabularnewline
\begin{minipage}[t]{0.39\columnwidth}\centering
Betaproteobacteria\strut
\end{minipage} & \begin{minipage}[t]{0.44\columnwidth}\centering
Betaproteobacteria\_unclassified\strut
\end{minipage}\tabularnewline
\begin{minipage}[t]{0.39\columnwidth}\centering
Gammaproteobacteria\strut
\end{minipage} & \begin{minipage}[t]{0.44\columnwidth}\centering
Pseudomonadales\strut
\end{minipage}\tabularnewline
\begin{minipage}[t]{0.39\columnwidth}\centering
Opitutae\strut
\end{minipage} & \begin{minipage}[t]{0.44\columnwidth}\centering
Opitutae\_unclassified\strut
\end{minipage}\tabularnewline
\begin{minipage}[t]{0.39\columnwidth}\centering
Gammaproteobacteria\strut
\end{minipage} & \begin{minipage}[t]{0.44\columnwidth}\centering
Pseudomonadales\strut
\end{minipage}\tabularnewline
\begin{minipage}[t]{0.39\columnwidth}\centering
Actinobacteria\strut
\end{minipage} & \begin{minipage}[t]{0.44\columnwidth}\centering
Actinomycetales\strut
\end{minipage}\tabularnewline
\begin{minipage}[t]{0.39\columnwidth}\centering
Betaproteobacteria\strut
\end{minipage} & \begin{minipage}[t]{0.44\columnwidth}\centering
Burkholderiales\strut
\end{minipage}\tabularnewline
\begin{minipage}[t]{0.39\columnwidth}\centering
Betaproteobacteria\strut
\end{minipage} & \begin{minipage}[t]{0.44\columnwidth}\centering
Burkholderiales\strut
\end{minipage}\tabularnewline
\begin{minipage}[t]{0.39\columnwidth}\centering
Actinobacteria\strut
\end{minipage} & \begin{minipage}[t]{0.44\columnwidth}\centering
Actinomycetales\strut
\end{minipage}\tabularnewline
\begin{minipage}[t]{0.39\columnwidth}\centering
Sphingobacteriia\strut
\end{minipage} & \begin{minipage}[t]{0.44\columnwidth}\centering
Sphingobacteriales\strut
\end{minipage}\tabularnewline
\begin{minipage}[t]{0.39\columnwidth}\centering
Alphaproteobacteria\strut
\end{minipage} & \begin{minipage}[t]{0.44\columnwidth}\centering
Sphingomonadales\strut
\end{minipage}\tabularnewline
\begin{minipage}[t]{0.39\columnwidth}\centering
Flavobacteriia\strut
\end{minipage} & \begin{minipage}[t]{0.44\columnwidth}\centering
Flavobacteriales\strut
\end{minipage}\tabularnewline
\begin{minipage}[t]{0.39\columnwidth}\centering
Alphaproteobacteria\strut
\end{minipage} & \begin{minipage}[t]{0.44\columnwidth}\centering
Rhizobiales\strut
\end{minipage}\tabularnewline
\begin{minipage}[t]{0.39\columnwidth}\centering
Betaproteobacteria\strut
\end{minipage} & \begin{minipage}[t]{0.44\columnwidth}\centering
Burkholderiales\strut
\end{minipage}\tabularnewline
\begin{minipage}[t]{0.39\columnwidth}\centering
Betaproteobacteria\strut
\end{minipage} & \begin{minipage}[t]{0.44\columnwidth}\centering
Burkholderiales\strut
\end{minipage}\tabularnewline
\begin{minipage}[t]{0.39\columnwidth}\centering
Sphingobacteriia\strut
\end{minipage} & \begin{minipage}[t]{0.44\columnwidth}\centering
Sphingobacteriales\strut
\end{minipage}\tabularnewline
\begin{minipage}[t]{0.39\columnwidth}\centering
Actinobacteria\strut
\end{minipage} & \begin{minipage}[t]{0.44\columnwidth}\centering
Solirubrobacterales\strut
\end{minipage}\tabularnewline
\bottomrule
\end{longtable}

\begin{longtable}[]{@{}cc@{}}
\toprule
\begin{minipage}[b]{0.44\columnwidth}\centering
Family\strut
\end{minipage} & \begin{minipage}[b]{0.46\columnwidth}\centering
Genus\strut
\end{minipage}\tabularnewline
\midrule
\endhead
\begin{minipage}[t]{0.44\columnwidth}\centering
Pseudomonadaceae\strut
\end{minipage} & \begin{minipage}[t]{0.46\columnwidth}\centering
Pseudomonas\strut
\end{minipage}\tabularnewline
\begin{minipage}[t]{0.44\columnwidth}\centering
Proteobacteria\_unclassified\strut
\end{minipage} & \begin{minipage}[t]{0.46\columnwidth}\centering
Proteobacteria\_unclassified\strut
\end{minipage}\tabularnewline
\begin{minipage}[t]{0.44\columnwidth}\centering
Betaproteobacteria\_unclassified\strut
\end{minipage} & \begin{minipage}[t]{0.46\columnwidth}\centering
Betaproteobacteria\_unclassified\strut
\end{minipage}\tabularnewline
\begin{minipage}[t]{0.44\columnwidth}\centering
Pseudomonadaceae\strut
\end{minipage} & \begin{minipage}[t]{0.46\columnwidth}\centering
Pseudomonas\strut
\end{minipage}\tabularnewline
\begin{minipage}[t]{0.44\columnwidth}\centering
Opitutae\_unclassified\strut
\end{minipage} & \begin{minipage}[t]{0.46\columnwidth}\centering
Opitutae\_unclassified\strut
\end{minipage}\tabularnewline
\begin{minipage}[t]{0.44\columnwidth}\centering
Pseudomonadaceae\strut
\end{minipage} & \begin{minipage}[t]{0.46\columnwidth}\centering
Pseudomonas\strut
\end{minipage}\tabularnewline
\begin{minipage}[t]{0.44\columnwidth}\centering
Micrococcaceae\strut
\end{minipage} & \begin{minipage}[t]{0.46\columnwidth}\centering
Micrococcus\strut
\end{minipage}\tabularnewline
\begin{minipage}[t]{0.44\columnwidth}\centering
Comamonadaceae\strut
\end{minipage} & \begin{minipage}[t]{0.46\columnwidth}\centering
Comamonas\strut
\end{minipage}\tabularnewline
\begin{minipage}[t]{0.44\columnwidth}\centering
Oxalobacteraceae\strut
\end{minipage} & \begin{minipage}[t]{0.46\columnwidth}\centering
Oxalobacteraceae\_unclassified\strut
\end{minipage}\tabularnewline
\begin{minipage}[t]{0.44\columnwidth}\centering
Micrococcaceae\strut
\end{minipage} & \begin{minipage}[t]{0.46\columnwidth}\centering
Arthrobacter\strut
\end{minipage}\tabularnewline
\begin{minipage}[t]{0.44\columnwidth}\centering
Sphingobacteriaceae\strut
\end{minipage} & \begin{minipage}[t]{0.46\columnwidth}\centering
Pedobacter\strut
\end{minipage}\tabularnewline
\begin{minipage}[t]{0.44\columnwidth}\centering
Sphingomonadaceae\strut
\end{minipage} & \begin{minipage}[t]{0.46\columnwidth}\centering
Sphingomonas\strut
\end{minipage}\tabularnewline
\begin{minipage}[t]{0.44\columnwidth}\centering
Flavobacteriaceae\strut
\end{minipage} & \begin{minipage}[t]{0.46\columnwidth}\centering
Flavobacterium\strut
\end{minipage}\tabularnewline
\begin{minipage}[t]{0.44\columnwidth}\centering
Bradyrhizobiaceae\strut
\end{minipage} & \begin{minipage}[t]{0.46\columnwidth}\centering
Bradyrhizobium\strut
\end{minipage}\tabularnewline
\begin{minipage}[t]{0.44\columnwidth}\centering
Oxalobacteraceae\strut
\end{minipage} & \begin{minipage}[t]{0.46\columnwidth}\centering
Duganella\strut
\end{minipage}\tabularnewline
\begin{minipage}[t]{0.44\columnwidth}\centering
Comamonadaceae\strut
\end{minipage} & \begin{minipage}[t]{0.46\columnwidth}\centering
Comamonadaceae\_unclassified\strut
\end{minipage}\tabularnewline
\begin{minipage}[t]{0.44\columnwidth}\centering
Sphingobacteriaceae\strut
\end{minipage} & \begin{minipage}[t]{0.46\columnwidth}\centering
Sphingobacteriaceae\_unclassified\strut
\end{minipage}\tabularnewline
\begin{minipage}[t]{0.44\columnwidth}\centering
Solirubrobacteraceae\strut
\end{minipage} & \begin{minipage}[t]{0.46\columnwidth}\centering
Solirubrobacter\strut
\end{minipage}\tabularnewline
\bottomrule
\end{longtable}

\begin{Shaded}
\begin{Highlighting}[]
\KeywordTok{pander}\NormalTok{(soil.core.increasing, }\DataTypeTok{caption =} \StringTok{"Core taxa found in soils that get more common along the transect."}\NormalTok{)}
\end{Highlighting}
\end{Shaded}

\begin{longtable}[]{@{}ccccc@{}}
\caption{Core taxa found in soils that get more common along the
transect. (continued below)}\tabularnewline
\toprule
\begin{minipage}[b]{0.13\columnwidth}\centering
OTU\strut
\end{minipage} & \begin{minipage}[b]{0.14\columnwidth}\centering
slope\strut
\end{minipage} & \begin{minipage}[b]{0.14\columnwidth}\centering
pval\strut
\end{minipage} & \begin{minipage}[b]{0.13\columnwidth}\centering
Domain\strut
\end{minipage} & \begin{minipage}[b]{0.21\columnwidth}\centering
Phylum\strut
\end{minipage}\tabularnewline
\midrule
\endfirsthead
\toprule
\begin{minipage}[b]{0.13\columnwidth}\centering
OTU\strut
\end{minipage} & \begin{minipage}[b]{0.14\columnwidth}\centering
slope\strut
\end{minipage} & \begin{minipage}[b]{0.14\columnwidth}\centering
pval\strut
\end{minipage} & \begin{minipage}[b]{0.13\columnwidth}\centering
Domain\strut
\end{minipage} & \begin{minipage}[b]{0.21\columnwidth}\centering
Phylum\strut
\end{minipage}\tabularnewline
\midrule
\endhead
\begin{minipage}[t]{0.13\columnwidth}\centering
Otu00001\strut
\end{minipage} & \begin{minipage}[t]{0.14\columnwidth}\centering
1.436e-05\strut
\end{minipage} & \begin{minipage}[t]{0.14\columnwidth}\centering
0.07999\strut
\end{minipage} & \begin{minipage}[t]{0.13\columnwidth}\centering
Bacteria\strut
\end{minipage} & \begin{minipage}[t]{0.21\columnwidth}\centering
Proteobacteria\strut
\end{minipage}\tabularnewline
\begin{minipage}[t]{0.13\columnwidth}\centering
Otu00002\strut
\end{minipage} & \begin{minipage}[t]{0.14\columnwidth}\centering
0.0002115\strut
\end{minipage} & \begin{minipage}[t]{0.14\columnwidth}\centering
0.002237\strut
\end{minipage} & \begin{minipage}[t]{0.13\columnwidth}\centering
Bacteria\strut
\end{minipage} & \begin{minipage}[t]{0.21\columnwidth}\centering
Actinobacteria\strut
\end{minipage}\tabularnewline
\begin{minipage}[t]{0.13\columnwidth}\centering
Otu00003\strut
\end{minipage} & \begin{minipage}[t]{0.14\columnwidth}\centering
9.899e-05\strut
\end{minipage} & \begin{minipage}[t]{0.14\columnwidth}\centering
0.006441\strut
\end{minipage} & \begin{minipage}[t]{0.13\columnwidth}\centering
Bacteria\strut
\end{minipage} & \begin{minipage}[t]{0.21\columnwidth}\centering
Verrucomicrobia\strut
\end{minipage}\tabularnewline
\begin{minipage}[t]{0.13\columnwidth}\centering
Otu00005\strut
\end{minipage} & \begin{minipage}[t]{0.14\columnwidth}\centering
3.61e-05\strut
\end{minipage} & \begin{minipage}[t]{0.14\columnwidth}\centering
0.01737\strut
\end{minipage} & \begin{minipage}[t]{0.13\columnwidth}\centering
Bacteria\strut
\end{minipage} & \begin{minipage}[t]{0.21\columnwidth}\centering
Bacteroidetes\strut
\end{minipage}\tabularnewline
\begin{minipage}[t]{0.13\columnwidth}\centering
Otu00006\strut
\end{minipage} & \begin{minipage}[t]{0.14\columnwidth}\centering
6.575e-06\strut
\end{minipage} & \begin{minipage}[t]{0.14\columnwidth}\centering
0.1618\strut
\end{minipage} & \begin{minipage}[t]{0.13\columnwidth}\centering
Bacteria\strut
\end{minipage} & \begin{minipage}[t]{0.21\columnwidth}\centering
Bacteroidetes\strut
\end{minipage}\tabularnewline
\begin{minipage}[t]{0.13\columnwidth}\centering
Otu00012\strut
\end{minipage} & \begin{minipage}[t]{0.14\columnwidth}\centering
7.541e-06\strut
\end{minipage} & \begin{minipage}[t]{0.14\columnwidth}\centering
0.09905\strut
\end{minipage} & \begin{minipage}[t]{0.13\columnwidth}\centering
Bacteria\strut
\end{minipage} & \begin{minipage}[t]{0.21\columnwidth}\centering
Proteobacteria\strut
\end{minipage}\tabularnewline
\begin{minipage}[t]{0.13\columnwidth}\centering
Otu00014\strut
\end{minipage} & \begin{minipage}[t]{0.14\columnwidth}\centering
8.464e-05\strut
\end{minipage} & \begin{minipage}[t]{0.14\columnwidth}\centering
0.0007891\strut
\end{minipage} & \begin{minipage}[t]{0.13\columnwidth}\centering
Bacteria\strut
\end{minipage} & \begin{minipage}[t]{0.21\columnwidth}\centering
Actinobacteria\strut
\end{minipage}\tabularnewline
\begin{minipage}[t]{0.13\columnwidth}\centering
Otu00023\strut
\end{minipage} & \begin{minipage}[t]{0.14\columnwidth}\centering
3.267e-07\strut
\end{minipage} & \begin{minipage}[t]{0.14\columnwidth}\centering
0.8\strut
\end{minipage} & \begin{minipage}[t]{0.13\columnwidth}\centering
Bacteria\strut
\end{minipage} & \begin{minipage}[t]{0.21\columnwidth}\centering
Proteobacteria\strut
\end{minipage}\tabularnewline
\begin{minipage}[t]{0.13\columnwidth}\centering
Otu00029\strut
\end{minipage} & \begin{minipage}[t]{0.14\columnwidth}\centering
3.32e-05\strut
\end{minipage} & \begin{minipage}[t]{0.14\columnwidth}\centering
0.004456\strut
\end{minipage} & \begin{minipage}[t]{0.13\columnwidth}\centering
Bacteria\strut
\end{minipage} & \begin{minipage}[t]{0.21\columnwidth}\centering
Actinobacteria\strut
\end{minipage}\tabularnewline
\begin{minipage}[t]{0.13\columnwidth}\centering
Otu00032\strut
\end{minipage} & \begin{minipage}[t]{0.14\columnwidth}\centering
3.56e-06\strut
\end{minipage} & \begin{minipage}[t]{0.14\columnwidth}\centering
0.8341\strut
\end{minipage} & \begin{minipage}[t]{0.13\columnwidth}\centering
Bacteria\strut
\end{minipage} & \begin{minipage}[t]{0.21\columnwidth}\centering
Bacteroidetes\strut
\end{minipage}\tabularnewline
\begin{minipage}[t]{0.13\columnwidth}\centering
Otu00033\strut
\end{minipage} & \begin{minipage}[t]{0.14\columnwidth}\centering
9.129e-06\strut
\end{minipage} & \begin{minipage}[t]{0.14\columnwidth}\centering
0.7085\strut
\end{minipage} & \begin{minipage}[t]{0.13\columnwidth}\centering
Bacteria\strut
\end{minipage} & \begin{minipage}[t]{0.21\columnwidth}\centering
Proteobacteria\strut
\end{minipage}\tabularnewline
\bottomrule
\end{longtable}

\begin{longtable}[]{@{}cc@{}}
\caption{Table continues below}\tabularnewline
\toprule
\begin{minipage}[b]{0.38\columnwidth}\centering
Class\strut
\end{minipage} & \begin{minipage}[b]{0.39\columnwidth}\centering
Order\strut
\end{minipage}\tabularnewline
\midrule
\endfirsthead
\toprule
\begin{minipage}[b]{0.38\columnwidth}\centering
Class\strut
\end{minipage} & \begin{minipage}[b]{0.39\columnwidth}\centering
Order\strut
\end{minipage}\tabularnewline
\midrule
\endhead
\begin{minipage}[t]{0.38\columnwidth}\centering
Betaproteobacteria\strut
\end{minipage} & \begin{minipage}[t]{0.39\columnwidth}\centering
Burkholderiales\strut
\end{minipage}\tabularnewline
\begin{minipage}[t]{0.38\columnwidth}\centering
Actinobacteria\strut
\end{minipage} & \begin{minipage}[t]{0.39\columnwidth}\centering
Actinomycetales\strut
\end{minipage}\tabularnewline
\begin{minipage}[t]{0.38\columnwidth}\centering
Spartobacteria\strut
\end{minipage} & \begin{minipage}[t]{0.39\columnwidth}\centering
Spartobacteria\_unclassified\strut
\end{minipage}\tabularnewline
\begin{minipage}[t]{0.38\columnwidth}\centering
Sphingobacteriia\strut
\end{minipage} & \begin{minipage}[t]{0.39\columnwidth}\centering
Sphingobacteriales\strut
\end{minipage}\tabularnewline
\begin{minipage}[t]{0.38\columnwidth}\centering
Sphingobacteriia\strut
\end{minipage} & \begin{minipage}[t]{0.39\columnwidth}\centering
Sphingobacteriales\strut
\end{minipage}\tabularnewline
\begin{minipage}[t]{0.38\columnwidth}\centering
Betaproteobacteria\strut
\end{minipage} & \begin{minipage}[t]{0.39\columnwidth}\centering
Burkholderiales\strut
\end{minipage}\tabularnewline
\begin{minipage}[t]{0.38\columnwidth}\centering
Actinobacteria\strut
\end{minipage} & \begin{minipage}[t]{0.39\columnwidth}\centering
Actinomycetales\strut
\end{minipage}\tabularnewline
\begin{minipage}[t]{0.38\columnwidth}\centering
Gammaproteobacteria\strut
\end{minipage} & \begin{minipage}[t]{0.39\columnwidth}\centering
Pseudomonadales\strut
\end{minipage}\tabularnewline
\begin{minipage}[t]{0.38\columnwidth}\centering
Actinobacteria\strut
\end{minipage} & \begin{minipage}[t]{0.39\columnwidth}\centering
Actinomycetales\strut
\end{minipage}\tabularnewline
\begin{minipage}[t]{0.38\columnwidth}\centering
Bacteroidetes\_unclassified\strut
\end{minipage} & \begin{minipage}[t]{0.39\columnwidth}\centering
Bacteroidetes\_unclassified\strut
\end{minipage}\tabularnewline
\begin{minipage}[t]{0.38\columnwidth}\centering
Alphaproteobacteria\strut
\end{minipage} & \begin{minipage}[t]{0.39\columnwidth}\centering
Rhizobiales\strut
\end{minipage}\tabularnewline
\bottomrule
\end{longtable}

\begin{longtable}[]{@{}cc@{}}
\toprule
\begin{minipage}[b]{0.41\columnwidth}\centering
Family\strut
\end{minipage} & \begin{minipage}[b]{0.41\columnwidth}\centering
Genus\strut
\end{minipage}\tabularnewline
\midrule
\endhead
\begin{minipage}[t]{0.41\columnwidth}\centering
Comamonadaceae\strut
\end{minipage} & \begin{minipage}[t]{0.41\columnwidth}\centering
Comamonadaceae\_unclassified\strut
\end{minipage}\tabularnewline
\begin{minipage}[t]{0.41\columnwidth}\centering
Actinomycetales\_unclassified\strut
\end{minipage} & \begin{minipage}[t]{0.41\columnwidth}\centering
Actinomycetales\_unclassified\strut
\end{minipage}\tabularnewline
\begin{minipage}[t]{0.41\columnwidth}\centering
Spartobacteria\_unclassified\strut
\end{minipage} & \begin{minipage}[t]{0.41\columnwidth}\centering
Spartobacteria\_unclassified\strut
\end{minipage}\tabularnewline
\begin{minipage}[t]{0.41\columnwidth}\centering
Chitinophagaceae\strut
\end{minipage} & \begin{minipage}[t]{0.41\columnwidth}\centering
Sediminibacterium\strut
\end{minipage}\tabularnewline
\begin{minipage}[t]{0.41\columnwidth}\centering
Saprospiraceae\strut
\end{minipage} & \begin{minipage}[t]{0.41\columnwidth}\centering
Saprospiraceae\_unclassified\strut
\end{minipage}\tabularnewline
\begin{minipage}[t]{0.41\columnwidth}\centering
Comamonadaceae\strut
\end{minipage} & \begin{minipage}[t]{0.41\columnwidth}\centering
Comamonadaceae\_unclassified\strut
\end{minipage}\tabularnewline
\begin{minipage}[t]{0.41\columnwidth}\centering
Actinomycetales\_unclassified\strut
\end{minipage} & \begin{minipage}[t]{0.41\columnwidth}\centering
Actinomycetales\_unclassified\strut
\end{minipage}\tabularnewline
\begin{minipage}[t]{0.41\columnwidth}\centering
Moraxellaceae\strut
\end{minipage} & \begin{minipage}[t]{0.41\columnwidth}\centering
Acinetobacter\strut
\end{minipage}\tabularnewline
\begin{minipage}[t]{0.41\columnwidth}\centering
Actinomycetales\_unclassified\strut
\end{minipage} & \begin{minipage}[t]{0.41\columnwidth}\centering
Actinomycetales\_unclassified\strut
\end{minipage}\tabularnewline
\begin{minipage}[t]{0.41\columnwidth}\centering
Bacteroidetes\_unclassified\strut
\end{minipage} & \begin{minipage}[t]{0.41\columnwidth}\centering
Bacteroidetes\_unclassified\strut
\end{minipage}\tabularnewline
\begin{minipage}[t]{0.41\columnwidth}\centering
Rhizobiales\_unclassified\strut
\end{minipage} & \begin{minipage}[t]{0.41\columnwidth}\centering
Rhizobiales\_unclassified\strut
\end{minipage}\tabularnewline
\bottomrule
\end{longtable}

\begin{Shaded}
\begin{Highlighting}[]
\CommentTok{# p1 <- as.data.frame(OTUsREL[,nonsoil.core.increasing$OTU]) %>% }
\CommentTok{#   rownames_to_column("sampleID") %>% }
\CommentTok{#   left_join(rownames_to_column(design, "sampleID")) %>% }
\CommentTok{#   gather(OTU, rel_abund, -station, -molecule, -type, -distance, -sampleID) %>% }
\CommentTok{#   filter(molecule == "DNA") %>% left_join(OTU.tax) %>% }
\CommentTok{#   mutate(taxon = paste(Phylum, Class, Order, Family, Genus)) %>% }
\CommentTok{#   ggplot(aes(x = distance, y = rel_abund, group = OTU)) + }
\CommentTok{#   #geom_point(alpha = 0.5) + }
\CommentTok{#   geom_line(stat = "smooth", alpha = 0.5, size = 1,}
\CommentTok{#             color = "black", method = "loess", span = 1, se = FALSE) + }
\CommentTok{#   scale_x_reverse() +}
\CommentTok{#   scale_y_log10(labels = scales::scientific) +}
\CommentTok{#   theme(legend.position = "none") +}
\CommentTok{#   guides(color = guide_legend(ncol = 1)) +}
\CommentTok{#   labs(x = "",}
\CommentTok{#        y = "Relative Abundance",}
\CommentTok{#        title = "Absent from soil and significantly increasing")}
\CommentTok{# }
\CommentTok{# p2 <- as.data.frame(OTUsREL[,soil.core.increasing$OTU]) %>% }
\CommentTok{#   rownames_to_column("sampleID") %>% }
\CommentTok{#   left_join(rownames_to_column(design, "sampleID")) %>% }
\CommentTok{#   gather(OTU, rel_abund, -station, -molecule, -type, -distance, -sampleID) %>% }
\CommentTok{#   filter(molecule == "DNA") %>% left_join(OTU.tax) %>% }
\CommentTok{#   mutate(taxon = paste(Class, Order)) %>% }
\CommentTok{#   ggplot(aes(x = distance, y = rel_abund, group = OTU)) + }
\CommentTok{#   #geom_point(alpha = 0.5) + }
\CommentTok{#   geom_line(stat = "smooth", alpha = 0.5, size = 1,}
\CommentTok{#             color = "black", method = "loess", span = 1, se = FALSE) + }
\CommentTok{#   scale_x_reverse() + }
\CommentTok{#   scale_y_log10(labels = scales::scientific) +}
\CommentTok{#   theme(legend.position = "none") +}
\CommentTok{#   guides(color = guide_legend(ncol = 1)) +}
\CommentTok{#   labs(x = "",}
\CommentTok{#        y = "Relative Abundance",}
\CommentTok{#        title = "Present in soil and significantly increasing")}
\CommentTok{# }
\CommentTok{# p3 <- as.data.frame(OTUsREL[,soil.core.decreasing$OTU]) %>% }
\CommentTok{#   rownames_to_column("sampleID") %>% }
\CommentTok{#   left_join(rownames_to_column(design, "sampleID")) %>% }
\CommentTok{#   gather(OTU, rel_abund, -station, -molecule, -type, -distance, -sampleID) %>% }
\CommentTok{#   filter(molecule == "DNA") %>% left_join(OTU.tax) %>% }
\CommentTok{#   mutate(taxon = paste(Class, Order)) %>% }
\CommentTok{#   ggplot(aes(x = distance, y = rel_abund, group = OTU)) + }
\CommentTok{#   #geom_point(alpha = 0.5) + }
\CommentTok{#   geom_line(stat = "smooth", alpha = 0.5, size = 1,}
\CommentTok{#             color = "black", method = "loess", span = 1, se = FALSE) + }
\CommentTok{#   scale_x_reverse() +}
\CommentTok{#   scale_y_log10(labels = scales::scientific) +}
\CommentTok{#   theme(legend.position = "none") +}
\CommentTok{#   guides(color = guide_legend(ncol = 1)) +}
\CommentTok{#   labs(x = "Reservoir Transect (m)",}
\CommentTok{#        y = "Relative Abundance",}
\CommentTok{#        title = "Present in soil and significantly decreasing")}
\CommentTok{# }
\CommentTok{# cowplot::plot_grid(p1, p2, p3, align = "hv", labels = "AUTO", ncol = 1)}

\NormalTok{df1 <-}\StringTok{ }\KeywordTok{as.data.frame}\NormalTok{(OTUsREL[,nonsoil.core.increasing}\OperatorTok{$}\NormalTok{OTU]) }\OperatorTok\StringTok{ }
\StringTok{  }\KeywordTok{rownames_to_column}\NormalTok{(}\StringTok{"sampleID"}\NormalTok{) }\OperatorTok\StringTok{ }
\StringTok{  }\KeywordTok{left_join}\NormalTok{(}\KeywordTok{rownames_to_column}\NormalTok{(design, }\StringTok{"sampleID"}\NormalTok{)) }\OperatorTok\StringTok{ }
\StringTok{  }\KeywordTok{gather}\NormalTok{(OTU, rel_abund, }\OperatorTok{-}\NormalTok{station, }\OperatorTok{-}\NormalTok{molecule, }\OperatorTok{-}\NormalTok{type, }\OperatorTok{-}\NormalTok{distance, }\OperatorTok{-}\NormalTok{sampleID) }\OperatorTok\StringTok{ }
\StringTok{  }\KeywordTok{filter}\NormalTok{(molecule }\OperatorTok{==}\StringTok{ "DNA"}\NormalTok{) }\OperatorTok\StringTok{ }\KeywordTok{left_join}\NormalTok{(OTU.tax) }\OperatorTok\StringTok{ }
\StringTok{  }\KeywordTok{mutate}\NormalTok{(}\DataTypeTok{soils =} \StringTok{"Absent from soils"}\NormalTok{, }\DataTypeTok{change =} \StringTok{"Increasing"}\NormalTok{)}
\end{Highlighting}
\end{Shaded}

\begin{verbatim}
## Warning: Column `OTU` joining character vector and factor, coercing into
## character vector
\end{verbatim}

\begin{Shaded}
\begin{Highlighting}[]
\NormalTok{n1 <-}\StringTok{ }\KeywordTok{length}\NormalTok{(}\KeywordTok{unique}\NormalTok{(df1}\OperatorTok{$}\NormalTok{OTU))}

\NormalTok{df2 <-}\StringTok{ }\KeywordTok{as.data.frame}\NormalTok{(OTUsREL[,soil.core.increasing}\OperatorTok{$}\NormalTok{OTU]) }\OperatorTok\StringTok{ }
\StringTok{  }\KeywordTok{rownames_to_column}\NormalTok{(}\StringTok{"sampleID"}\NormalTok{) }\OperatorTok\StringTok{ }
\StringTok{  }\KeywordTok{left_join}\NormalTok{(}\KeywordTok{rownames_to_column}\NormalTok{(design, }\StringTok{"sampleID"}\NormalTok{)) }\OperatorTok\StringTok{ }
\StringTok{  }\KeywordTok{gather}\NormalTok{(OTU, rel_abund, }\OperatorTok{-}\NormalTok{station, }\OperatorTok{-}\NormalTok{molecule, }\OperatorTok{-}\NormalTok{type, }\OperatorTok{-}\NormalTok{distance, }\OperatorTok{-}\NormalTok{sampleID) }\OperatorTok\StringTok{ }
\StringTok{  }\KeywordTok{filter}\NormalTok{(molecule }\OperatorTok{==}\StringTok{ "DNA"}\NormalTok{) }\OperatorTok\StringTok{ }\KeywordTok{left_join}\NormalTok{(OTU.tax) }\OperatorTok\StringTok{ }
\StringTok{  }\KeywordTok{mutate}\NormalTok{(}\DataTypeTok{soils =} \StringTok{"Present in soils"}\NormalTok{, }\DataTypeTok{change =} \StringTok{"Increasing"}\NormalTok{)}
\end{Highlighting}
\end{Shaded}

\begin{verbatim}
## Warning: Column `OTU` joining character vector and factor, coercing into
## character vector
\end{verbatim}

\begin{Shaded}
\begin{Highlighting}[]
\NormalTok{n2 <-}\StringTok{ }\KeywordTok{length}\NormalTok{(}\KeywordTok{unique}\NormalTok{(df2}\OperatorTok{$}\NormalTok{OTU))}

\NormalTok{df3 <-}\StringTok{ }\KeywordTok{as.data.frame}\NormalTok{(OTUsREL[,soil.core.decreasing}\OperatorTok{$}\NormalTok{OTU]) }\OperatorTok\StringTok{ }
\StringTok{  }\KeywordTok{rownames_to_column}\NormalTok{(}\StringTok{"sampleID"}\NormalTok{) }\OperatorTok\StringTok{ }
\StringTok{  }\KeywordTok{left_join}\NormalTok{(}\KeywordTok{rownames_to_column}\NormalTok{(design, }\StringTok{"sampleID"}\NormalTok{)) }\OperatorTok\StringTok{ }
\StringTok{  }\KeywordTok{gather}\NormalTok{(OTU, rel_abund, }\OperatorTok{-}\NormalTok{station, }\OperatorTok{-}\NormalTok{molecule, }\OperatorTok{-}\NormalTok{type, }\OperatorTok{-}\NormalTok{distance, }\OperatorTok{-}\NormalTok{sampleID) }\OperatorTok\StringTok{ }
\StringTok{  }\KeywordTok{filter}\NormalTok{(molecule }\OperatorTok{==}\StringTok{ "DNA"}\NormalTok{) }\OperatorTok\StringTok{ }\KeywordTok{left_join}\NormalTok{(OTU.tax) }\OperatorTok\StringTok{ }
\StringTok{  }\KeywordTok{mutate}\NormalTok{(}\DataTypeTok{soils =} \StringTok{"Present in soils"}\NormalTok{, }\DataTypeTok{change =} \StringTok{"Decreasing"}\NormalTok{)}
\end{Highlighting}
\end{Shaded}

\begin{verbatim}
## Warning: Column `OTU` joining character vector and factor, coercing into
## character vector
\end{verbatim}

\begin{Shaded}
\begin{Highlighting}[]
\NormalTok{n3 <-}\StringTok{ }\KeywordTok{length}\NormalTok{(}\KeywordTok{unique}\NormalTok{(df3}\OperatorTok{$}\NormalTok{OTU))}

\NormalTok{df4 <-}\StringTok{ }\KeywordTok{as.data.frame}\NormalTok{(OTUsREL[,nonsoil.core.decreasing}\OperatorTok{$}\NormalTok{OTU]) }\OperatorTok\StringTok{ }
\StringTok{  }\KeywordTok{rownames_to_column}\NormalTok{(}\StringTok{"sampleID"}\NormalTok{) }\OperatorTok\StringTok{ }
\StringTok{  }\KeywordTok{left_join}\NormalTok{(}\KeywordTok{rownames_to_column}\NormalTok{(design, }\StringTok{"sampleID"}\NormalTok{)) }\OperatorTok\StringTok{ }
\StringTok{  }\KeywordTok{gather}\NormalTok{(OTU, rel_abund, }\OperatorTok{-}\NormalTok{station, }\OperatorTok{-}\NormalTok{molecule, }\OperatorTok{-}\NormalTok{type, }\OperatorTok{-}\NormalTok{distance, }\OperatorTok{-}\NormalTok{sampleID) }\OperatorTok\StringTok{ }
\StringTok{  }\KeywordTok{filter}\NormalTok{(molecule }\OperatorTok{==}\StringTok{ "DNA"}\NormalTok{) }\OperatorTok\StringTok{ }\KeywordTok{left_join}\NormalTok{(OTU.tax) }\OperatorTok\StringTok{ }
\StringTok{  }\KeywordTok{mutate}\NormalTok{(}\DataTypeTok{soils =} \StringTok{"Absent from soils"}\NormalTok{, }\DataTypeTok{change =} \StringTok{"Decreasing"}\NormalTok{)}
\end{Highlighting}
\end{Shaded}

\begin{verbatim}
## Warning: Column `OTU` joining character vector and factor, coercing into
## character vector
\end{verbatim}

\begin{Shaded}
\begin{Highlighting}[]
\NormalTok{n4 <-}\StringTok{ }\KeywordTok{length}\NormalTok{(}\KeywordTok{unique}\NormalTok{(df4}\OperatorTok{$}\NormalTok{OTU))}


\NormalTok{df.plot <-}\StringTok{ }\KeywordTok{as_tibble}\NormalTok{(}\KeywordTok{rbind.data.frame}\NormalTok{(df1, df2, df3, df4)) }\OperatorTok\StringTok{ }\KeywordTok{filter}\NormalTok{(type }\OperatorTok{==}\StringTok{ "water"}\NormalTok{)}

\NormalTok{taxon_fate.plot <-}\StringTok{ }\NormalTok{df.plot }\OperatorTok\StringTok{ }\KeywordTok{mutate}\NormalTok{(}\DataTypeTok{rel_abund =} \KeywordTok{ifelse}\NormalTok{(rel_abund }\OperatorTok{==}\StringTok{ }\DecValTok{0}\NormalTok{, }\FloatTok{1e-6}\NormalTok{, rel_abund)) }\OperatorTok\StringTok{ }
\StringTok{  }\KeywordTok{filter}\NormalTok{(soils }\OperatorTok{==}\StringTok{ "Present in soils"}\NormalTok{) }\OperatorTok\StringTok{ }
\StringTok{  }\CommentTok{#mutate(change = ifelse(change == "Increasing", }
\StringTok{  }\CommentTok{#                       paste0("Increasing (n = ", n2,")"),}
\StringTok{  }\CommentTok{#                       paste0("Decreasing (n = ", n3,")"))) %>% }
\StringTok{  }\KeywordTok{ggplot}\NormalTok{(}\KeywordTok{aes}\NormalTok{(}\DataTypeTok{x =}\NormalTok{ distance, }\DataTypeTok{y =}\NormalTok{ rel_abund, }\DataTypeTok{group =}\NormalTok{ OTU)) }\OperatorTok{+}\StringTok{ }
\StringTok{  }\CommentTok{#geom_jitter(alpha = 0.15) + }
\StringTok{  }\KeywordTok{geom_line}\NormalTok{(}\DataTypeTok{stat =} \StringTok{"smooth"}\NormalTok{, }\DataTypeTok{alpha =} \FloatTok{0.3}\NormalTok{, }\DataTypeTok{size =} \DecValTok{1}\NormalTok{,}
            \DataTypeTok{method =} \StringTok{"loess"}\NormalTok{, }\DataTypeTok{span =} \FloatTok{.7}\NormalTok{, }\DataTypeTok{se =} \OtherTok{FALSE}\NormalTok{) }\OperatorTok{+}
\StringTok{  }\KeywordTok{scale_y_log10}\NormalTok{(}\DataTypeTok{labels =}\NormalTok{ scales}\OperatorTok{::}\NormalTok{scientific) }\OperatorTok{+}
\StringTok{  }\KeywordTok{scale_x_continuous}\NormalTok{(}\DataTypeTok{limits =} \KeywordTok{c}\NormalTok{(}\DecValTok{0}\NormalTok{,}\DecValTok{380}\NormalTok{)) }\OperatorTok{+}
\StringTok{  }\CommentTok{#theme(legend.position = "none") +}
\StringTok{  }\CommentTok{#guides(color = guide_legend(ncol = 1)) +}
\StringTok{  }\KeywordTok{labs}\NormalTok{(}\DataTypeTok{x =} \StringTok{"Reservoir distance (m)"}\NormalTok{,}
       \DataTypeTok{y =} \StringTok{"Active relative abundance"}\NormalTok{) }\OperatorTok{+}
\StringTok{  }\KeywordTok{annotate}\NormalTok{(}\StringTok{"text"}\NormalTok{, }\DataTypeTok{x =} \DecValTok{365}\NormalTok{, }\DataTypeTok{y =} \FloatTok{1e-1}\NormalTok{, }\DataTypeTok{size =} \DecValTok{5}\NormalTok{, }\DataTypeTok{hjust =} \DecValTok{1}\NormalTok{, }\DataTypeTok{vjust =} \DecValTok{1}\NormalTok{, }\DataTypeTok{angle =} \DecValTok{90}\NormalTok{,}
           \DataTypeTok{label =} \StringTok{"Maintained"}\NormalTok{) }\OperatorTok{+}
\StringTok{  }\KeywordTok{annotate}\NormalTok{(}\StringTok{"text"}\NormalTok{, }\DataTypeTok{x =} \DecValTok{365}\NormalTok{, }\DataTypeTok{y =} \FloatTok{1e-5}\NormalTok{, }\DataTypeTok{size =} \DecValTok{5}\NormalTok{, }\DataTypeTok{hjust =} \FloatTok{0.5}\NormalTok{, }\DataTypeTok{vjust =} \DecValTok{1}\NormalTok{, }\DataTypeTok{angle =} \DecValTok{90}\NormalTok{,}
           \DataTypeTok{label =} \StringTok{"Decaying"}\NormalTok{) }\OperatorTok{+}
\StringTok{  }\KeywordTok{ggsave}\NormalTok{(}\StringTok{"figures/taxa_origins.pdf"}\NormalTok{)}
\NormalTok{ taxon_fate.plot}
\end{Highlighting}
\end{Shaded}

\begin{center}\includegraphics{ReservoirGradient_files/figure-latex/sig_taxa-1} \end{center}

\begin{Shaded}
\begin{Highlighting}[]
\CommentTok{# how much do the different core components contribute to total abundances}
\NormalTok{in.lake.core.soil.REL <-}\StringTok{ }\KeywordTok{rowSums}\NormalTok{(in.lake.core.from.soils) }\OperatorTok{/}\StringTok{ }\KeywordTok{rowSums}\NormalTok{(w.dna)}
\NormalTok{in.lake.core.water.REL <-}\StringTok{ }\KeywordTok{rowSums}\NormalTok{(in.lake.core.not.soils) }\OperatorTok{/}\StringTok{ }\KeywordTok{rowSums}\NormalTok{(w.dna)}
\end{Highlighting}
\end{Shaded}

\begin{Shaded}
\begin{Highlighting}[]
\KeywordTok{plot_grid}\NormalTok{(transient.plot }\OperatorTok{+}\StringTok{ }\KeywordTok{labs}\NormalTok{(}\DataTypeTok{x =} \StringTok{""}\NormalTok{),}
\NormalTok{          taxon_fate.plot,}
          \DataTypeTok{align =} \StringTok{"hv"}\NormalTok{, }\DataTypeTok{axis =} \StringTok{"rltb"}\NormalTok{,}
          \DataTypeTok{labels =} \StringTok{"auto"}\NormalTok{,}
          \DataTypeTok{ncol =} \DecValTok{1}\NormalTok{) }\OperatorTok{+}
\StringTok{  }\KeywordTok{ggsave}\NormalTok{(}\StringTok{"figures/fate_panel.pdf"}\NormalTok{)}
\end{Highlighting}
\end{Shaded}

\begin{center}\includegraphics{ReservoirGradient_files/figure-latex/fate_panel-1} \end{center}

\begin{Shaded}
\begin{Highlighting}[]
\CommentTok{# soil.mods <- t(soil.core.mods) %>% as.data.frame()}
\CommentTok{# soil.mods$habitat <- "Present in soils"}
\CommentTok{# soil.mods <- soil.mods %>% rownames_to_column(var = "OTU")}
\CommentTok{# nonsoil.mods <- t(nonsoil.core.mods) %>% as.data.frame()}
\CommentTok{# nonsoil.mods$habitat <- "Absent from soils"}
\CommentTok{# nonsoil.mods <- nonsoil.mods %>% rownames_to_column(var = "OTU")}
\CommentTok{# rbind.data.frame(soil.mods, nonsoil.mods) %>% }
\CommentTok{#   filter(pval < 0.05) %>% }
\CommentTok{#   ggplot(aes(x = -slope, fill = habitat, color = habitat)) +}
\CommentTok{#   geom_line(stat = "density", alpha = 0.5, adjust = .8) +}
\CommentTok{#   geom_density(color = NA, adjust = .8, alpha = 0.2)}
\end{Highlighting}
\end{Shaded}

\hypertarget{are-the-persistent-reservoir-taxa-really-representative-look-over-time}{%
\section{Are the ``persistent'' reservoir taxa really representative?
Look over
time\ldots{}}\label{are-the-persistent-reservoir-taxa-really-representative-look-over-time}}

\begin{Shaded}
\begin{Highlighting}[]
\NormalTok{total.OTUs <-}\StringTok{ }\KeywordTok{read.otu}\NormalTok{(}\DataTypeTok{shared =}\NormalTok{ shared, }\DataTypeTok{cutoff =} \StringTok{"0.03"}\NormalTok{)    }\CommentTok{# 97% Similarity}

\CommentTok{# Import Taxonomy}
\NormalTok{total.OTU.tax <-}\StringTok{ }\KeywordTok{read.tax}\NormalTok{(}\DataTypeTok{taxonomy =}\NormalTok{ taxon, }\DataTypeTok{format =} \StringTok{"rdp"}\NormalTok{)}

\CommentTok{# Subset to just the time series sites}
\NormalTok{UL.ts.OTUs <-}\StringTok{ }\NormalTok{total.OTUs[}\KeywordTok{str_which}\NormalTok{(}\KeywordTok{rownames}\NormalTok{(total.OTUs), }\StringTok{"UL"}\NormalTok{),]}

\CommentTok{# make sure OTU table matches up with design order}
\NormalTok{UL.ts.design <-}\StringTok{ }\KeywordTok{read_csv}\NormalTok{(}\StringTok{"data/UL_timeseries_design.csv"}\NormalTok{)}
\NormalTok{UL.ts.OTUs <-}\StringTok{ }\NormalTok{UL.ts.OTUs[}\KeywordTok{match}\NormalTok{(UL.ts.design}\OperatorTok{$}\NormalTok{sample.name, }\KeywordTok{rownames}\NormalTok{(UL.ts.OTUs)),]}
\NormalTok{UL.ts.OTUs.RNA <-}\StringTok{ }\KeywordTok{decostand}\NormalTok{(UL.ts.OTUs[}\KeywordTok{which}\NormalTok{(UL.ts.design}\OperatorTok{$}\NormalTok{sample.type }\OperatorTok{==}\StringTok{ "RNA"}\NormalTok{),], }\DataTypeTok{method =} \StringTok{"total"}\NormalTok{)}
\NormalTok{UL.ts.OTUs.DNA <-}\StringTok{ }\KeywordTok{decostand}\NormalTok{(UL.ts.OTUs[}\KeywordTok{which}\NormalTok{(UL.ts.design}\OperatorTok{$}\NormalTok{sample.type }\OperatorTok{==}\StringTok{ "DNA"}\NormalTok{),], }\DataTypeTok{method =} \StringTok{"total"}\NormalTok{)}


\NormalTok{env.ts.data <-}\StringTok{ }\KeywordTok{read.table}\NormalTok{(}\StringTok{"data/ul-seedbank.env.txt"}\NormalTok{, }\DataTypeTok{sep=}\StringTok{"}\CharTok{\textbackslash{}t}\StringTok{"}\NormalTok{, }\DataTypeTok{header=}\OtherTok{TRUE}\NormalTok{)}
\NormalTok{env.ts.data}\OperatorTok{$}\NormalTok{date <-}\StringTok{ }\KeywordTok{as.Date}\NormalTok{(}\KeywordTok{parse_date_time}\NormalTok{(env.ts.data}\OperatorTok{$}\NormalTok{date, }\StringTok{"m d y"}\NormalTok{))}
\NormalTok{env.ts.data}\OperatorTok{$}\NormalTok{doc[}\KeywordTok{which}\NormalTok{(env.ts.data}\OperatorTok{$}\NormalTok{doc }\OperatorTok{==}\StringTok{ "**"}\NormalTok{)] <-}\StringTok{ }\OtherTok{NA}
\NormalTok{env.ts.data}\OperatorTok{$}\NormalTok{doc <-}\StringTok{ }\KeywordTok{as.numeric}\NormalTok{(env.ts.data}\OperatorTok{$}\NormalTok{doc)}
\KeywordTok{summary}\NormalTok{(env.ts.data)}
\end{Highlighting}
\end{Shaded}

\begin{verbatim}
##    sample.id           date                 temp            spc        
##  Min.   :  1.00   Min.   :2013-04-19   Min.   : 2.21   Min.   :0.3300  
##  1st Qu.: 31.75   1st Qu.:2013-11-20   1st Qu.: 5.50   1st Qu.:0.4600  
##  Median : 62.50   Median :2014-06-23   Median :17.73   Median :0.5320  
##  Mean   : 62.50   Mean   :2014-06-24   Mean   :16.18   Mean   :0.5172  
##  3rd Qu.: 93.25   3rd Qu.:2015-01-25   3rd Qu.:25.05   3rd Qu.:0.5660  
##  Max.   :124.00   Max.   :2015-09-14   Max.   :29.77   Max.   :0.6700  
##                                        NA's   :2       NA's   :2       
##      oxygen          salinity          secchi            ph        
##  Min.   : 1.870   Min.   :0.1500   Min.   :0.200   Min.   : 6.890  
##  1st Qu.: 5.237   1st Qu.:0.2200   1st Qu.:1.200   1st Qu.: 7.920  
##  Median : 8.355   Median :0.2550   Median :1.600   Median : 8.415  
##  Mean   : 8.961   Mean   :0.2487   Mean   :1.668   Mean   : 8.567  
##  3rd Qu.:10.178   3rd Qu.:0.2700   3rd Qu.:2.200   3rd Qu.: 9.123  
##  Max.   :22.240   Max.   :0.3200   Max.   :3.600   Max.   :10.860  
##  NA's   :2        NA's   :2        NA's   :1       NA's   :2       
##       chla              tp                tn              doc        
##  Min.   :  0.92   Min.   :   8.26   Min.   : 0.407   Min.   :  2.00  
##  1st Qu.: 12.63   1st Qu.:  26.30   1st Qu.: 0.882   1st Qu.: 32.25  
##  Median : 37.67   Median :  34.85   Median : 1.210   Median : 61.50  
##  Mean   : 79.25   Mean   :  84.25   Mean   : 1.889   Mean   : 61.57  
##  3rd Qu.:121.31   3rd Qu.:  47.95   3rd Qu.: 1.490   3rd Qu.: 90.75  
##  Max.   :523.56   Max.   :3200.00   Max.   :42.600   Max.   :121.00  
##  NA's   :2        NA's   :2         NA's   :3        NA's   :2       
##       orp             air.temp     
##  Min.   :-41.800   Min.   :-11.60  
##  1st Qu.:  9.325   1st Qu.:  7.00  
##  Median : 21.700   Median : 18.50  
##  Mean   : 50.507   Mean   : 15.57  
##  3rd Qu.:104.975   3rd Qu.: 24.00  
##  Max.   :225.200   Max.   : 32.00  
##  NA's   :68        NA's   :2
\end{verbatim}

\begin{Shaded}
\begin{Highlighting}[]
\NormalTok{UL.ts.design <-}\StringTok{ }\KeywordTok{left_join}\NormalTok{(UL.ts.design, env.ts.data[,}\KeywordTok{c}\NormalTok{(}\StringTok{"sample.id"}\NormalTok{, }\StringTok{"date"}\NormalTok{)])}
\NormalTok{env.ts.data <-}\StringTok{ }\NormalTok{env.ts.data[}\OperatorTok{-}\KeywordTok{which}\NormalTok{(}\OperatorTok{!}\NormalTok{(env.ts.data}\OperatorTok{$}\NormalTok{date }\OperatorTok\StringTok{ }\NormalTok{UL.ts.design}\OperatorTok{$}\NormalTok{date)),]}

\NormalTok{OTUs.in.core <-}\StringTok{ }\NormalTok{UL.ts.OTUs.RNA[, }\KeywordTok{which}\NormalTok{(}\KeywordTok{colnames}\NormalTok{(UL.ts.OTUs) }\OperatorTok\StringTok{ }\NormalTok{df.plot}\OperatorTok{$}\NormalTok{OTU)]}
\KeywordTok{cbind.data.frame}\NormalTok{(UL.ts.design[}\KeywordTok{which}\NormalTok{(UL.ts.design}\OperatorTok{$}\NormalTok{sample.type }\OperatorTok{==}\StringTok{ "RNA"}\NormalTok{),], OTUs.in.core) }\OperatorTok\StringTok{ }\KeywordTok{as_tibble}\NormalTok{() }\OperatorTok\StringTok{ }
\StringTok{  }\KeywordTok{gather}\NormalTok{(}\OperatorTok{-}\NormalTok{sample.name, }\OperatorTok{-}\NormalTok{sample.type, }\OperatorTok{-}\NormalTok{sample.id, }\OperatorTok{-}\NormalTok{date, }\DataTypeTok{key =}\NormalTok{ OTU, }\DataTypeTok{value =}\NormalTok{ rel_abund) }\OperatorTok\StringTok{ }
\StringTok{  }\KeywordTok{mutate}\NormalTok{(}\DataTypeTok{soils =} \KeywordTok{ifelse}\NormalTok{(OTU }\OperatorTok\StringTok{ }\KeywordTok{unique}\NormalTok{(}\KeywordTok{c}\NormalTok{(df2}\OperatorTok{$}\NormalTok{OTU, df3}\OperatorTok{$}\NormalTok{OTU)), }
                        \StringTok{"Present in soils"}\NormalTok{, }\StringTok{"Absent from soils"}\NormalTok{)) }\OperatorTok\StringTok{ }
\StringTok{  }\KeywordTok{mutate}\NormalTok{(}\DataTypeTok{change =} \KeywordTok{ifelse}\NormalTok{(OTU }\OperatorTok\StringTok{ }\KeywordTok{unique}\NormalTok{(}\KeywordTok{c}\NormalTok{(df3}\OperatorTok{$}\NormalTok{OTU, df4}\OperatorTok{$}\NormalTok{OTU)), }
                        \StringTok{"Decreasing"}\NormalTok{, }\StringTok{"Increasing"}\NormalTok{)) }\OperatorTok\StringTok{ }
\StringTok{  }\KeywordTok{mutate}\NormalTok{(}\DataTypeTok{rel_abund =} \KeywordTok{ifelse}\NormalTok{(rel_abund }\OperatorTok{==}\StringTok{ }\DecValTok{0}\NormalTok{, }\FloatTok{1e-6}\NormalTok{, rel_abund)) }\OperatorTok\StringTok{ }
\StringTok{  }\KeywordTok{ggplot}\NormalTok{(}\KeywordTok{aes}\NormalTok{(}\DataTypeTok{x =}\NormalTok{ date, }\DataTypeTok{y =}\NormalTok{ rel_abund, }\DataTypeTok{group =}\NormalTok{ OTU)) }\OperatorTok{+}
\StringTok{  }\KeywordTok{geom_point}\NormalTok{(}\DataTypeTok{alpha =} \FloatTok{.1}\NormalTok{) }\OperatorTok{+}\StringTok{ }
\StringTok{  }\KeywordTok{geom_line}\NormalTok{(}\DataTypeTok{stat =} \StringTok{"smooth"}\NormalTok{, }\DataTypeTok{method =} \StringTok{"loess"}\NormalTok{, }\DataTypeTok{color =} \StringTok{"blue"}\NormalTok{,}
            \DataTypeTok{alpha =} \FloatTok{0.5}\NormalTok{, }\DataTypeTok{span =} \FloatTok{.5}\NormalTok{, }\DataTypeTok{se =}\NormalTok{ F) }\OperatorTok{+}
\StringTok{  }\KeywordTok{geom_vline}\NormalTok{(}\KeywordTok{aes}\NormalTok{(}\DataTypeTok{xintercept =} \KeywordTok{as_date}\NormalTok{(}\StringTok{"2013-07-15"}\NormalTok{))) }\OperatorTok{+}
\StringTok{  }\KeywordTok{scale_y_log10}\NormalTok{() }\OperatorTok{+}\StringTok{ }
\StringTok{  }\KeywordTok{scale_x_date}\NormalTok{(}\DataTypeTok{labels =}\NormalTok{ scales}\OperatorTok{::}\KeywordTok{date_format}\NormalTok{(}\DataTypeTok{format =} \StringTok{"%b %y"}\NormalTok{)) }\OperatorTok{+}
\StringTok{  }\KeywordTok{theme}\NormalTok{(}\DataTypeTok{axis.text.x =} \KeywordTok{element_text}\NormalTok{(}\DataTypeTok{angle =} \DecValTok{45}\NormalTok{, }\DataTypeTok{hjust =} \DecValTok{1}\NormalTok{)) }\OperatorTok{+}
\StringTok{  }\KeywordTok{facet_grid}\NormalTok{(soils }\OperatorTok{~}\StringTok{ }\NormalTok{change) }\OperatorTok{+}
\StringTok{  }\KeywordTok{labs}\NormalTok{(}\DataTypeTok{x =} \StringTok{""}\NormalTok{,}
       \DataTypeTok{y =} \StringTok{"Relative abundance"}\NormalTok{)}
\end{Highlighting}
\end{Shaded}

\begin{center}\includegraphics{ReservoirGradient_files/figure-latex/time_series-1} \end{center}

Many of them do appear to track the seasons quite well, suggesting there
could be a seasonality component to the role of terrestrial inputs into
the reservoir.

\hypertarget{ecosystem-functions}{%
\subsection{Ecosystem functions}\label{ecosystem-functions}}

\begin{Shaded}
\begin{Highlighting}[]
\NormalTok{metab <-}\StringTok{ }\KeywordTok{read.table}\NormalTok{(}\StringTok{"data/res.grad.metab.txt"}\NormalTok{, }\DataTypeTok{sep=}\StringTok{"}\CharTok{\textbackslash{}t}\StringTok{"}\NormalTok{, }\DataTypeTok{header=}\OtherTok{TRUE}\NormalTok{)}
\KeywordTok{colnames}\NormalTok{(metab) <-}\StringTok{ }\KeywordTok{c}\NormalTok{(}\StringTok{"dist"}\NormalTok{, }\StringTok{"BP"}\NormalTok{, }\StringTok{"BR"}\NormalTok{)}
\NormalTok{BGE <-}\StringTok{ }\KeywordTok{round}\NormalTok{((metab}\OperatorTok{$}\NormalTok{BP}\OperatorTok{/}\NormalTok{(metab}\OperatorTok{$}\NormalTok{BP }\OperatorTok{+}\StringTok{ }\NormalTok{metab}\OperatorTok{$}\NormalTok{BR)),}\DecValTok{3}\NormalTok{)}
\NormalTok{metab <-}\StringTok{ }\KeywordTok{cbind}\NormalTok{(metab, BGE)}
\NormalTok{metab <-}\StringTok{ }\NormalTok{metab[}\OperatorTok{-}\KeywordTok{c}\NormalTok{(}\DecValTok{16}\OperatorTok{:}\DecValTok{18}\NormalTok{),]}
\NormalTok{metab}\OperatorTok{$}\NormalTok{dist <-}\StringTok{ }\DecValTok{350} \OperatorTok{-}\StringTok{ }\NormalTok{metab}\OperatorTok{$}\NormalTok{dist}


\CommentTok{# Quadratic regression for BP}
\NormalTok{dist <-}\StringTok{ }\NormalTok{metab}\OperatorTok{$}\NormalTok{dist}
\NormalTok{dist2 <-}\StringTok{ }\NormalTok{metab}\OperatorTok{$}\NormalTok{dist}\OperatorTok{^}\DecValTok{2}
\NormalTok{BP.fit <-}\StringTok{ }\KeywordTok{lm}\NormalTok{(metab}\OperatorTok{$}\NormalTok{BP }\OperatorTok{~}\StringTok{ }\NormalTok{dist }\OperatorTok{+}\StringTok{ }\NormalTok{dist2)}
\NormalTok{BP.R2 <-}\StringTok{ }\KeywordTok{round}\NormalTok{(}\KeywordTok{summary}\NormalTok{(BP.fit)}\OperatorTok{$}\NormalTok{r.squared, }\DecValTok{2}\NormalTok{)}

\CommentTok{# Simple linear regression for BR}
\NormalTok{BR.fit <-}\StringTok{ }\KeywordTok{lm}\NormalTok{(metab}\OperatorTok{$}\NormalTok{BR }\OperatorTok{~}\StringTok{ }\NormalTok{metab}\OperatorTok{$}\NormalTok{dist)}
\NormalTok{BR.R2 <-}\StringTok{ }\KeywordTok{round}\NormalTok{(}\KeywordTok{summary}\NormalTok{(BR.fit)}\OperatorTok{$}\NormalTok{r.squared, }\DecValTok{2}\NormalTok{)}
\NormalTok{BR.int <-}\StringTok{ }\NormalTok{BR.fit}\OperatorTok{$}\NormalTok{coefficients[}\DecValTok{1}\NormalTok{]}
\NormalTok{BR.slp <-}\StringTok{ }\NormalTok{BR.fit}\OperatorTok{$}\NormalTok{coefficients[}\DecValTok{2}\NormalTok{]}

\CommentTok{# Simple linear regression for BGE}
\NormalTok{BGE.fit <-}\StringTok{ }\KeywordTok{lm}\NormalTok{(metab}\OperatorTok{$}\NormalTok{BGE }\OperatorTok{~}\StringTok{ }\NormalTok{metab}\OperatorTok{$}\NormalTok{dist)}
\NormalTok{BGE.R2 <-}\StringTok{ }\KeywordTok{round}\NormalTok{(}\KeywordTok{summary}\NormalTok{(BGE.fit)}\OperatorTok{$}\NormalTok{r.squared, }\DecValTok{2}\NormalTok{)}
\NormalTok{BGE.int <-}\StringTok{ }\NormalTok{BGE.fit}\OperatorTok{$}\NormalTok{coefficients[}\DecValTok{1}\NormalTok{]}
\NormalTok{BGE.slp <-}\StringTok{ }\NormalTok{BGE.fit}\OperatorTok{$}\NormalTok{coefficients[}\DecValTok{2}\NormalTok{]}

\NormalTok{BP.R2}
\end{Highlighting}
\end{Shaded}

\begin{verbatim}
## [1] 0.36
\end{verbatim}

\begin{Shaded}
\begin{Highlighting}[]
\NormalTok{BR.R2}
\end{Highlighting}
\end{Shaded}

\begin{verbatim}
## [1] 0.69
\end{verbatim}

\begin{Shaded}
\begin{Highlighting}[]
\NormalTok{BGE.R2}
\end{Highlighting}
\end{Shaded}

\begin{verbatim}
## [1] 0.27
\end{verbatim}

\begin{Shaded}
\begin{Highlighting}[]
\NormalTok{BP.plot <-}\StringTok{ }\KeywordTok{ggplot}\NormalTok{(metab, }\KeywordTok{aes}\NormalTok{(}\DataTypeTok{x =}\NormalTok{ dist, }\DataTypeTok{y =}\NormalTok{ BP)) }\OperatorTok{+}\StringTok{ }
\StringTok{  }\KeywordTok{geom_point}\NormalTok{() }\OperatorTok{+}\StringTok{ }
\StringTok{  }\KeywordTok{geom_smooth}\NormalTok{(}\DataTypeTok{method =} \StringTok{"lm"}\NormalTok{, }\DataTypeTok{formula =}\NormalTok{ y }\OperatorTok{~}\StringTok{ }\NormalTok{x }\OperatorTok{+}\StringTok{ }\KeywordTok{I}\NormalTok{(x}\OperatorTok{^}\DecValTok{2}\NormalTok{), }\DataTypeTok{color =} \StringTok{"black"}\NormalTok{) }\OperatorTok{+}
\StringTok{  }\KeywordTok{annotate}\NormalTok{(}\DataTypeTok{geom =} \StringTok{"text"}\NormalTok{, }\DataTypeTok{x =} \DecValTok{350}\NormalTok{, }\DataTypeTok{y =} \FloatTok{1.5}\NormalTok{, }\DataTypeTok{size =} \DecValTok{5}\NormalTok{, }\DataTypeTok{hjust =} \DecValTok{1}\NormalTok{, }\DataTypeTok{vjust =} \DecValTok{1}\NormalTok{,}
           \DataTypeTok{label =} \KeywordTok{paste0}\NormalTok{(}\StringTok{"r^2== "}\NormalTok{,BP.R2), }\DataTypeTok{parse =}\NormalTok{ T) }\OperatorTok{+}
\StringTok{  }\KeywordTok{labs}\NormalTok{(}\DataTypeTok{y =} \KeywordTok{expression}\NormalTok{(}\KeywordTok{paste}\NormalTok{(}\StringTok{'BP ('}\NormalTok{, mu ,}\StringTok{'M C h'}\OperatorTok{^-}\DecValTok{1}\OperatorTok{*}\StringTok{ ')'}\NormalTok{)), }
       \DataTypeTok{x =} \StringTok{"Reservoir Transect (m)"}\NormalTok{)}
\NormalTok{BR.plot <-}\StringTok{ }\KeywordTok{ggplot}\NormalTok{(metab, }\KeywordTok{aes}\NormalTok{(}\DataTypeTok{x =}\NormalTok{ dist, }\DataTypeTok{y =}\NormalTok{ BR)) }\OperatorTok{+}\StringTok{ }
\StringTok{  }\KeywordTok{geom_point}\NormalTok{() }\OperatorTok{+}\StringTok{ }
\StringTok{  }\KeywordTok{geom_smooth}\NormalTok{(}\DataTypeTok{method =} \StringTok{"lm"}\NormalTok{, }\DataTypeTok{formula =}\NormalTok{ y }\OperatorTok{~}\StringTok{ }\NormalTok{x, }\DataTypeTok{color =} \StringTok{"black"}\NormalTok{) }\OperatorTok{+}\StringTok{ }
\StringTok{  }\KeywordTok{annotate}\NormalTok{(}\StringTok{"text"}\NormalTok{, }\DataTypeTok{x =} \DecValTok{350}\NormalTok{, }\DataTypeTok{y =} \FloatTok{1.5}\NormalTok{, }\DataTypeTok{size =} \DecValTok{5}\NormalTok{, }\DataTypeTok{hjust =} \DecValTok{1}\NormalTok{, }\DataTypeTok{vjust =} \DecValTok{0}\NormalTok{,}
           \DataTypeTok{label =} \KeywordTok{paste0}\NormalTok{(}\StringTok{"r^2== "}\NormalTok{,BR.R2), }\DataTypeTok{parse =}\NormalTok{ T ) }\OperatorTok{+}
\StringTok{  }\KeywordTok{labs}\NormalTok{(}\DataTypeTok{y =} \KeywordTok{expression}\NormalTok{(}\KeywordTok{paste}\NormalTok{(}\StringTok{'BR ('}\NormalTok{, mu ,}\StringTok{'M C h'}\OperatorTok{^-}\DecValTok{1}\OperatorTok{*}\StringTok{ ')'}\NormalTok{)), }
       \DataTypeTok{x =} \StringTok{"Reservoir Transect (m)"}\NormalTok{)}
\NormalTok{BGE.plot <-}\StringTok{ }\KeywordTok{ggplot}\NormalTok{(metab, }\KeywordTok{aes}\NormalTok{(}\DataTypeTok{x =}\NormalTok{ dist, }\DataTypeTok{y =}\NormalTok{ BGE)) }\OperatorTok{+}\StringTok{ }
\StringTok{  }\KeywordTok{geom_point}\NormalTok{() }\OperatorTok{+}\StringTok{ }
\StringTok{  }\KeywordTok{geom_smooth}\NormalTok{(}\DataTypeTok{method =} \StringTok{"lm"}\NormalTok{, }\DataTypeTok{formula =}\NormalTok{ y }\OperatorTok{~}\StringTok{ }\NormalTok{x }\OperatorTok{+}\StringTok{ }\KeywordTok{I}\NormalTok{(x}\OperatorTok{^}\DecValTok{2}\NormalTok{), }\DataTypeTok{color =} \StringTok{"black"}\NormalTok{) }\OperatorTok{+}
\StringTok{  }\KeywordTok{annotate}\NormalTok{(}\StringTok{"text"}\NormalTok{, }\DataTypeTok{x =} \DecValTok{350}\NormalTok{, }\DataTypeTok{y =} \FloatTok{.5}\NormalTok{, }\DataTypeTok{size =} \DecValTok{5}\NormalTok{, }\DataTypeTok{hjust =} \DecValTok{1}\NormalTok{, }\DataTypeTok{vjust =} \DecValTok{1}\NormalTok{,}
           \DataTypeTok{label =} \KeywordTok{paste0}\NormalTok{(}\StringTok{"R^2== "}\NormalTok{,BGE.R2), }\DataTypeTok{parse =}\NormalTok{ T ) }\OperatorTok{+}
\StringTok{  }\KeywordTok{labs}\NormalTok{(}\DataTypeTok{y =} \StringTok{"BGE"}\NormalTok{, }
       \DataTypeTok{x =} \StringTok{"Reservoir Transect (m)"}\NormalTok{)}
\end{Highlighting}
\end{Shaded}

\begin{Shaded}
\begin{Highlighting}[]
\KeywordTok{plot_grid}\NormalTok{(BP.plot }\OperatorTok{+}\StringTok{ }\KeywordTok{theme}\NormalTok{(}\DataTypeTok{axis.title.x =} \KeywordTok{element_blank}\NormalTok{(), }\DataTypeTok{axis.text.x =} \KeywordTok{element_blank}\NormalTok{(), }
                          \DataTypeTok{plot.margin =} \KeywordTok{unit}\NormalTok{(}\KeywordTok{c}\NormalTok{(}\DecValTok{1}\NormalTok{, }\DecValTok{1}\NormalTok{, }\DecValTok{-1}\NormalTok{, }\DecValTok{0}\NormalTok{), }\StringTok{"cm"}\NormalTok{)), }
\NormalTok{          BR.plot }\OperatorTok{+}\StringTok{ }\KeywordTok{theme}\NormalTok{(}\DataTypeTok{axis.title.x =} \KeywordTok{element_blank}\NormalTok{(), }\DataTypeTok{axis.text.x =} \KeywordTok{element_blank}\NormalTok{(),}
                          \DataTypeTok{plot.margin =} \KeywordTok{unit}\NormalTok{(}\KeywordTok{c}\NormalTok{(}\OperatorTok{-}\DecValTok{1}\NormalTok{, }\DecValTok{1}\NormalTok{, }\DecValTok{-1}\NormalTok{, }\DecValTok{0}\NormalTok{), }\StringTok{"cm"}\NormalTok{)), }
\NormalTok{          BGE.plot }\OperatorTok{+}\StringTok{ }\KeywordTok{theme}\NormalTok{(}\DataTypeTok{plot.margin =} \KeywordTok{unit}\NormalTok{(}\KeywordTok{c}\NormalTok{(}\OperatorTok{-}\DecValTok{1}\NormalTok{, }\DecValTok{1}\NormalTok{, }\DecValTok{0}\NormalTok{, }\DecValTok{0}\NormalTok{), }\StringTok{"cm"}\NormalTok{)), }
          \DataTypeTok{align =} \StringTok{"hv"}\NormalTok{, }\DataTypeTok{ncol =} \DecValTok{1}\NormalTok{, }\DataTypeTok{labels =} \StringTok{"auto"}\NormalTok{)}
\end{Highlighting}
\end{Shaded}

\begin{center}\includegraphics{ReservoirGradient_files/figure-latex/metab_plot-1} \end{center}

\hypertarget{relation-of-ecosystem-functions-and-community-structure}{%
\subsection{Relation of ecosystem functions and community
structure}\label{relation-of-ecosystem-functions-and-community-structure}}

\begin{Shaded}
\begin{Highlighting}[]
\NormalTok{metab.joined <-}\StringTok{ }\KeywordTok{cbind.data.frame}\NormalTok{(design.dna, metab[}\OperatorTok{-}\DecValTok{5}\NormalTok{,])}

\NormalTok{transient.metabolism <-}\StringTok{ }\KeywordTok{cbind.data.frame}\NormalTok{(}\DataTypeTok{transients =}\NormalTok{ terr.rich, metab.joined) }

\NormalTok{p1 <-}\StringTok{ }\NormalTok{transient.metabolism }\OperatorTok\StringTok{ }
\StringTok{  }\KeywordTok{ggplot}\NormalTok{(}\KeywordTok{aes}\NormalTok{(}\DataTypeTok{x=}\NormalTok{transients, }\DataTypeTok{y =}\NormalTok{ BP)) }\OperatorTok{+}
\StringTok{  }\KeywordTok{geom_smooth}\NormalTok{(}\DataTypeTok{color =} \StringTok{"black"}\NormalTok{) }\OperatorTok{+}
\StringTok{  }\KeywordTok{geom_point}\NormalTok{() }\OperatorTok{+}\StringTok{ }
\StringTok{  }\KeywordTok{scale_x_continuous}\NormalTok{(}\DataTypeTok{limits =} \KeywordTok{c}\NormalTok{(}\DecValTok{0}\NormalTok{, }\OtherTok{NA}\NormalTok{)) }\OperatorTok{+}
\StringTok{  }\KeywordTok{labs}\NormalTok{(}\DataTypeTok{x =} \StringTok{"Terrestrial-derived taxa"}\NormalTok{,}
       \DataTypeTok{y =} \KeywordTok{expression}\NormalTok{(}\KeywordTok{paste}\NormalTok{(}\StringTok{'BP ('}\NormalTok{, mu ,}\StringTok{'M C h'}\OperatorTok{^-}\DecValTok{1}\OperatorTok{*}\StringTok{ ')'}\NormalTok{))) }\OperatorTok{+}
\StringTok{  }\KeywordTok{theme}\NormalTok{(}\DataTypeTok{axis.title.x =} \KeywordTok{element_blank}\NormalTok{(), }
                          \DataTypeTok{plot.margin =} \KeywordTok{unit}\NormalTok{(}\KeywordTok{c}\NormalTok{(}\DecValTok{1}\NormalTok{, }\DecValTok{1}\NormalTok{, }\DecValTok{0}\NormalTok{, }\DecValTok{0}\NormalTok{), }\StringTok{"cm"}\NormalTok{))}
\NormalTok{p2 <-}\StringTok{ }\NormalTok{transient.metabolism }\OperatorTok\StringTok{ }
\StringTok{  }\KeywordTok{ggplot}\NormalTok{(}\KeywordTok{aes}\NormalTok{(}\DataTypeTok{x=}\NormalTok{transients, }\DataTypeTok{y =}\NormalTok{ BR)) }\OperatorTok{+}
\StringTok{  }\KeywordTok{geom_smooth}\NormalTok{(}\DataTypeTok{color =} \StringTok{"black"}\NormalTok{) }\OperatorTok{+}
\StringTok{  }\KeywordTok{geom_point}\NormalTok{() }\OperatorTok{+}\StringTok{ }
\StringTok{  }\KeywordTok{scale_x_continuous}\NormalTok{(}\DataTypeTok{limits =} \KeywordTok{c}\NormalTok{(}\DecValTok{0}\NormalTok{, }\OtherTok{NA}\NormalTok{)) }\OperatorTok{+}
\StringTok{  }\KeywordTok{labs}\NormalTok{(}\DataTypeTok{x =} \StringTok{"Terrestrial-derived taxa"}\NormalTok{,}
       \DataTypeTok{y =} \KeywordTok{expression}\NormalTok{(}\KeywordTok{paste}\NormalTok{(}\StringTok{'BR ('}\NormalTok{, mu ,}\StringTok{'M C h'}\OperatorTok{^-}\DecValTok{1}\OperatorTok{*}\StringTok{ ')'}\NormalTok{))) }\OperatorTok{+}
\StringTok{  }\KeywordTok{theme}\NormalTok{(}\DataTypeTok{axis.title.x =} \KeywordTok{element_blank}\NormalTok{(),}
                          \DataTypeTok{plot.margin =} \KeywordTok{unit}\NormalTok{(}\KeywordTok{c}\NormalTok{(}\DecValTok{0}\NormalTok{, }\DecValTok{1}\NormalTok{, }\DecValTok{0}\NormalTok{, }\DecValTok{0}\NormalTok{), }\StringTok{"cm"}\NormalTok{))}
\NormalTok{p3 <-}\StringTok{ }\NormalTok{transient.metabolism }\OperatorTok\StringTok{ }
\StringTok{  }\KeywordTok{ggplot}\NormalTok{(}\KeywordTok{aes}\NormalTok{(}\DataTypeTok{x=}\NormalTok{transients, }\DataTypeTok{y =}\NormalTok{ BGE)) }\OperatorTok{+}
\StringTok{  }\KeywordTok{geom_smooth}\NormalTok{(}\DataTypeTok{color =} \StringTok{"black"}\NormalTok{) }\OperatorTok{+}
\StringTok{  }\KeywordTok{geom_point}\NormalTok{() }\OperatorTok{+}\StringTok{ }
\StringTok{  }\KeywordTok{scale_x_continuous}\NormalTok{(}\DataTypeTok{limits =} \KeywordTok{c}\NormalTok{(}\DecValTok{0}\NormalTok{, }\OtherTok{NA}\NormalTok{)) }\OperatorTok{+}
\StringTok{  }\KeywordTok{labs}\NormalTok{(}\DataTypeTok{x =} \StringTok{"Terrestrial-derived taxa"}\NormalTok{) }\OperatorTok{+}
\StringTok{  }\KeywordTok{theme}\NormalTok{(}\DataTypeTok{plot.margin =} \KeywordTok{unit}\NormalTok{(}\KeywordTok{c}\NormalTok{(}\DecValTok{0}\NormalTok{, }\DecValTok{1}\NormalTok{, }\DecValTok{0}\NormalTok{, }\DecValTok{0}\NormalTok{), }\StringTok{"cm"}\NormalTok{))}

\KeywordTok{plot_grid}\NormalTok{(p1, }\OtherTok{NULL}\NormalTok{, p2, }\OtherTok{NULL}\NormalTok{, p3, }
          \DataTypeTok{rel_heights =} \KeywordTok{c}\NormalTok{(}\DecValTok{1}\NormalTok{, }\FloatTok{-.15}\NormalTok{, }\DecValTok{1}\NormalTok{, }\FloatTok{-.15}\NormalTok{, }\DecValTok{1}\NormalTok{), }\DataTypeTok{align =} \StringTok{"hv"}\NormalTok{, }
          \DataTypeTok{ncol =} \DecValTok{1}\NormalTok{, }\DataTypeTok{labels =} \KeywordTok{c}\NormalTok{(}\StringTok{"a"}\NormalTok{, }\StringTok{"NULL"}\NormalTok{, }\StringTok{"b"}\NormalTok{, }\StringTok{"NULL"}\NormalTok{, }\StringTok{"c"}\NormalTok{)) }\OperatorTok{+}
\StringTok{  }\KeywordTok{ggsave}\NormalTok{(}\StringTok{"figures/functions.pdf"}\NormalTok{)}
\end{Highlighting}
\end{Shaded}

\begin{center}\includegraphics{ReservoirGradient_files/figure-latex/unnamed-chunk-10-1} \end{center}

\begin{Shaded}
\begin{Highlighting}[]
\CommentTok{# identify otus in soil samples and lake samples}
\NormalTok{in.soil <-}\StringTok{ }\NormalTok{OTUs[, }\KeywordTok{which}\NormalTok{(}\KeywordTok{colSums}\NormalTok{(OTUs[}\KeywordTok{c}\NormalTok{(}\DecValTok{1}\OperatorTok{:}\DecValTok{3}\NormalTok{),]) }\OperatorTok{>}\StringTok{ }\DecValTok{0}\NormalTok{ )]}

\CommentTok{# isolate just the rna water samples and convert to presence-absence}
\NormalTok{in.lake.rna <-}\StringTok{ }\NormalTok{OTUs[}\KeywordTok{which}\NormalTok{(design}\OperatorTok{$}\NormalTok{molecule }\OperatorTok{==}\StringTok{ "RNA"} \OperatorTok{&}\StringTok{ }\NormalTok{design}\OperatorTok{$}\NormalTok{type }\OperatorTok{==}\StringTok{ "water"}\NormalTok{), ]}
\NormalTok{in.lake.rna.pa <-}\StringTok{ }\NormalTok{(in.lake.rna }\OperatorTok{>}\StringTok{ }\DecValTok{0}\NormalTok{) }\OperatorTok{*}\StringTok{ }\DecValTok{1}

\NormalTok{threshlist <-}\StringTok{ }\KeywordTok{c}\NormalTok{(.}\DecValTok{3}\NormalTok{, }\FloatTok{.4}\NormalTok{, }\FloatTok{.5}\NormalTok{, }\FloatTok{.6}\NormalTok{, }\FloatTok{.7}\NormalTok{, }\FloatTok{.8}\NormalTok{, }\FloatTok{.9}\NormalTok{)}
\NormalTok{df.plot <-}\StringTok{ }\KeywordTok{data.frame}\NormalTok{()}
\ControlFlowTok{for}\NormalTok{(thresh }\ControlFlowTok{in}\NormalTok{ threshlist)\{}
  \CommentTok{# define the 'core' taxa as otus present in 50% of samples}
\NormalTok{in.lake.core <-}\StringTok{ }\NormalTok{w.dna[, }\KeywordTok{which}\NormalTok{((}\KeywordTok{colSums}\NormalTok{(in.lake.rna.pa) }\OperatorTok{/}\StringTok{ }\KeywordTok{nrow}\NormalTok{(in.lake.rna.pa)) }\OperatorTok{>=}\StringTok{ }\NormalTok{thresh)]}

\CommentTok{# of the core, how many are also in the soil samples?}
\NormalTok{in.lake.core.from.soils <-}\StringTok{ }\NormalTok{in.lake.core[, }\KeywordTok{intersect}\NormalTok{(}\KeywordTok{colnames}\NormalTok{(in.lake.core), }\KeywordTok{colnames}\NormalTok{(in.soil))]}

\CommentTok{# of the core which are not in the soil samples}
\NormalTok{in.lake.core.not.soils <-}\StringTok{ }\NormalTok{in.lake.core[, }\KeywordTok{setdiff}\NormalTok{(}\KeywordTok{colnames}\NormalTok{(in.lake.core), }\KeywordTok{colnames}\NormalTok{(in.soil))]}

\CommentTok{# Find the relative abundance of the core taxa and prepare data frame to plot}
\NormalTok{in.lake.core.from.soils.REL <-}\StringTok{ }\NormalTok{in.lake.core.from.soils }\OperatorTok{/}\StringTok{ }\KeywordTok{rowSums}\NormalTok{(w.dna)}

\NormalTok{in.soil.to.plot <-}\StringTok{ }\KeywordTok{as.data.frame}\NormalTok{(in.lake.core.from.soils.REL) }\OperatorTok\StringTok{ }
\StringTok{  }\KeywordTok{rownames_to_column}\NormalTok{(}\StringTok{"sample_ID"}\NormalTok{) }\OperatorTok\StringTok{ }
\StringTok{  }\KeywordTok{gather}\NormalTok{(otu_id, rel_abundance, }\OperatorTok{-}\NormalTok{sample_ID) }\OperatorTok\StringTok{ }
\StringTok{  }\KeywordTok{left_join}\NormalTok{(}\KeywordTok{rownames_to_column}\NormalTok{(design.dna, }\StringTok{"sample_ID"}\NormalTok{)) }\OperatorTok\StringTok{ }
\StringTok{  }\KeywordTok{add_column}\NormalTok{(}\DataTypeTok{found =} \StringTok{"soils"}\NormalTok{)}

\NormalTok{in.lake.core.not.soils.REL <-}\StringTok{ }\NormalTok{in.lake.core.not.soils }\OperatorTok{/}\StringTok{ }\KeywordTok{rowSums}\NormalTok{(w.dna)}

\NormalTok{in.lake.to.plot <-}\StringTok{ }\KeywordTok{as.data.frame}\NormalTok{(in.lake.core.not.soils.REL) }\OperatorTok\StringTok{ }
\StringTok{  }\KeywordTok{rownames_to_column}\NormalTok{(}\StringTok{"sample_ID"}\NormalTok{) }\OperatorTok\StringTok{ }
\StringTok{  }\KeywordTok{gather}\NormalTok{(otu_id, rel_abundance, }\OperatorTok{-}\NormalTok{sample_ID) }\OperatorTok\StringTok{ }
\StringTok{  }\KeywordTok{left_join}\NormalTok{(}\KeywordTok{rownames_to_column}\NormalTok{(design.dna, }\StringTok{"sample_ID"}\NormalTok{)) }\OperatorTok\StringTok{ }
\StringTok{  }\KeywordTok{add_column}\NormalTok{(}\DataTypeTok{found =} \StringTok{"lake"}\NormalTok{)}

\CommentTok{# model distance effect on rel abundance to get slope and pval}
\NormalTok{soil.core.mods <-}\StringTok{ }\KeywordTok{apply}\NormalTok{(in.lake.core.from.soils.REL, }\DataTypeTok{MARGIN =} \DecValTok{2}\NormalTok{, }
    \DataTypeTok{FUN =} \ControlFlowTok{function}\NormalTok{(x) }\KeywordTok{summary}\NormalTok{(}\KeywordTok{lm}\NormalTok{(x }\OperatorTok{~}\StringTok{ }\NormalTok{design.dna}\OperatorTok{$}\NormalTok{distance))}\OperatorTok{$}\NormalTok{coefficients[}\DecValTok{2}\NormalTok{,}\KeywordTok{c}\NormalTok{(}\DecValTok{1}\NormalTok{,}\DecValTok{4}\NormalTok{)])}
\KeywordTok{rownames}\NormalTok{(soil.core.mods) <-}\StringTok{ }\KeywordTok{c}\NormalTok{(}\StringTok{"slope"}\NormalTok{, }\StringTok{"pval"}\NormalTok{)}

\CommentTok{# classify otus as significantly increasing or decreasing along reservoir}
\NormalTok{soil.core.decreasing <-}\StringTok{ }\KeywordTok{as.data.frame}\NormalTok{(}\KeywordTok{t}\NormalTok{(soil.core.mods)) }\OperatorTok\StringTok{ }
\StringTok{  }\KeywordTok{rownames_to_column}\NormalTok{(}\StringTok{"OTU"}\NormalTok{) }\OperatorTok\StringTok{ }
\StringTok{  }\KeywordTok{filter}\NormalTok{(slope }\OperatorTok{<}\StringTok{ }\DecValTok{0}\NormalTok{) }\OperatorTok\StringTok{   }\CommentTok{# rel abund decreases toward dam}
\StringTok{  }\KeywordTok{left_join}\NormalTok{(OTU.tax)}
\NormalTok{soil.core.increasing <-}\StringTok{ }\KeywordTok{as.data.frame}\NormalTok{(}\KeywordTok{t}\NormalTok{(soil.core.mods)) }\OperatorTok\StringTok{ }
\StringTok{  }\KeywordTok{rownames_to_column}\NormalTok{(}\StringTok{"OTU"}\NormalTok{) }\OperatorTok\StringTok{ }
\StringTok{  }\KeywordTok{filter}\NormalTok{(slope }\OperatorTok{>}\StringTok{ }\DecValTok{0}\NormalTok{) }\OperatorTok\StringTok{   }\CommentTok{# rel abund increases toward dam}
\StringTok{  }\KeywordTok{left_join}\NormalTok{(OTU.tax)}

\NormalTok{nonsoil.core.mods <-}\StringTok{ }\KeywordTok{apply}\NormalTok{(in.lake.core.not.soils.REL, }\DataTypeTok{MARGIN =} \DecValTok{2}\NormalTok{, }
    \DataTypeTok{FUN =} \ControlFlowTok{function}\NormalTok{(x) }\KeywordTok{summary}\NormalTok{(}\KeywordTok{lm}\NormalTok{(x }\OperatorTok{~}\StringTok{ }\NormalTok{design.dna}\OperatorTok{$}\NormalTok{distance))}\OperatorTok{$}\NormalTok{coefficients[}\DecValTok{2}\NormalTok{,}\KeywordTok{c}\NormalTok{(}\DecValTok{1}\NormalTok{,}\DecValTok{4}\NormalTok{)])}
\KeywordTok{rownames}\NormalTok{(nonsoil.core.mods) <-}\StringTok{ }\KeywordTok{c}\NormalTok{(}\StringTok{"slope"}\NormalTok{, }\StringTok{"pval"}\NormalTok{)}
\NormalTok{nonsoil.core.decreasing <-}\StringTok{ }\KeywordTok{as.data.frame}\NormalTok{(}\KeywordTok{t}\NormalTok{(nonsoil.core.mods)) }\OperatorTok\StringTok{ }
\StringTok{  }\KeywordTok{rownames_to_column}\NormalTok{(}\StringTok{"OTU"}\NormalTok{) }\OperatorTok\StringTok{ }
\StringTok{  }\KeywordTok{filter}\NormalTok{(slope }\OperatorTok{<}\StringTok{ }\DecValTok{0}\NormalTok{) }\OperatorTok\StringTok{   }\CommentTok{# rel abund decreases toward dam}
\StringTok{  }\KeywordTok{left_join}\NormalTok{(OTU.tax)}
\NormalTok{nonsoil.core.increasing <-}\StringTok{ }\KeywordTok{as.data.frame}\NormalTok{(}\KeywordTok{t}\NormalTok{(nonsoil.core.mods)) }\OperatorTok\StringTok{ }
\StringTok{  }\KeywordTok{rownames_to_column}\NormalTok{(}\StringTok{"OTU"}\NormalTok{) }\OperatorTok\StringTok{ }
\StringTok{  }\KeywordTok{filter}\NormalTok{(slope }\OperatorTok{>}\StringTok{ }\DecValTok{0}\NormalTok{) }\OperatorTok\StringTok{   }\CommentTok{# rel abund increases toward dam}
\StringTok{  }\KeywordTok{left_join}\NormalTok{(OTU.tax)}

\NormalTok{df1 <-}\StringTok{ }\KeywordTok{as.data.frame}\NormalTok{(OTUsREL[,nonsoil.core.increasing}\OperatorTok{$}\NormalTok{OTU]) }\OperatorTok\StringTok{ }
\StringTok{  }\KeywordTok{rownames_to_column}\NormalTok{(}\StringTok{"sampleID"}\NormalTok{) }\OperatorTok\StringTok{ }
\StringTok{  }\KeywordTok{left_join}\NormalTok{(}\KeywordTok{rownames_to_column}\NormalTok{(design, }\StringTok{"sampleID"}\NormalTok{)) }\OperatorTok\StringTok{ }
\StringTok{  }\KeywordTok{gather}\NormalTok{(OTU, rel_abund, }\OperatorTok{-}\NormalTok{station, }\OperatorTok{-}\NormalTok{molecule, }\OperatorTok{-}\NormalTok{type, }\OperatorTok{-}\NormalTok{distance, }\OperatorTok{-}\NormalTok{sampleID) }\OperatorTok\StringTok{ }
\StringTok{  }\KeywordTok{filter}\NormalTok{(molecule }\OperatorTok{==}\StringTok{ "DNA"}\NormalTok{) }\OperatorTok\StringTok{ }\KeywordTok{left_join}\NormalTok{(OTU.tax) }\OperatorTok\StringTok{ }
\StringTok{  }\KeywordTok{mutate}\NormalTok{(}\DataTypeTok{soils =} \StringTok{"Absent from soils"}\NormalTok{, }\DataTypeTok{change =} \StringTok{"Increasing"}\NormalTok{)}
\NormalTok{n1 <-}\StringTok{ }\KeywordTok{length}\NormalTok{(}\KeywordTok{unique}\NormalTok{(df1}\OperatorTok{$}\NormalTok{OTU))}

\NormalTok{df2 <-}\StringTok{ }\KeywordTok{as.data.frame}\NormalTok{(OTUsREL[,soil.core.increasing}\OperatorTok{$}\NormalTok{OTU]) }\OperatorTok\StringTok{ }
\StringTok{  }\KeywordTok{rownames_to_column}\NormalTok{(}\StringTok{"sampleID"}\NormalTok{) }\OperatorTok\StringTok{ }
\StringTok{  }\KeywordTok{left_join}\NormalTok{(}\KeywordTok{rownames_to_column}\NormalTok{(design, }\StringTok{"sampleID"}\NormalTok{)) }\OperatorTok\StringTok{ }
\StringTok{  }\KeywordTok{gather}\NormalTok{(OTU, rel_abund, }\OperatorTok{-}\NormalTok{station, }\OperatorTok{-}\NormalTok{molecule, }\OperatorTok{-}\NormalTok{type, }\OperatorTok{-}\NormalTok{distance, }\OperatorTok{-}\NormalTok{sampleID) }\OperatorTok\StringTok{ }
\StringTok{  }\KeywordTok{filter}\NormalTok{(molecule }\OperatorTok{==}\StringTok{ "DNA"}\NormalTok{) }\OperatorTok\StringTok{ }\KeywordTok{left_join}\NormalTok{(OTU.tax) }\OperatorTok\StringTok{ }
\StringTok{  }\KeywordTok{mutate}\NormalTok{(}\DataTypeTok{soils =} \StringTok{"Present in soils"}\NormalTok{, }\DataTypeTok{change =} \StringTok{"Increasing"}\NormalTok{)}
\NormalTok{n2 <-}\StringTok{ }\KeywordTok{length}\NormalTok{(}\KeywordTok{unique}\NormalTok{(df2}\OperatorTok{$}\NormalTok{OTU))}

\NormalTok{df3 <-}\StringTok{ }\KeywordTok{as.data.frame}\NormalTok{(OTUsREL[,soil.core.decreasing}\OperatorTok{$}\NormalTok{OTU]) }\OperatorTok\StringTok{ }
\StringTok{  }\KeywordTok{rownames_to_column}\NormalTok{(}\StringTok{"sampleID"}\NormalTok{) }\OperatorTok\StringTok{ }
\StringTok{  }\KeywordTok{left_join}\NormalTok{(}\KeywordTok{rownames_to_column}\NormalTok{(design, }\StringTok{"sampleID"}\NormalTok{)) }\OperatorTok\StringTok{ }
\StringTok{  }\KeywordTok{gather}\NormalTok{(OTU, rel_abund, }\OperatorTok{-}\NormalTok{station, }\OperatorTok{-}\NormalTok{molecule, }\OperatorTok{-}\NormalTok{type, }\OperatorTok{-}\NormalTok{distance, }\OperatorTok{-}\NormalTok{sampleID) }\OperatorTok\StringTok{ }
\StringTok{  }\KeywordTok{filter}\NormalTok{(molecule }\OperatorTok{==}\StringTok{ "DNA"}\NormalTok{) }\OperatorTok\StringTok{ }\KeywordTok{left_join}\NormalTok{(OTU.tax) }\OperatorTok\StringTok{ }
\StringTok{  }\KeywordTok{mutate}\NormalTok{(}\DataTypeTok{soils =} \StringTok{"Present in soils"}\NormalTok{, }\DataTypeTok{change =} \StringTok{"Decreasing"}\NormalTok{)}
\NormalTok{n3 <-}\StringTok{ }\KeywordTok{length}\NormalTok{(}\KeywordTok{unique}\NormalTok{(df3}\OperatorTok{$}\NormalTok{OTU))}

\NormalTok{df4 <-}\StringTok{ }\KeywordTok{as.data.frame}\NormalTok{(OTUsREL[,nonsoil.core.decreasing}\OperatorTok{$}\NormalTok{OTU]) }\OperatorTok\StringTok{ }
\StringTok{  }\KeywordTok{rownames_to_column}\NormalTok{(}\StringTok{"sampleID"}\NormalTok{) }\OperatorTok\StringTok{ }
\StringTok{  }\KeywordTok{left_join}\NormalTok{(}\KeywordTok{rownames_to_column}\NormalTok{(design, }\StringTok{"sampleID"}\NormalTok{)) }\OperatorTok\StringTok{ }
\StringTok{  }\KeywordTok{gather}\NormalTok{(OTU, rel_abund, }\OperatorTok{-}\NormalTok{station, }\OperatorTok{-}\NormalTok{molecule, }\OperatorTok{-}\NormalTok{type, }\OperatorTok{-}\NormalTok{distance, }\OperatorTok{-}\NormalTok{sampleID) }\OperatorTok\StringTok{ }
\StringTok{  }\KeywordTok{filter}\NormalTok{(molecule }\OperatorTok{==}\StringTok{ "DNA"}\NormalTok{) }\OperatorTok\StringTok{ }\KeywordTok{left_join}\NormalTok{(OTU.tax) }\OperatorTok\StringTok{ }
\StringTok{  }\KeywordTok{mutate}\NormalTok{(}\DataTypeTok{soils =} \StringTok{"Absent from soils"}\NormalTok{, }\DataTypeTok{change =} \StringTok{"Decreasing"}\NormalTok{)}
\NormalTok{n4 <-}\StringTok{ }\KeywordTok{length}\NormalTok{(}\KeywordTok{unique}\NormalTok{(df4}\OperatorTok{$}\NormalTok{OTU))}


\NormalTok{df.plot <-}\StringTok{ }\KeywordTok{as_tibble}\NormalTok{(}\KeywordTok{rbind.data.frame}\NormalTok{(df1, df2, df3, df4)) }\OperatorTok\StringTok{ }
\StringTok{  }\KeywordTok{mutate}\NormalTok{(}\DataTypeTok{thresh =}\NormalTok{ thresh) }\OperatorTok\StringTok{ }\KeywordTok{filter}\NormalTok{(type }\OperatorTok{==}\StringTok{ "water"}\NormalTok{) }\OperatorTok\StringTok{ }
\StringTok{  }\KeywordTok{bind_rows}\NormalTok{(df.plot)}

\NormalTok{\}}
\end{Highlighting}
\end{Shaded}

\begin{verbatim}
## Warning: Column `OTU` joining character vector and factor, coercing into
## character vector

## Warning: Column `OTU` joining character vector and factor, coercing into
## character vector

## Warning: Column `OTU` joining character vector and factor, coercing into
## character vector

## Warning: Column `OTU` joining character vector and factor, coercing into
## character vector

## Warning: Column `OTU` joining character vector and factor, coercing into
## character vector

## Warning: Column `OTU` joining character vector and factor, coercing into
## character vector

## Warning: Column `OTU` joining character vector and factor, coercing into
## character vector

## Warning: Column `OTU` joining character vector and factor, coercing into
## character vector

## Warning: Column `OTU` joining character vector and factor, coercing into
## character vector

## Warning: Column `OTU` joining character vector and factor, coercing into
## character vector

## Warning: Column `OTU` joining character vector and factor, coercing into
## character vector

## Warning: Column `OTU` joining character vector and factor, coercing into
## character vector

## Warning: Column `OTU` joining character vector and factor, coercing into
## character vector

## Warning: Column `OTU` joining character vector and factor, coercing into
## character vector

## Warning: Column `OTU` joining character vector and factor, coercing into
## character vector

## Warning: Column `OTU` joining character vector and factor, coercing into
## character vector

## Warning: Column `OTU` joining character vector and factor, coercing into
## character vector

## Warning: Column `OTU` joining character vector and factor, coercing into
## character vector

## Warning: Column `OTU` joining character vector and factor, coercing into
## character vector

## Warning: Column `OTU` joining character vector and factor, coercing into
## character vector

## Warning: Column `OTU` joining character vector and factor, coercing into
## character vector

## Warning: Column `OTU` joining character vector and factor, coercing into
## character vector

## Warning: Column `OTU` joining character vector and factor, coercing into
## character vector

## Warning: Column `OTU` joining character vector and factor, coercing into
## character vector

## Warning: Column `OTU` joining character vector and factor, coercing into
## character vector

## Warning: Column `OTU` joining character vector and factor, coercing into
## character vector

## Warning: Column `OTU` joining character vector and factor, coercing into
## character vector

## Warning: Column `OTU` joining character vector and factor, coercing into
## character vector

## Warning: Column `OTU` joining character vector and factor, coercing into
## character vector

## Warning: Column `OTU` joining character vector and factor, coercing into
## character vector

## Warning: Column `OTU` joining character vector and factor, coercing into
## character vector

## Warning: Column `OTU` joining character vector and factor, coercing into
## character vector

## Warning: Column `OTU` joining character vector and factor, coercing into
## character vector

## Warning: Column `OTU` joining character vector and factor, coercing into
## character vector

## Warning: Column `OTU` joining character vector and factor, coercing into
## character vector

## Warning: Column `OTU` joining character vector and factor, coercing into
## character vector

## Warning: Column `OTU` joining character vector and factor, coercing into
## character vector

## Warning: Column `OTU` joining character vector and factor, coercing into
## character vector

## Warning: Column `OTU` joining character vector and factor, coercing into
## character vector

## Warning: Column `OTU` joining character vector and factor, coercing into
## character vector

## Warning: Column `OTU` joining character vector and factor, coercing into
## character vector

## Warning: Column `OTU` joining character vector and factor, coercing into
## character vector

## Warning: Column `OTU` joining character vector and factor, coercing into
## character vector

## Warning: Column `OTU` joining character vector and factor, coercing into
## character vector

## Warning: Column `OTU` joining character vector and factor, coercing into
## character vector

## Warning: Column `OTU` joining character vector and factor, coercing into
## character vector

## Warning: Column `OTU` joining character vector and factor, coercing into
## character vector

## Warning: Column `OTU` joining character vector and factor, coercing into
## character vector

## Warning: Column `OTU` joining character vector and factor, coercing into
## character vector

## Warning: Column `OTU` joining character vector and factor, coercing into
## character vector

## Warning: Column `OTU` joining character vector and factor, coercing into
## character vector

## Warning: Column `OTU` joining character vector and factor, coercing into
## character vector

## Warning: Column `OTU` joining character vector and factor, coercing into
## character vector

## Warning: Column `OTU` joining character vector and factor, coercing into
## character vector

## Warning: Column `OTU` joining character vector and factor, coercing into
## character vector

## Warning: Column `OTU` joining character vector and factor, coercing into
## character vector
\end{verbatim}

\begin{Shaded}
\begin{Highlighting}[]
\NormalTok{taxon_fate.plot <-}\StringTok{ }\NormalTok{df.plot }\OperatorTok\StringTok{ }\KeywordTok{mutate}\NormalTok{(}\DataTypeTok{rel_abund =} \KeywordTok{ifelse}\NormalTok{(rel_abund }\OperatorTok{==}\StringTok{ }\DecValTok{0}\NormalTok{, }\FloatTok{1e-6}\NormalTok{, rel_abund)) }\OperatorTok\StringTok{ }
\StringTok{  }\KeywordTok{filter}\NormalTok{(soils }\OperatorTok{==}\StringTok{ "Present in soils"}\NormalTok{) }\OperatorTok\StringTok{ }
\StringTok{  }\CommentTok{#mutate(change = ifelse(change == "Increasing", }
\StringTok{  }\CommentTok{#                       paste0("Increasing (n = ", n2,")"),}
\StringTok{  }\CommentTok{#                       paste0("Decreasing (n = ", n3,")"))) %>% }
\StringTok{  }\KeywordTok{ggplot}\NormalTok{(}\KeywordTok{aes}\NormalTok{(}\DataTypeTok{x =}\NormalTok{ distance, }\DataTypeTok{y =}\NormalTok{ rel_abund, }\DataTypeTok{group =}\NormalTok{ OTU)) }\OperatorTok{+}\StringTok{ }
\StringTok{  }\CommentTok{#geom_jitter(alpha = 0.15) + }
\StringTok{  }\KeywordTok{geom_line}\NormalTok{(}\DataTypeTok{stat =} \StringTok{"smooth"}\NormalTok{, }\DataTypeTok{alpha =} \FloatTok{0.3}\NormalTok{, }\DataTypeTok{size =} \FloatTok{.5}\NormalTok{,}
            \DataTypeTok{method =} \StringTok{"loess"}\NormalTok{, }\DataTypeTok{span =} \FloatTok{.7}\NormalTok{, }\DataTypeTok{se =} \OtherTok{FALSE}\NormalTok{) }\OperatorTok{+}
\StringTok{  }\KeywordTok{scale_y_log10}\NormalTok{(}\DataTypeTok{labels =}\NormalTok{ scales}\OperatorTok{::}\NormalTok{scientific) }\OperatorTok{+}
\StringTok{  }\KeywordTok{scale_x_continuous}\NormalTok{(}\DataTypeTok{limits =} \KeywordTok{c}\NormalTok{(}\DecValTok{0}\NormalTok{,}\DecValTok{380}\NormalTok{)) }\OperatorTok{+}
\StringTok{  }\KeywordTok{facet_wrap}\NormalTok{(}\OperatorTok{~}\NormalTok{thresh) }\OperatorTok{+}
\StringTok{  }\CommentTok{#theme(legend.position = "none") +}
\StringTok{  }\CommentTok{#guides(color = guide_legend(ncol = 1)) +}
\StringTok{  }\KeywordTok{labs}\NormalTok{(}\DataTypeTok{x =} \StringTok{"Reservoir distance (m)"}\NormalTok{,}
       \DataTypeTok{y =} \StringTok{"Active relative abundance"}\NormalTok{) }\OperatorTok{+}
\StringTok{  }\CommentTok{# annotate("text", x = 365, y = 1e-1, size = 5, hjust = 1, vjust = 1, angle = 90,}
\StringTok{  }\CommentTok{#          label = "Maintained") +}
\StringTok{  }\CommentTok{# annotate("text", x = 365, y = 1e-5, size = 5, hjust = 0.5, vjust = 1, angle = 90,}
\StringTok{  }\CommentTok{#          label = "Decaying") +}
\StringTok{  }\KeywordTok{ggsave}\NormalTok{(}\StringTok{"figures/taxa_origins_threshold.pdf"}\NormalTok{, }\DataTypeTok{width =} \DecValTok{8}\NormalTok{, }\DataTypeTok{height =} \DecValTok{6}\NormalTok{, }\DataTypeTok{units =} \StringTok{"in"}\NormalTok{)}
\NormalTok{taxon_fate.plot}
\end{Highlighting}
\end{Shaded}

\begin{center}\includegraphics{ReservoirGradient_files/figure-latex/unnamed-chunk-11-1} \end{center}


\end{document}
